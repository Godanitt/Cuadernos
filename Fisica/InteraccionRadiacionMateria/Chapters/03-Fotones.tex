\chapter{Fotones}

\section{Sección eficaz y coeficiente de atenuación}

\section{Efecto Fotoeléctrico}

\section{Efecto Compton}

El efecto Compton se define como la dispersión inelástica de un fotón con un electrón.

\subsection{Diferencia de longitud de onda}

Una de las característiccas que definen el efecto Compton es que el fotón, tras la colisión, sale con una longitud de onda mayor que la incidente, con una diferencia dependiente únicamente del ángulo de salida del electrón $\theta$, tal que 

\begin{equation}
    \Delta \lambda = \lambda' - \lambda = \lambda_C (1-\cos \theta) \qquad \lambda_C = \frac{h}{m_ec}  \qquad \lambda_C = 2.43 \ \unit{fm}
\end{equation}
siendo $\lambda_C$ la \textbf{longitud de onda Compton}. Esta fórmula se deduce directamente de la conservación del cuadrimomento $p^\mu$ en la colsión, tal que si ($c=1$, para el cálculo luego para recuperar la expresión lo haremos mediante las unidades):

\begin{equation}
    p_{i,\gamma}^\mu = \pqty{h \nu, \pn_i } \qquad 
    p_{i,e}^\mu = \pqty{m_e, 0} \qquad 
    p_{f,\gamma}^\mu = \pqty{h \nu, \pn_f } \qquad 
    p_{f,e}^\mu = \pqty{E_e, \qn_f }
\end{equation}
tenemos que se verifica la triple igualdad

\begin{eqnarray}
    h \nu + m_e = E_k + m_e + h \nu' \\
    \frac{h}{\lambda}= \frac{h}{\lambda'}\cos \theta + q_f \cos \phi  \\
    0 = \frac{h}{\lambda'} \sin \theta + q_f \sin \phi
\end{eqnarray}
Mandando todos los términos a la izquierda de las ecuaciones del momento, y elevando al cuadrado obtenemos: 
\begin{eqnarray}
    \pqty{\frac{h}{\lambda}- \frac{h}{\lambda'}\cos \theta}^2= \pqty{q_f \cos \phi}^2  \\
    \pqty{\frac{h}{\lambda'} \sin \theta}^2 = \pqty{q_f \sin \phi}^2
\end{eqnarray}
tal que al sumarse: 
\begin{equation}
    \pqty{\frac{h}{\lambda}- \frac{h}{\lambda'}\cos \theta}^2+ \pqty{\frac{h}{\lambda'} \sin \theta}^2 = (q_f)^2
\end{equation}
tal que se reduce a: 

\begin{equation}
    \pqty{\frac{h}{\lambda}}^2 + \pqty{\frac{h}{\lambda'}}^2 
    - 2 \frac{h^2}{\lambda \lambda'} \cos \theta = (q_f)^2
\end{equation}
y luego: 

\begin{equation}
    \lambda'^2 + \lambda^2 - 2 \lambda\lambda' \cos \theta = \lambda^2 \lambda'^2 \pqty{\frac{q_f}{h}}^2
\end{equation}
y la energía: 

\begin{equation}
    h \nu + m_e = E_k + m_e + h \nu' \Rightarrow  h(\lambda'-\lambda) =\lambda \lambda' E_k \Rightarrow \frac{h(\lambda'-\lambda)}{\lambda \lambda'} + m_e = E_{tot}
\end{equation}
\begin{equation}
    q_f^2 = \pqty{\frac{h(\lambda'-\lambda)}{\lambda \lambda'} + m_e }^2 - m_e^2 
\end{equation}
Ahora, al sustituir: 

\begin{equation}
    \lambda'^2 + \lambda^2 - 2 \lambda\lambda' \cos \theta = \frac{\lambda^2 \lambda'^2}{h^2} \bqty{\pqty{\frac{h(\lambda'-\lambda)}{\lambda \lambda'} + m_e }^2 - m_e^2 }
\end{equation}
Ahora: 

\begin{equation}
    \lambda'^2 + \lambda^2 - 2 \lambda \lambda' \cos \theta = (\lambda'- \lambda)^2 + \frac{2 m_e}{h}  \lambda \lambda' (\lambda'-\lambda)
\end{equation}
De lo que se deduce directamente: 

\begin{equation}
    2\lambda \lambda' (1-\cos \theta) = \frac{2 m_e}{h}  \lambda \lambda' (\lambda'-\lambda)
\end{equation}
Finalmente obteniendo 

\begin{equation}
     (\lambda'-\lambda)= \frac{h}{m_e}(1-\cos \theta) 
\end{equation}
y finalmente deshaciendo el $c=1$ tenemos 

\begin{equation}
    \Delta \lambda = \lambda'-\lambda = \frac{h}{cm_e}(1-\cos \theta) 
\end{equation}

\subsection{Energía media del electrón}

La fracción de energía transferida al electrón $\bar{f}_C$ es: 

\begin{equation} \scriptsize
\bar{f}_C(\varepsilon) =
\frac{
\left\{
\dfrac{2(1+\varepsilon)^2}{\varepsilon^2(1+2\varepsilon)}
- \dfrac{1+3\varepsilon}{(1+2\varepsilon)^2}
- \dfrac{(1+\varepsilon)(2\varepsilon^2 - 2\varepsilon - 1)}{\varepsilon^2(1+2\varepsilon)^2}
- \dfrac{4\varepsilon^2}{3(1+2\varepsilon)^3}
- \left[\dfrac{1+\varepsilon}{\varepsilon^3} - \dfrac{1}{2\varepsilon} + \dfrac{1}{2\varepsilon^3}\right] \log(1+2\varepsilon)
\right\}
}{
\left\{
\dfrac{1+\varepsilon}{\varepsilon^2} \left[\dfrac{2(1+\varepsilon)}{1+2\varepsilon} - \dfrac{\log(1+2\varepsilon)}{\varepsilon}\right]
+ \dfrac{\log(1+2\varepsilon)}{2\varepsilon}
- \dfrac{1+3\varepsilon}{(1+2\varepsilon)^2}
\right\}
}
\end{equation}
Siendo, en el límite ultrarrelativista $\varepsilon\gg 1$


\begin{equation} 
\bar{f}_C(\varepsilon) = \frac{\log (2\varepsilon)-0.82}{\log(2\varepsilon)+0.5}
\end{equation}

\section{Producción de pares en el campo núcleo}

\begin{equation}
    \overline{E}_{k}^{NPP} = \frac{h \nu - 2 m_e c^2}{2} 
\end{equation}
\begin{equation}
    \overline{E}_{k}^{TP}= \frac{h \nu - 4 m_e c^2}{3} 
\end{equation}

\section{Producción de pares en el campo de electrón}

\section{Dispersión Rayleigh}

\section{Coeficiente de atenuación total: comparación de los processos}

El coeficiente de atenuación lineal $\mu$ total se define como la suma de los diferentes coeicientes de atenuación: 

\begin{equation}
    \mu = \tau_K + \sigma_{KN} + \sigma_R + \kappa 
\end{equation}
siendo $\tau_K$ el efecto fotoeléctrico, $\sigma_{KN}$ el efecto Compton (Klein-Nishina), $\sigma_R$ Rayleigh y $\kappa$ producción de pares. 

%%%%%%%%%%%%%%%%%%%%%%%
%%%%%%%%%%%5%%%%%%%%%
%%%%%%%%%%%%%%%%%%%%%%%
%%%%%%%%%%%5%%%%%%%%%
%%%%%%%%%%%%%%%%%%%%%%%
%%%%%%%%%%%5%%%%%%%%%
%%%%%%%%%%%%%%%%%%%%%%%
%%%%%%%%%%%5%%%%%%%%%
%%%%%%%%%%%%%%%%%%%%%%%
%%%%%%%%%%%5%%%%%%%%%
%%%%%%%%%%%%%%%%%%%%%%%
%%%%%%%%%%%5%%%%%%%%%
%%%%%%%%%%%%%%%%%%%%%%%
%%%%%%%%%%%5%%%%%%%%%
%%%%%%%%%%%%%%%%%%%%%%%
%%%%%%%%%%%5%%%%%%%%%
%%%%%%%%%%%%%%%%%%%%%%%
%%%%%%%%%%%5%%%%%%%%%

\newpage
\section*{\textit{Scripts} en python para cálculos.}
\addcontentsline{toc}{section}{\textit{Scripts}}

\subsubsection{Paquetes fundamentales}
\begin{lstlisting}[language=python]
import numpy as np  
import matplotlib.pyplot as plt 
import scipy.constants as cte 
from scipy.constants import physical_constants
\end{lstlisting}

\subsubsection{Coeficiente de atenuación y sección eficaz}
\begin{lstlisting}[language=python]
r_e = physical_constants["classical electron radius"][0]
alpha=cte.alpha
sigmaTh=(8/3)*np.pi*(r_e**2)*1e28

# Fundamentales: 

def sigma(mu,rho,A): 
    """ Devuelve la sección eficaz [m²] para un coeficiente de atenuación, densidad y número másico"""
    return mu*A/cte.N_A/rho

def mu(sigma,rho,A): 
    """ Devuelve el coeficiente de atenuación [m-1] para una sección eficaz [m²], densidad y número másico"""
    return sigma*(A/cte.N_A/rho)**(-1)
\end{lstlisting}
\subsubsection{Sección eficaz}
\begin{lstlisting}[language=python]
# Secciones eficaces: 

def sigma_compton(eps):
    """ Devuelve la sección eficaz de Compton [m²] (Klein-Nishina)"""
    term1 = (1 + eps) / eps**2 * (2*(1+eps)/(1+2*eps) - np.log(1+2*eps)/eps)
    term2 = np.log(1+2*eps) / (2*eps)
    term3 = (1+3*eps) / (1+2*eps)**2
    return 2 * np.pi * r_e**2 * (term1 + term2 - term3)
    

def sigma_fotoelec(Z, eps,sigma_Th=sigmaTh, n=4):
    """  Devuelve el coeficiente de atenuación [m] del efecto fotoeléctrico. """
    return alpha**4 * sigma_Th * Z**n * np.sqrt(32/(eps**7))

def sigma_pp(epsilon, Z):
    """     Calcula  seccion eficaz [m²] par produccitoin (kappa).     """
    # Screening threshold
    screening_limit = 1 / (alpha * Z**(1/3))
    # Factor prefactorial común
    prefactor = alpha * re**2 * Z**2
    
    if epsilon > 4:  # Región fuera de los límites anteriores
        P = (28/9) * np.log(2*epsilon) - (218/27) - 1.027
    
    elif epsilon > screening_limit:  # Complete screening
        P = (28/9) * np.log(183 / Z**(1/3)) - 2/27
    
    else:  # No screening
        P = (28/9) * np.log(2*epsilon) - (218/27)
    
    return prefactor * P
\end{lstlisting}
\subsubsection{Coeficiente de atenuación}
\begin{lstlisting}[language=python]
# Coeficientes de Atenuación: 

def mu_fotoelec(Z, eps, A,rho, sigma_Th=sigmaTh, n=4):
    """  Devuelve el coeficiente de atenuación [m] del efecto fotoeléctrico. """
    return mu(sigma_fotoelec(Z,eps,sigma_Th,n),rho,A)

def mu_compton(eps,rho,A):
    """  Devuelve el coeficiente de atenuación compton """
    return mu(sigma_compton(eps),rho,A)

def mu_pp(eps,Z,rho,A):
    """  Devuelve el coeficiente de atenuación por producción de pares """

    return mu(sigma_pp(eps,Z),rho,A)

def mu_rayleith(sigma_raylei,rho,A):
    """  Devuelve el coeficiente de atenuación por rayleigh """
    return mu(sigma_raylei,rho,A)

\end{lstlisting}