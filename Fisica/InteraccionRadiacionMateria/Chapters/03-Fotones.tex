\chapter{Fotones}

\section{Procesos mayores}

\subsection{Efecto Compton}

El efecto Compton se define como la dispersión inelástica de un fotón con un electrón.

\subsection{Diferencia de longitud de onda}

Una de las característiccas que definen el efecto Compton es que el fotón, tras la colisión, sale con una longitud de onda mayor que la incidente, con una diferencia dependiente únicamente del ángulo de salida del electrón $\theta$, tal que 

\begin{equation}
    \Delta \lambda = \lambda' - \lambda = \lambda_C (1-\cos \theta) \qquad \lambda_C = \frac{h}{m_ec}  \qquad \lambda_C = 2.43 \ \unit{fm}
\end{equation}
siendo $\lambda_C$ la \textbf{longitud de onda Compton}. Esta fórmula se deduce directamente de la conservación del cuadrimomento $p^\mu$ en la colsión, tal que si ($c=1$, para el cálculo luego para recuperar la expresión lo haremos mediante las unidades):

\begin{equation}
    p_{i,\gamma}^\mu = \pqty{h \nu, \pn_i } \qquad 
    p_{i,e}^\mu = \pqty{m_e, 0} \qquad 
    p_{f,\gamma}^\mu = \pqty{h \nu, \pn_f } \qquad 
    p_{f,e}^\mu = \pqty{E_e, \qn_f }
\end{equation}
tenemos que se verifica la triple igualdad

\begin{eqnarray}
    h \nu + m_e = E_k + m_e + h \nu' \\
    \frac{h}{\lambda}= \frac{h}{\lambda'}\cos \theta + q_f \cos \phi  \\
    0 = \frac{h}{\lambda'} \sin \theta + q_f \sin \phi
\end{eqnarray}
Mandando todos los términos a la izquierda de las ecuaciones del momento, y elevando al cuadrado obtenemos: 
\begin{eqnarray}
    \pqty{\frac{h}{\lambda}- \frac{h}{\lambda'}\cos \theta}^2= \pqty{q_f \cos \phi}^2  \\
    \pqty{\frac{h}{\lambda'} \sin \theta}^2 = \pqty{q_f \sin \phi}^2
\end{eqnarray}
tal que al sumarse: 
\begin{equation}
    \pqty{\frac{h}{\lambda}- \frac{h}{\lambda'}\cos \theta}^2+ \pqty{\frac{h}{\lambda'} \sin \theta}^2 = (q_f)^2
\end{equation}
tal que se reduce a: 

\begin{equation}
    \pqty{\frac{h}{\lambda}}^2 + \pqty{\frac{h}{\lambda'}}^2 
    - 2 \frac{h^2}{\lambda \lambda'} \cos \theta = (q_f)^2
\end{equation}
y luego: 

\begin{equation}
    \lambda'^2 + \lambda^2 - 2 \lambda\lambda' \cos \theta = \lambda^2 \lambda'^2 \pqty{\frac{q_f}{h}}^2
\end{equation}
y la energía: 

\begin{equation}
    h \nu + m_e = E_k + m_e + h \nu' \Rightarrow  h(\lambda'-\lambda) =\lambda \lambda' E_k \Rightarrow \frac{h(\lambda'-\lambda)}{\lambda \lambda'} + m_e = E_{tot}
\end{equation}
\begin{equation}
    q_f^2 = \pqty{\frac{h(\lambda'-\lambda)}{\lambda \lambda'} + m_e }^2 - m_e^2 
\end{equation}
Ahora, al sustituir: 

\begin{equation}
    \lambda'^2 + \lambda^2 - 2 \lambda\lambda' \cos \theta = \frac{\lambda^2 \lambda'^2}{h^2} \bqty{\pqty{\frac{h(\lambda'-\lambda)}{\lambda \lambda'} + m_e }^2 - m_e^2 }
\end{equation}
Ahora: 

\begin{equation}
    \lambda'^2 + \lambda^2 - 2 \lambda \lambda' \cos \theta = (\lambda'- \lambda)^2 + \frac{2 m_e}{h}  \lambda \lambda' (\lambda'-\lambda)
\end{equation}
De lo que se deduce directamente: 

\begin{equation}
    2\lambda \lambda' (1-\cos \theta) = \frac{2 m_e}{h}  \lambda \lambda' (\lambda'-\lambda)
\end{equation}
Finalmente obteniendo 

\begin{equation}
     (\lambda'-\lambda)= \frac{h}{m_e}(1-\cos \theta) 
\end{equation}
y finalmente deshaciendo el $c=1$ tenemos 

\begin{equation}
    \Delta \lambda = \lambda'-\lambda = \frac{h}{cm_e}(1-\cos \theta) 
\end{equation}

\subsection{Energía media del electrón}

La fracción de energía transferida al electrón $\bar{f}_C$ es: 

\begin{equation} \scriptsize
\bar{f}_C(\varepsilon) =
\frac{
\left\{
\dfrac{2(1+\varepsilon)^2}{\varepsilon^2(1+2\varepsilon)}
- \dfrac{1+3\varepsilon}{(1+2\varepsilon)^2}
- \dfrac{(1+\varepsilon)(2\varepsilon^2 - 2\varepsilon - 1)}{\varepsilon^2(1+2\varepsilon)^2}
- \dfrac{4\varepsilon^2}{3(1+2\varepsilon)^3}
- \left[\dfrac{1+\varepsilon}{\varepsilon^3} - \dfrac{1}{2\varepsilon} + \dfrac{1}{2\varepsilon^3}\right] \log(1+2\varepsilon)
\right\}
}{
\left\{
\dfrac{1+\varepsilon}{\varepsilon^2} \left[\dfrac{2(1+\varepsilon)}{1+2\varepsilon} - \dfrac{\log(1+2\varepsilon)}{\varepsilon}\right]
+ \dfrac{\log(1+2\varepsilon)}{2\varepsilon}
- \dfrac{1+3\varepsilon}{(1+2\varepsilon)^2}
\right\}
}
\end{equation}
Siendo, en el límite ultrarrelativista $\varepsilon\gg 1$_


\begin{equation} 
\bar{f}_C(\varepsilon) = \frac{\log (2\varepsilon)-0.82}{\log(2\varepsilon)+0.5}
\end{equation}

%%%%%%%%%%%%%%%%%%%%%%%%%%%%%%%%%%%%%%%%%%%%%%%%%%%%%%%%%%%%%%%%%%%%%%%%%
%%%%%%%%%%%%%%%%%%%%%%%%%%%%%%%%%%%%%%%%%%%%%%%%%%%%%%%%%%%%%%%%%%%%%%%%%
%%%%%%%%%%%%%%%%%%%%%%%%%%%%%%%%%%%%%%%%%%%%%%%%%%%%%%%%%%%%%%%%%%%%%%%%%
%%%%%%%%%%%%%%%%%%%%%%%%%%%%%%%%%%%%%%%%%%%%%%%%%%%%%%%%%%%%%%%%%%%%%%%%%
%%%%%%%%%%%%%%%%%%%%%%%%%%%%%%%%%%%%%%%%%%%%%%%%%%%%%%%%%%%%%%%%%%%%%%%%%
%%%%%%%%%%%%%%%%%%%%%%%%%%%%%%%%%%%%%%%%%%%%%%%%%%%%%%%%%%%%%%%%%%%%%%%%%
%%%%%%%%%%%%%%%%%%%%%%%%%%%%%%%%%%%%%%%%%%%%%%%%%%%%%%%%%%%%%%%%%%%%%%%%%
%%%%%%%%%%%%%%%%%%%%%%%%%%%%%%%%%%%%%%%%%%%%%%%%%%%%%%%%%%%%%%%%%%%%%%%%%
%%%%%%%%%%%%%%%%%%%%%%%%%%%%%%%%%%%%%%%%%%%%%%%%%%%%%%%%%%%%%%%%%%%%%%%%%
%%%%%%%%%%%%%%%%%%%%%%%%%%%%%%%%%%%%%%%%%%%%%%%%%%%%%%%%%%%%%%%%%%%%%%%%%
%%%%%%%%%%%%%%%%%%%%%%%%%%%%%%%%%%%%%%%%%%%%%%%%%%%%%%%%%%%%%%%%%%%%%%%%%


\section*{Ejercicios}
\addcontentsline{toc}{section}{Ejercicios}


\begin{Ejercicio}{Efecto Fotoeléctrico.}
    Calcule la energía de un fotoelectrón arrancado de la capa K del estaño por un fotón de 40 keV de energía.
\end{Ejercicio}

Primero lo que tenemos que saber es cual es la energía de ligadura de un electrón de la capa $K$ del estaño, denotada por $E_B$. Como nos están preguntando concretamente por el ``fotoelecctrón arrancado'' no tenemos por que tener en cuenta los electrones auger (si no la pregunta debería ser ``calcule la energía cinética transferida a electrones en el proceso fotoeléctrico''). Así pues, si $E_B=29$ keV, tenemos que 

\begin{equation}
    E_k = E_{\gamma} - E_B = 11 \unit{keV}
\end{equation}


%%%%%%%%%%%%%%%%%%%%%%%%%%%%%%%%%%%%%%%%%%%%%%%%%%%%%%%%%%%%%%%%%%%%%%%%%
%%%%%%%%%%%%%%%%%%%%%%%%%%%%%%%%%%%%%%%%%%%%%%%%%%%%%%%%%%%%%%%%%%%%%%%%%
%%%%%%%%%%%%%%%%%%%%%%%%%%%%%%%%%%%%%%%%%%%%%%%%%%%%%%%%%%%%%%%%%%%%%%%%%
%%%%%%%%%%%%%%%%%%%%%%%%%%%%%%%%%%%%%%%%%%%%%%%%%%%%%%%%%%%%%%%%%%%%%%%%%
%%%%%%%%%%%%%%%%%%%%%%%%%%%%%%%%%%%%%%%%%%%%%%%%%%%%%%%%%%%%%%%%%%%%%%%%%

\begin{Ejercicio}{Efecto Compton.}
    Calcule la longitud de onda del fotón dispersado, y la velocidad del electrón de retroceso, producidos cuando un haz de rayos X de 0,03 nm de longitud de onda se dispersa Compton un ángulo de 45°.
\end{Ejercicio}

La única dependencia de la longitud de onda resultate es la longitud de onda inicial y el ángulo de salida, datos que nos dan directamente. Aplicando entonces la ecuación: 

\begin{equation}
    \Delta \lambda = \lambda'-\lambda = \frac{h}{cm_e}(1-\cos \theta) 
\end{equation}
Aplicando la ecuación, la longtiud de onda de salida es

\begin{equation}
    \lambda' = \lambda + \frac{h}{cm_e} (1-\cos \theta)
\end{equation}
que obtenemos: 

\begin{equation}
    \lambda' = 0.03024 \ \unit{fm}
\end{equation}

%%%%%%%%%%%%%%%%%%%%%%%%%%%%%%%%%%%%%%%%%%%%%%%%%%%%%%%%%%%%%%%%%%%%%%%%%
%%%%%%%%%%%%%%%%%%%%%%%%%%%%%%%%%%%%%%%%%%%%%%%%%%%%%%%%%%%%%%%%%%%%%%%%%
%%%%%%%%%%%%%%%%%%%%%%%%%%%%%%%%%%%%%%%%%%%%%%%%%%%%%%%%%%%%%%%%%%%%%%%%%
%%%%%%%%%%%%%%%%%%%%%%%%%%%%%%%%%%%%%%%%%%%%%%%%%%%%%%%%%%%%%%%%%%%%%%%%%
%%%%%%%%%%%%%%%%%%%%%%%%%%%%%%%%%%%%%%%%%%%%%%%%%%%%%%%%%%%%%%%%%%%%%%%%%

\begin{Ejercicio}{Efecto Fotoeléctrico}
    Calcule el coeficiente másico de transferencia de energía fotoeléctrico para la capa K del estaño y un fotón de 40 keV de energía.
\end{Ejercicio}

El coeficiente de transferencia energético es 

\begin{equation}
    \frac{(\tau_K)_{tr}}{\rho} = \frac{\tau}{\rho} \pqty{1-\frac{P_K\omega_K\eta_K E_B(K)}{h \nu}}
\end{equation}
siendo

\begin{equation} \small
    E_B (K) = 29.2 \text{keV} \quad  P_K = 0.839 \quad \omega_K = 0.859 \quad \eta_K =0.892 \quad h \nu = 40 \ \unit{keV} \quad \rho = 7365 \unit{kg/m^3}
\end{equation}
y finalmente, la sección eficaz: 

\begin{equation}
    a\tau_K = \alpha^4 (_e \sigma_{Th}) Z^n  \sqrt{\frac{32}{\epsilon^7}}
\end{equation}
con $n=4$ y y $\tau_K \approx 498$ b. 

\begin{align} 
    \sigma_{ph} &= \alpha^4 \, (e\sigma_{Th}) \, Z^n \, \sqrt{\frac{32}{\varepsilon^7}}, \\[1ex]
    \varepsilon &= \frac{h\nu}{m_e c^2} = \frac{40}{511} = 0.0783, \\[1ex]
    Z^n &= 50^4 = 6.25 \times 10^6, \\[1ex]
    \sqrt{\frac{32}{\varepsilon^7}} &= \sqrt{\frac{32}{(0.0783)^7}} \approx 4.215 \times 10^4, \\[1ex]
    \alpha^4 (e\sigma_{Th}) &\approx (1/137)^4 \times 0.665\ \text{b} 
    \approx 1.89 \times 10^{-9}\ \text{b}, \\[1ex]
    \sigma_{ph} &\approx (1.89 \times 10^{-9})(6.25 \times 10^6)(4.215 \times 10^4)\ \text{b}, \\[1ex]
    \sigma_{ph} &\approx 498\ \text{b}.
\end{align}
Por tanto,
\[
    {\tau_K \approx 500\ \text{barns}}
\]
para el estaño a \(h\nu = 40\ \text{keV}\), con \(n=4\). Finalmente: 

\begin{align} \small
    \tau_K &= 500\ \text{b} = 5.0\times 10^{-22}\ \text{cm}^2/\text{átomo}, \\[1ex]
    \frac{\tau}{\rho} &= \frac{N_A}{M}\,\tau 
    = \frac{6.022\times 10^{23}}{118.71}\,(5.0\times 10^{-22}) \notag \\
    &= 2.536\ \text{cm}^2/\text{g}, \\[1ex]
    1 - \frac{P_K \omega_K \eta_K E_B(K)}{h\nu} 
    &= 1 - (0.839)(0.859)(0.892)\frac{29.2}{40} \notag \\
    &= 0.5307, \\[1ex]
\end{align}
Finalmente, el coeficiente de trasmisión 
\begin{align}
    \frac{(\tau_K)_{tr}}{\rho} 
    &= \frac{\tau}{\rho}\,\Biggl(1 - \frac{P_K \omega_K \eta_K E_B(K)}{h\nu}\Biggr) \notag \\
    &= 2.536 \times 0.5307 \\[1ex]
    &= 1.35\ \text{cm}^2/\text{g} \;\;=\;\; 0.135\ \text{m}^2/\text{kg}.
\end{align}
Claro que esto es siguiendo el modelo, en la realidad $\tau/\rho=18.87$ cm$^2$/g, que como podemos ver dista mucho del modelo. 



%%%%%%%%%%%%%%%%%%%%%%%%%%%%%%%%%%%%%%%%%%%%%%%%%%%%%%%%%%%%%%%%%%%%%%%%%
%%%%%%%%%%%%%%%%%%%%%%%%%%%%%%%%%%%%%%%%%%%%%%%%%%%%%%%%%%%%%%%%%%%%%%%%%
%%%%%%%%%%%%%%%%%%%%%%%%%%%%%%%%%%%%%%%%%%%%%%%%%%%%%%%%%%%%%%%%%%%%%%%%%
%%%%%%%%%%%%%%%%%%%%%%%%%%%%%%%%%%%%%%%%%%%%%%%%%%%%%%%%%%%%%%%%%%%%%%%%%
%%%%%%%%%%%%%%%%%%%%%%%%%%%%%%%%%%%%%%%%%%%%%%%%%%%%%%%%%%%%%%%%%%%%%%%%%

\begin{Ejercicio}{Efecto Compton}
    Suponga que un fotón de 3 MeV de energía interacciona con un medio material vía efecto Compton. Calcule la energía del electrón emitido y del fotón dispersado si el ángulo de dispersión es de 90°. Repita el cálculo para un ángulo de dispersión de 180°. ¿A qué ángulo saldrá el fotón dispersado si se llevara el 56\% de la energía del fotón incidente?
\end{Ejercicio}

El efecto Compton dispersa el fotón con una energía con $\theta=\pi/2$:

\begin{equation*}
    E_{\gamma}' = \frac{E_{\gamma}}{1+\frac{E_{\gamma}}{m_ec^2} (1-\cos(\theta))} = 0.44 \ \unit{MeV}
\end{equation*}
obteniendo como enerǵia del electrón: 
\begin{equation*}
    E_K = E_{\gamma}'-E_{\gamma} =  2.56 \unit{MeV}
\end{equation*}
Y por otro lado, cuando $\theta=\pi$:

$$ E_{\gamma}'  = 0.24 \ \unit{MeV} \qquad 
E_K = 2.76 \ \unit{MeV}$$

Si $E_{\gamma}'=0.56 E_{\gamma}$, tenemos que: 

\begin{equation*}
    \frac{E_{\gamma}'}{E_{\gamma}} = \alpha = \frac{1}{1+\varepsilon (1-\cos \theta)} \Rightarrow \alpha (1+\varepsilon (1-\cos \theta)) = 1 
\end{equation*}
\begin{equation*}
    -\varepsilon (1-\cos \theta) = \alpha-1 \Rightarrow \cos \theta =  1 -  \frac{(1-\alpha + \epsilon)}{\epsilon} = \frac{1-\alpha}{\epsilon}
\end{equation*}
que siempre es menor que uno. Así pues, obtenemos el ángulo: 

\begin{equation*}
    \theta \approx 29.8^\circ
\end{equation*}



%%%%%%%%%%%%%%%%%%%%%%%%%%%%%%%%%%%%%%%%%%%%%%%%%%%%%%%%%%%%%%%%%%%%%%%%%
%%%%%%%%%%%%%%%%%%%%%%%%%%%%%%%%%%%%%%%%%%%%%%%%%%%%%%%%%%%%%%%%%%%%%%%%%
%%%%%%%%%%%%%%%%%%%%%%%%%%%%%%%%%%%%%%%%%%%%%%%%%%%%%%%%%%%%%%%%%%%%%%%%%
%%%%%%%%%%%%%%%%%%%%%%%%%%%%%%%%%%%%%%%%%%%%%%%%%%%%%%%%%%%%%%%%%%%%%%%%%
%%%%%%%%%%%%%%%%%%%%%%%%%%%%%%%%%%%%%%%%%%%%%%%%%%%%%%%%%%%%%%%%%%%%%%%%%

\begin{Ejercicio}{Efecto Comptón y sección eficaz de Klein Nishina}
    Calcule la sección eficaz de Klein-Nishina para fotones de 1 MeV de energía y obtenga el coeficiente de atenuación másico Compton para el cobre.
\end{Ejercicio}

El resultado nos lleva a: 

\begin{equation}
    \sigma^{KN} \simeq \pi r_e^2 \frac{2\log(2\epsilon)+1}{2\epsilon} = 0.2373 \ \unit{b}
\end{equation}
usando la fórmula $\epsilon\gg 1$, aunque $\epsilon = 1.956$. Por otro lado, el coeficiente de atenuación másico: 

\begin{equation}
    \frac{\mu}{\rho} = \frac{Z}{N_A}N_A (\sigma^{KN}) = 0.0542 \ \unit{cm^2/g}
\end{equation}


%%%%%%%%%%%%%%%%%%%%%%%%%%%%%%%%%%%%%%%%%%%%%%%%%%%%%%%%%%%%%%%%%%%%%%%%%
%%%%%%%%%%%%%%%%%%%%%%%%%%%%%%%%%%%%%%%%%%%%%%%%%%%%%%%%%%%%%%%%%%%%%%%%%
%%%%%%%%%%%%%%%%%%%%%%%%%%%%%%%%%%%%%%%%%%%%%%%%%%%%%%%%%%%%%%%%%%%%%%%%%
%%%%%%%%%%%%%%%%%%%%%%%%%%%%%%%%%%%%%%%%%%%%%%%%%%%%%%%%%%%%%%%%%%%%%%%%%
%%%%%%%%%%%%%%%%%%%%%%%%%%%%%%%%%%%%%%%%%%%%%%%%%%%%%%%%%%%%%%%%%%%%%%%%%

\begin{Ejercicio}{} 
    Calcule la energía máxima y la promedio que adquieren los electrones de retroceso Compton generados por radiación electromagnética de 20 keV y 20 MeV de energía.
\end{Ejercicio}

La fracción energía transferida al electrón media viene dada por: 
\begin{equation} \scriptsize
\bar{f}_C(\varepsilon) =
\frac{
\left\{
\dfrac{2(1+\varepsilon)^2}{\varepsilon^2(1+2\varepsilon)}
- \dfrac{1+3\varepsilon}{(1+2\varepsilon)^2}
- \dfrac{(1+\varepsilon)(2\varepsilon^2 - 2\varepsilon - 1)}{\varepsilon^2(1+2\varepsilon)^2}
- \dfrac{4\varepsilon^2}{3(1+2\varepsilon)^3}
- \left[\dfrac{1+\varepsilon}{\varepsilon^3} - \dfrac{1}{2\varepsilon} + \dfrac{1}{2\varepsilon^3}\right] \log(1+2\varepsilon)
\right\}
}{
\left\{
\dfrac{1+\varepsilon}{\varepsilon^2} \left[\dfrac{2(1+\varepsilon)}{1+2\varepsilon} - \dfrac{\log(1+2\varepsilon)}{\varepsilon}\right]
+ \dfrac{\log(1+2\varepsilon)}{2\varepsilon}
- \dfrac{1+3\varepsilon}{(1+2\varepsilon)^2}
\right\}
}
\end{equation}
y en el caso ultrarrelativista es: 

\begin{equation} 
\bar{f}_C(\varepsilon) = \frac{\log (2\varepsilon)-0.82}{\log(2\varepsilon)+0.5}
\end{equation}
La fracción de energía media transferida entonces es: 

\begin{equation*}
    \bar{f}_C(20\text{keV}) = 0.036 \qquad 
    \bar{f}_C(20\text{MeV}) = 0.73
\end{equation*}
Ahora quedaría multiplicarla por la energía incidente. Por otro lado, la energía máxima transferida que pueden adquirir. La máxima transferencia de energía viene dada por el máximo de la función

\begin{equation*}
    f_C(E_\gamma,\theta) = \frac{\varepsilon(1-\cos(\theta))}{1+\varepsilon(1-\cos \theta)}
\end{equation*}
claramente cuando $\theta=\pi$ tenemos el máximo, siendo la máxima fracción de trasferencia de energía: 

\begin{equation*}
   f_{C,\max}(20\text{keV}) = 7.3 \%   \quad 
    f_{C,\max}(20\text{MeV}) = 99\%
\end{equation*}
Las energías, para $20$ keV:
\begin{equation*}
    \bar{E}_K = 0.722 \ \text{keV} \qquad  E_{K,\max}= 1.45 \ \text{keV}
\end{equation*}
Para 20 MeV: 
\begin{equation*}
    \bar{E}_K = 14.53 \  \text{MeV} \qquad  E_{K,\max}= 19.74 \ \text{MeV}
\end{equation*}


