\chapter{Dispersiones}

\section{Teoría de dispersiónes: mecánica clásica y  mecánica cuántica}

Para describir la dispersión de una partícula por un potencial (Coulomb) en la mecánica cuántica en principio deberíamos estudiar la evolución temporal de un paquete de ondas, que representará la partícula incidente con una energía (y anchura) y una localización espacial (y anchura). Sin embargo, se suelen usar ondas planas, ya que en general las dimensiones del paquete de ondas suele ser mucho más grande que el tamaño del rango efectivo del potencial. La expresión de Rutherford no es una excepción.  

Para poder entender como se obtiene en la mecánica cuántica la expresión de la sección eficaz lo primero que debemos hacer es entender que la sección eficaz está realcionada directamente con el cuadrado de la amplitud de probabilidad:

\begin{equation}
    \dv{\sigma}{\Omega} = |f(\theta,\varphi)|^2 
\end{equation}
Uno podría plantearse ¿por qué es así? En la mecánica cuántica la sección eficaz diferencial representa la probabilidad de que una partícula salga dispersada en un ángulo sólido diferencial $\D \Omega$, aunque también nos habla de la tasa de partículas que atarviesan una superficie diferencial perpendicular $\D \Omega$. En la mecánica cuántica toda probabilidad es representada por una amplitud de probaiblidad por su complejo, por lo que la relación es directa. 

\section{Análisis en ondas parciales}

\subsection{Introducción}

\subsection{Fases}

\subsection{Teorema óptico}

\section{Aproximación de Born}

\subsection{Primera aproximación de Born}

\subsection{Teoría de Born}

\section*{Ejercicios}
\addcontentsline{toc}{section}{Ejercicios}


\begin{Ejercicio}{Sección eficaz de Rutherford con modelo atómico de Fermi} \label{Ej:01.01}
    Obten la expresión de la sección de eficaz de Rutherford, usando la aproximación de Born con ondas planas incidentes y salientes con el potencial de Coulomb:

    \begin{equation}
        V(r) = \frac{zZe^2}{4\pi \varepsilon} \frac{1}{r}
    \end{equation}
\end{Ejercicio}


