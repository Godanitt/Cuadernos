\chapter{Partículas Cargadas}

\section{Dispersión de Rutherford}

\subsection{Introducción}

Rutherford describió la dispersión de una partícula $\alpha$ por núcleos atómicos (con $m_\alpha \ll M, z=2$) sobre una trayectoria hiperbólica a través del potencial de Coulomb. La energía total viene dada como la suma de la energía cinética y potencial, y permanece \textit{constante} sobre toda la trayectoria:

\begin{equation}
    E(r) = E_K(r) + E_p (r) \tquad E_p (r)
\end{equation}
siendo $\theta$ el \textbf{ángulo de dispersión}. La distancia de máxima aproximación se define como la distancia a la cual se para una partícula con parámetro de impacto $b=0$, i.e. colisión frontal. Así, denotando $E_K$ como la \textit{energía cinética inicial}, tenemos que

\begin{equation}
   E_K = \frac{zZe^2}{4\pi\varepsilon_0} \frac{1}{D_{\alpha-N}}
\end{equation}
aunque lógicamente esta expresión no será del todo cierta, ya que el potencial no es infinito en $r=0$, debido a que el núcleo es finito. Aún así es una buena aproximación.

%Aquí hay bastante que copiar por parte de bernardo

\subsection{Relación entre el parámetro de impacto y el ángulo de dispersión}

El momento transferido en la interacción, siempre y cuando $|\pn_i |= |\pn_f|=p$ vale 

\begin{equation}
    \Delta p = 2 p \sin (\theta/2)
\end{equation}
Es fácil de deducir matemáticamente, pero incluso físicamente tiene sentido: cuando $\theta=\pi$ la trasferencia de momento es máxima (la partícula se da la vuelta completamente) lo que implicla un intercambio de $2$ veces el momento inicial. Definimos como \textbf{parámetro de impacto} 

\begin{equation}
    b = r \sin \theta
\end{equation}
Además, también se conserva el momento angular $L = |\rn \times \pn|=L p \sin \theta$, lo que lleva a la relación 

\begin{equation}
    v_i b = \omega r^2
\end{equation}
siendo $\omega=\dv{\phi}{t}$ en el vértice, definiendo $\phi$ como el ángulo entre el vértice y la partícula desde el núcleo. En el vértice, que es el punto en el cual la posición respecto el átomo y el momento lineal de la partícula son perpendiculares. Por definición $L = m_\alpha r^2 \omega$, y en el infinito se cumple que

\begin{equation}
    L_\infty = |\rn \times \pn | = m_\alpha b v_i
\end{equation}
El momento transferido $\Delta p$ se calcula como la acción total en el tiempo de la fuerza de Coulomb proyectada sobre el eje de la hipérbola, tal que

\begin{equation}
    \Delta p = \int_{-\infty}^{\infty} F_{\Delta p} \d t = (\ldots) = 
\end{equation}
de lo que se puede deducir que el parámetro de impacto $b$ y el ángulo de dispersión se relacionan tal que:

\begin{equation}
    b = \frac{1}{2} D_{\alpha - N} \cot (\theta/2) = \frac{1}{2} D_{\alpha-N} \sqrt{\frac{1+\cos \theta}{1-\cos \theta}} \qquad D_{\alpha-N} = \frac{zZe^2}{4\pi \varepsilon_0} \frac{1}{E_K}
\end{equation}
siendo $\D_{\alpha-N}$ la distancia de máxima aproximación y $E_K$ la energía cinética. 

\subsection{La sección eficaz diferencial de Rutherford}

La dispersión de las partícuals alfa que halló Rutherford la describió en función de su sección diferencial, tal que:

\begin{equation}
    \dv{\sigma_{\Ruth}}{\Omega} =  \pqty{\frac{D_{\alpha-N}}{4}}^2 \frac{1}{\sin^4 \frac{\theta}{2}} \qquad D_{\alpha-N} = \frac{zZe^2}{4\pi\varepsilon_0} \frac{1}{E_K}
\end{equation}
Como podemos comprobar, \textit{diverge} para $\theta=0$, lo cual es debido al carácter ideal del núcleo puntual. La razón por la que en realidad no diverge, o la solución a esta divergencia, es que la carga nuclear está apantallada por los \textit{electrones orbitales atómicos}, lo que conduce a un ángulo mínimo $\theta_{\min}$. La dispersión hacia atrás ($\theta \to \pi, b \to 0$) tampoco se logra nunca, debido al radio finito del núcleo. 

Un modelo que tiene en cuenta el apantallamiento es el \textit{modelo atómico estadístico de Fermi}, el cual introduce una exponencial tabulada por un \textit{radio efectivo} $\alpha_{TF}$ de tal modo que el potencial de Coulomb no caiga ``lentamente''. Así pues: 

\begin{equation}
    V_{TF} (r) = \frac{zZe^2}{4\pi\varepsilon_0} \frac{1}{r} e^{-\frac{r}{a_{TF}}}
\end{equation}
el \textit{radio efectivo} $a_{TF}$ de la nube electrónica (de Thomas-Fermi) decrece por debajo del radio de Bohr $\alpha_0$ al aumentar $Z$, como $a_{TF}=\frac{\zeta a_0}{\sqrt[3]{Z}}$, debido a la disminución del tamaño de los orbitales más internos, por el teorema de Gauss. 

La aproximación de Born requiere ahora, con $\Delta k = K=(2p/\hbar) \sin(\theta/2) (p=p_i)$ el momento de onda transferido ($p = \hbar k$), el cálculo de la transformada seno de $V_{TF}(r)$, que es finita. Rutherford reciba entonces un \textit{factor corrector} tal qeu:

\begin{equation}
    \dv{\sigma_{\Ruth}}{\Omega} = \vqty{\frac{2m_\alpha}{\hbar^2}\int_{0}^{\infty} r^2 V_{TF}(r) \frac{\sin Kr}{Kr} \D r }^2 = \pqty{\ldots} = \pqty{\frac{D_{\alpha-N}}{4}}^2 \frac{1}{\sin^4 \theta/2} \pqty{\frac{1}{1+\frac{1}{K^2 a_{TF}^2}}}
\end{equation}
Cuando $\theta<<1$, tenemos que: 

\begin{equation}
    \dv{\sigma_{\Ruth}}{\Omega} = \frac{D_{\alpha-N}^2}{\theta^4} \frac{1}{\bqty{1+\pqty{\frac{\theta_{\min}^2}{\theta^2}}^2}} = \frac{D_{\alpha-N}^2}{\theta^2+\theta_{\min}^2} 
\end{equation}
donde se introduce el concepto el concepto de ángulo mínimo, que como podemos ver viene expresado: 

\begin{equation}
    \theta_{\min} = \frac{\hbar}{p a_{TF}} = \frac{\hbar \sqrt[3]{Z}}{pa_{TF}} = \frac{\hbar c \sqrt[3]{Z}}{a_0 \sqrt[root]{E_K(E_K+2Mc^2)}} 
\end{equation}
La mejora de la aparición de este ángulo mńimo es que ahora \textit{la sección eficaz total es finita}, lo cual se corresponde a un resultado mucho más físico. Este ángulo mínimo tiene un \textit{origen cuántico} debido al \textit{principio de incertidumbre}. Se produce por la localización de la partícula sobre el radio $a_{TF}$, que le imprime un momento \textit{transversal} inevitable $\Delta p$, relacionado con la longitud de onda: 
\begin{equation}
    \theta_{\min} = \frac{\Delta p}{p} \approx \frac{\hbar}{pa_{TF}} = \frac{\lambdabar}{a_{TF}}
\end{equation}


\subsection{Corrección por radio finito del núcleo}

El potencial que vve una carga elemental $z$ cerca del núcleo $V(r)$ tiene un plateu en su interior hasta $r=R$ y para $r>R$ adopta la forma de Coulomb $E_p(r)$. 

\subsection{Dispersión de Mott} 

\subsection{Correciones por espín del electrón y retroceso del núcleo}

\section{Poder de frenado másico de partículas cargadas en la materia}

\section{El rango másico $R_{CSDA}$}


\section*{Ejercicios}
\addcontentsline{toc}{section}{Ejercicios}

\begin{Ejercicio}
    \subsubsection*{Ejercicio 1}
    Obten la expresión de la sección de eficaz de Rutherford, con el potencial de Coulomb, usando la aproximación de Born con ondas planas incidentes y salientes

    \begin{equation*}
        \dv{\sigma_{\Ruth}}{\Omega} =  \pqty{\frac{D_{\alpha-N}}{4}}^2 \frac{1}{\sin^4 \frac{\theta}{2}} \qquad D_{\alpha-N} = \frac{zZe^2}{4\pi\varepsilon_0} \frac{1}{E_K}
    \end{equation*}

\end{Ejercicio}

\begin{Ejercicio}
    \subsubsection*{Ejercicio 2}
    Obten la expresión de la sección de eficaz de Rutherford, ahora con el potencial del modelo atómico estadístico de Fermi, usando la aproximación de Born con ondas planas incidentes y salientes

    \begin{equation*}
        \dv{\sigma_{\Ruth}}{\Omega} =  \pqty{\frac{D_{\alpha-N}}{4}}^2 \frac{1}{\sin^4 \theta/2} \pqty{\frac{1}{1+\frac{1}{K^2 a_{TF}^2}}}
    \end{equation*}
    y luego llegar a la expresión final cuando $\theta\ll 1$: 
\end{Ejercicio}


% Igual estaría bien hacer la demostración aquí, aunque sea trivial

