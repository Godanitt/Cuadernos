\section*{Ejercicios}
\addcontentsline{toc}{section}{Ejercicios}

\begin{Ejercicio}{Transferencia máxima (choque elástico)}\label{Ej:04-01}
Calcule la energía máxima que un neutrón de $4 \ \mathrm{MeV}$ puede transferir a un núcleo de $^{10}\mathrm{B}$ en una colisión elástica.
\end{Ejercicio}

Como ya hemos visto, la energía de trasferencia máxima es (en el caso de que la trasferencia sea no relativist, lo cual es cierto para un neutrón de 4 MeV): 

\begin{equation}
    (\Delta E)_{\max} = \frac{4m_n M}{(M+m_n)^2} E_{k}
\end{equation}
así pues: 

\begin{equation}
     (\Delta E)_{\max} \approx \frac{40}{(11)^2} E_k = \frac{40}{121} E_k \approx 0.3306 E_k
\end{equation}
obteniendo así: 

\begin{equation}
     (\Delta E)_{\max} \approx \SI{1.32}{MeV}
\end{equation}



%%%%%%%%%%%%%%%%%%%%%%%%%%%%%%%%%%%%%%%%%%%%%%%%%%%%%%%%%%%%%
%%%%%%%%%%%%%%%%%%%%%%%%%%%%%%%%%%%%%%%%%%%%%%%%%%%%%%%%%%%%%
%%%%%%%%%%%%%%%%%%%%%%%%%%%%%%%%%%%%%%%%%%%%%%%%%%%%%%%%%%%%%
%%%%%%%%%%%%%%%%%%%%%%%%%%%%%%%%%%%%%%%%%%%%%%%%%%%%%%%%%%%%%
%%%%%%%%%%%%%%%%%%%%%%%%%%%%%%%%%%%%%%%%%%%%%%%%%%%%%%%%%%%%%
%%%%%%%%%%%%%%%%%%%%%%%%%%%%%%%%%%%%%%%%%%%%%%%%%%%%%%%%%%%%%
%%%%%%%%%%%%%%%%%%%%%%%%%%%%%%%%%%%%%%%%%%%%%%%%%%%%%%%%%%%%%
%%%%%%%%%%%%%%%%%%%%%%%%%%%%%%%%%%%%%%%%%%%%%%%%%%%%%%%%%%%%%

\begin{Ejercicio}{Transferencia promedio a deuterón y carbono}\label{Ej:04-02}
Estime la energía promedio que un neutrón de $2\,\mathrm{MeV}$ transfiere a un deuterón en una única colisión. 
Estime también la energía promedio transferida por neutrones de $2{,}6\,\mathrm{MeV}$ al colisionar con carbono.
¿qué energía máxima pueden transferir en ambos casos?
\end{Ejercicio}

La transferencia de energía promedio es: 
\begin{equation}
    \overline{\Delta E} = \frac{1}{2} (\Delta E)_{\max}
\end{equation}
que nos lleva a los siguientes resultados:

\begingroup
\makeatletter
\let\old@floatboxreset\@floatboxreset
\def\@floatboxreset{\old@floatboxreset\centering} % fuerza centrado dentro 
\makeatother
\begin{tabular}{lccccccccccccc}
\toprule
 & $^2$ H & $^{12}$C \\
\midrule
$(\Delta E)_{\max}/E_K$ (\%) & $\SI{8.89e+01}{}$ & $\SI{2.84e+01}{}$ \\
$\overline{\Delta E}$ [MeV] & $\SI{8.89e-01}{}$ & $\SI{3.69e-01}{}$ \\
$(\Delta E)_{\max}$ [MeV] & $\SI{1.78e+00}{}$ & $\SI{7.38e-01}{}$ \\
\bottomrule
\end{tabular}

\endgroup


%%%%%%%%%%%%%%%%%%%%%%%%%%%%%%%%%%%%%%%%%%%%%%%%%%%%%%%%%%%%%
%%%%%%%%%%%%%%%%%%%%%%%%%%%%%%%%%%%%%%%%%%%%%%%%%%%%%%%%%%%%%
%%%%%%%%%%%%%%%%%%%%%%%%%%%%%%%%%%%%%%%%%%%%%%%%%%%%%%%%%%%%%
%%%%%%%%%%%%%%%%%%%%%%%%%%%%%%%%%%%%%%%%%%%%%%%%%%%%%%%%%%%%%
%%%%%%%%%%%%%%%%%%%%%%%%%%%%%%%%%%%%%%%%%%%%%%%%%%%%%%%%%%%%%
%%%%%%%%%%%%%%%%%%%%%%%%%%%%%%%%%%%%%%%%%%%%%%%%%%%%%%%%%%%%%
%%%%%%%%%%%%%%%%%%%%%%%%%%%%%%%%%%%%%%%%%%%%%%%%%%%%%%%%%%%%%
%%%%%%%%%%%%%%%%%%%%%%%%%%%%%%%%%%%%%%%%%%%%%%%%%%%%%%%%%%%%%

\begin{Ejercicio}{Colisión n-H de $2{,}6$ MeV}\label{Ej:04-03}
Un neutrón de $2{,}6\,\mathrm{MeV}$ de energía colisiona elásticamente con hidrógeno. 
¿Cuál es la probabilidad de que en la colisión pierda una energía entre $0{,}63$ y $0{,}75\,\mathrm{MeV}$? 
Si pierde realmente $0{,}75\,\mathrm{MeV}$, ¿a qué ángulo se dispersa? 
¿Qué energía pierde en el sistema de centro de masas? 
¿Cuáles son las energías en el sistema de centro de masas de neutrón y protón? 
¿Cuánta energía se asocia con el movimiento del centro de masas en el sistema del laboratorio?
\end{Ejercicio}


Recordemos que en el caso del hidrógeno la \textit{distribución de energías} o \textit{espectro} es uniforme entre 0 y $(\Delta E)_{\max}$, siendo en $(\Delta E)_{\max} = E_k$. Así pues, si $P$ es la probabilidad de que la colisión pierda una energía entre $E_1=$0.63 y $E_2=$0.75 MeV, esta se calcula como: 

\begin{equation}
    P = \frac{E_2-E_1}{E_k} = 0.0462
\end{equation}
También nos preguntan a que ángulo se dispersa si se emite se transfieren $\Delta E=0.75$ MeV, lo cual en el caso de que $m_p\approx m_n$ viene dada por: 

\begin{equation}
    \cos \phi = \sqrt{\frac{\Delta E}{E}}  = 0.8592\qquad \phi = 57.515^\circ
\end{equation}
siendo $\phi$ el ángulo del núcleo. Si queremos calcular la del neutrón tenemos que hacer: 

\begin{equation}
    \sin \theta = \sqrt{\frac{M\Delta E}{m_n(E-\Delta E)}} \sin \phi \to \theta = 32.486^\circ
\end{equation}
que tiene una aparente indeterminación cuando $\Delta E \to E$, la cual es falsa ya que cuando se trasnfiere toda la energía cinética del neutrón significa que este se para: no existe $\theta$, i.e., para la indeterminación $\theta$ tampoco está definido.

En el sistema de masas no se transfiere energía (en el caso de que la masa de las partículas sea igual, como este), solo cambia el ángulo $\Theta$. 

Luego nos preguntan por la energía en el centro de masas del neutrón y protón, que es claramente la  mitad de la disponible en el otro sistema (no relativista)
\begin{equation}
    E_{\text{disponible}} = \frac{E_k}{2}
\end{equation}
Con el centro de masas podemos asociar la mitad de la energía en el sistem laboratorio.


%%%%%%%%%%%%%%%%%%%%%%%%%%%%%%%%%%%%%%%%%%%%%%%%%%%%%%%%%%%%%
%%%%%%%%%%%%%%%%%%%%%%%%%%%%%%%%%%%%%%%%%%%%%%%%%%%%%%%%%%%%%
%%%%%%%%%%%%%%%%%%%%%%%%%%%%%%%%%%%%%%%%%%%%%%%%%%%%%%%%%%%%%
%%%%%%%%%%%%%%%%%%%%%%%%%%%%%%%%%%%%%%%%%%%%%%%%%%%%%%%%%%%%%
%%%%%%%%%%%%%%%%%%%%%%%%%%%%%%%%%%%%%%%%%%%%%%%%%%%%%%%%%%%%%
%%%%%%%%%%%%%%%%%%%%%%%%%%%%%%%%%%%%%%%%%%%%%%%%%%%%%%%%%%%%%
%%%%%%%%%%%%%%%%%%%%%%%%%%%%%%%%%%%%%%%%%%%%%%%%%%%%%%%%%%%%%
%%%%%%%%%%%%%%%%%%%%%%%%%%%%%%%%%%%%%%%%%%%%%%%%%%%%%%%%%%%%%

\begin{Ejercicio}{Energía umbral para $^{32}_{16}\mathrm{S}(\mathrm{n},\mathrm{p})\,^{32}_{15}\mathrm{P}$}\label{Ej:09}
Calcule la energía umbral para la reacción ${}^{32}_{16}\mathrm{S}(\mathrm{n},\mathrm{p}){}^{32}_{15}\mathrm{P}$.
\end{Ejercicio}


La energía umbral, que sería $-Q$, viene dada por la diferencia de las masas (basicamente se llama $Q$ a la energía sobrante, es decir, si $Q>0$ tendríamos que la reacción sería espontánea):

\begin{equation}
    Q =m_n + m \pqty{^{32}_{16}\text{S}} -  m(^{32}_{15}\text{P}) - m_p  
\end{equation}
tal que (usando datos del \href{https://www.nndc.bnl.gov/nudat3/}{nndc}):

\begin{equation}
    Q = - 928.31 \unit{keV}
\end{equation}
sería la energía que tendríamos que aportar con energía cinética (energía umbral).

%%%%%%%%%%%%%%%%%%%%%%%%%%%%%%%%%%%%%%%%%%%%%%%%%%%%%%%%%%%%%
%%%%%%%%%%%%%%%%%%%%%%%%%%%%%%%%%%%%%%%%%%%%%%%%%%%%%%%%%%%%%
%%%%%%%%%%%%%%%%%%%%%%%%%%%%%%%%%%%%%%%%%%%%%%%%%%%%%%%%%%%%%
%%%%%%%%%%%%%%%%%%%%%%%%%%%%%%%%%%%%%%%%%%%%%%%%%%%%%%%%%%%%%
%%%%%%%%%%%%%%%%%%%%%%%%%%%%%%%%%%%%%%%%%%%%%%%%%%%%%%%%%%%%%
%%%%%%%%%%%%%%%%%%%%%%%%%%%%%%%%%%%%%%%%%%%%%%%%%%%%%%%%%%%%%
%%%%%%%%%%%%%%%%%%%%%%%%%%%%%%%%%%%%%%%%%%%%%%%%%%%%%%%%%%%%%
%%%%%%%%%%%%%%%%%%%%%%%%%%%%%%%%%%%%%%%%%%%%%%%%%%%%%%%%%%%%%

\begin{Ejercicio}{Sección eficaz n+$^{1}$H $\to$ $^{2}$H+$\gamma$}\label{Ej:10}
La sección eficaz de la reacción de captura con neutrones térmicos
\[
\mathrm{n} + {}^{1}\mathrm{H} \;\to\; {}^{2}\mathrm{H} + \gamma
\]
es de $0{,}33\,\mathrm{b}$. Estime la sección eficaz para neutrones de energía $10\,\mathrm{eV}$.
\end{Ejercicio}


Aquí tenemos qeu aplicar la \textit{ley 1/v}, qeu nos dice que la sección eficaz de captura es inversamente proporicional a la velocidad de la partícula. Es decir, el cociente de dos secciones eficaces de captura es: 

\begin{equation}
    \frac{\sigma}{\sigma_0} = \frac{v_0}{v} \approx \sqrt{\frac{E_0}{E}}
\end{equation}
ley que se sigue en el rango de energías 0.1 a 1 keV. Teniendo en cuenta que un neutrón térmico tiene una energía de $E_0\approx 0.025$ eV:

\begin{equation}
    \sigma = 0.0165 \unit{b}
\end{equation}

%%%%%%%%%%%%%%%%%%%%%%%%%%%%%%%%%%%%%%%%%%%%%%%%%%%%%%%%%%%%%
%%%%%%%%%%%%%%%%%%%%%%%%%%%%%%%%%%%%%%%%%%%%%%%%%%%%%%%%%%%%%
%%%%%%%%%%%%%%%%%%%%%%%%%%%%%%%%%%%%%%%%%%%%%%%%%%%%%%%%%%%%%
%%%%%%%%%%%%%%%%%%%%%%%%%%%%%%%%%%%%%%%%%%%%%%%%%%%%%%%%%%%%%
%%%%%%%%%%%%%%%%%%%%%%%%%%%%%%%%%%%%%%%%%%%%%%%%%%%%%%%%%%%%%
%%%%%%%%%%%%%%%%%%%%%%%%%%%%%%%%%%%%%%%%%%%%%%%%%%%%%%%%%%%%%
%%%%%%%%%%%%%%%%%%%%%%%%%%%%%%%%%%%%%%%%%%%%%%%%%%%%%%%%%%%%%
%%%%%%%%%%%%%%%%%%%%%%%%%%%%%%%%%%%%%%%%%%%%%%%%%%%%%%%%%%%%%

\begin{Ejercicio}{Ley $1/v$ en ${}^{197}\mathrm{Au}$}\label{Ej:11}
Suponiendo que la sección eficaz de captura del ${}^{197}\mathrm{Au}$ sigue la ley $1/v$, y que la sección eficaz de absorción para neutrones de $0{,}025\,\mathrm{eV}$ vale $0{,}99\,\mathrm{b}$, 
encuentre la sección eficaz de absorción del ${}^{197}\mathrm{Au}$ para neutrones de $1\,\mathrm{eV}$. 
¿Qué espesor debe tener una hoja de oro que absorba el $20\%$ de los neutrones de un haz de energía $1\,\mathrm{eV}$? ($\rho_\text{Au}$ = 19,3 g cm$^{-3}$)
\end{Ejercicio} 

Al igual que antes, debemos aplicar  \textit{la ley 1/v}, tal que: 

\begin{equation}
    \frac{\sigma}{\sigma_0} = \frac{v_0}{v} \approx \sqrt{\frac{E_0}{E}} 
\end{equation}
que nos lleva a: 

\begin{equation}
   \sigma =  0.1565 \ \unit{b}
\end{equation}

El espesor, dado que podemos suponer 
\begin{equation}
    N(x) = N_0 e^{-\mu x} \qquad \mu = \pqty{\frac{\rho N_A}{A}} \sigma 
\end{equation}
vienee dado por:

\begin{equation}
    \mu =0.00923 \ \unit{cm^{-1}} \qquad  x = \frac{\ln(5/4)}{\mu} = 24.16 \ \unit{cm}
\end{equation}

%%%%%%%%%%%%%%%%%%%%%%%%%%%%%%%%%%%%%%%%%%%%%%%%%%%%%%%%%%%%%
%%%%%%%%%%%%%%%%%%%%%%%%%%%%%%%%%%%%%%%%%%%%%%%%%%%%%%%%%%%%%
%%%%%%%%%%%%%%%%%%%%%%%%%%%%%%%%%%%%%%%%%%%%%%%%%%%%%%%%%%%%%
%%%%%%%%%%%%%%%%%%%%%%%%%%%%%%%%%%%%%%%%%%%%%%%%%%%%%%%%%%%%%
%%%%%%%%%%%%%%%%%%%%%%%%%%%%%%%%%%%%%%%%%%%%%%%%%%%%%%%%%%%%%
%%%%%%%%%%%%%%%%%%%%%%%%%%%%%%%%%%%%%%%%%%%%%%%%%%%%%%%%%%%%%
%%%%%%%%%%%%%%%%%%%%%%%%%%%%%%%%%%%%%%%%%%%%%%%%%%%%%%%%%%%%%
%%%%%%%%%%%%%%%%%%%%%%%%%%%%%%%%%%%%%%%%%%%%%%%%%%%%%%%%%%%%%

\begin{Ejercicio}{Activación de ${}^{197}\mathrm{Au}$ con neutrones térmicos}\label{Ej:07}
Se irradia una lámina de oro de $0{,}02\,\mathrm{cm}$ de grosor con un haz de neutrones térmicos con un flujo de $1\times10^{12}\,\mathrm{n\,cm^{-2}\,s^{-1}}$.  
El núcleo ${}^{198}\mathrm{Au}$ tiene un periodo de semidesintegración de $2{,}7\,\mathrm{d}$ y se produce en la reacción ${}^{197}\mathrm{Au}(\mathrm{n},\gamma){}^{198}\mathrm{Au}$.  
La densidad del oro es de $19{,}3\,\mathrm{g\,cm^{-3}}$ y la sección eficaz de la reacción anterior es $97{,}8\,\mathrm{b}$.  
Considerando que el ${}^{197}\mathrm{Au}$ tiene una abundancia natural del $100\%$:

\begin{enumerate}
\item[(a)] Determine la actividad de ${}^{198}\mathrm{Au}$ tras cinco minutos de irradiación de la lámina (en $\mathrm{Bq/cm^2}$).
\item[(b)] Determine el flujo máximo de ${}^{198}\mathrm{Au}$ que puede producirse en esta lámina.
\item[(c)] ¿Durante cuánto tiempo tendrá que irradiar la lámina para conseguir una actividad de $2/3$ de su actividad máxima?
\end{enumerate}
\end{Ejercicio}

Primero calculamos algunos datos: 

\begin{equation}
N = \frac{m}{M} \, N_A \;\;\approx\;\; 1.18 \times 10^{21} \;\; \text{átomos/cm}^2
\end{equation}
\begin{equation}
\lambda = \frac{\ln 2}{T_{1/2}} \;\;\approx\;\; 2.97 \times 10^{-6} \;\; \text{s}^{-1}
\end{equation}
\begin{equation}
R = N \, \phi \, \sigma \;\;\approx\;\; 1.15 \times 10^{11} \;\; \text{s}^{-1}\,\text{cm}^{-2}
\end{equation}
\begin{equation}
\sigma \, \phi = (97.8 \times 10^{-24}) \cdot (1 \times 10^{12}) \;\;\approx\;\; 9.78 \times 10^{-11} \;\; \text{s}^{-1}
\end{equation}
Si consideramos que la cantidad de oro 197 no varía (es decir, no se ``elimina'' el suficiente para que tengamos qeu considerar $N_{^{197}\text{Au}}(t)$, que es lo mismo que $\lambda_1 t= \sigma \Phi t \to 0$ y $\lambda_1\ll \lambda_2$, lo cual es verdad como podemos comprobar arriba), la ecuación que rige el número de núcleos de oro 198 es:
\begin{equation}
\dv{N(t)}{t} = - \lambda N(t) +  \sigma \Phi N_T \to  N(t) = \frac{N_T \Phi \sigma}{\lambda} (1-e^{-\lambda t})
\end{equation}
y por tanto la actividad es 

\begin{equation}
    A(t) = \sigma \Phi N_T \pqty{1-e^{-\lambda t}}
\end{equation}
\begin{enumerate}[label=\alph*)]
    \item La actividad tras 5 minutos (300 s) será: 
    \begin{equation}
        A(t=300\unit{s}) = 10^8 \unit{Bq/cm^2}
    \end{equation}
    \item El flujo máximo es lo mismo la actividad es máxima por unidad de superficie, que claramente ocurre cuando $t\to \infty$:
    \begin{equation}
        A_{\max} = N_T \sigma \Phi =  \SI{1.15e11}{cm^{-2}}
    \end{equation}
    \item El flujo máximo es lo mismo la actividad es máxima por unidad de superficie, que claramente ocurre cuando $t\to \infty$:
    \begin{equation}
        A (t)=  \frac{2}{3} A_{\max} \to 1-e^{-\lambda t} = \frac{2}{3} \to t = \frac{\log(3)}{\lambda} = \SI{3.7e5}{s} = \SI{4.3}{dias}
    \end{equation}
\end{enumerate}

%%%%%%%%%%%%%%%%%%%%%%%%%%%%%%%%%%%%%%%%%%%%%%%%%%%%%%%%%%%%%
%%%%%%%%%%%%%%%%%%%%%%%%%%%%%%%%%%%%%%%%%%%%%%%%%%%%%%%%%%%%%
%%%%%%%%%%%%%%%%%%%%%%%%%%%%%%%%%%%%%%%%%%%%%%%%%%%%%%%%%%%%%
%%%%%%%%%%%%%%%%%%%%%%%%%%%%%%%%%%%%%%%%%%%%%%%%%%%%%%%%%%%%%
%%%%%%%%%%%%%%%%%%%%%%%%%%%%%%%%%%%%%%%%%%%%%%%%%%%%%%%%%%%%%
%%%%%%%%%%%%%%%%%%%%%%%%%%%%%%%%%%%%%%%%%%%%%%%%%%%%%%%%%%%%%
%%%%%%%%%%%%%%%%%%%%%%%%%%%%%%%%%%%%%%%%%%%%%%%%%%%%%%%%%%%%%
%%%%%%%%%%%%%%%%%%%%%%%%%%%%%%%%%%%%%%%%%%%%%%%%%%%%%%%%%%%%%

\begin{Ejercicio}{Blanco de boro natural e irradiación de neutrones}\label{Ej:08}
Un blanco de boro natural (composición atómica: $20\%$ ${}^{10}_{5}\mathrm{B}$, $80\%$ ${}^{11}_{5}\mathrm{B}$), en forma de hoja delgada, con un espesor másico de $1\,\mathrm{kg\,m^{-2}}$, es atravesado normalmente por un haz de neutrones de energía cinética $1\,\mathrm{keV}$.  
El ${}^{10}\mathrm{B}$ es el único elemento que, de forma significativa, absorbe neutrones mediante la reacción $(\mathrm{n},\alpha)$. La sección eficaz para dicha reacción y a dicha energía de neutrones es de $19{,}3\,\mathrm{b}$.  

El núcleo residual queda en su estado fundamental en un $30\%$ de las reacciones, y en el resto de ellas en un estado excitado a aproximadamente $500\,\mathrm{keV}$, del que decae al fundamental emitiendo un único fotón. 

Establezca cuál es el núcleo residual y calcule la producción de fotones considerando que la intensidad del haz es de $1\times10^{5}\,\mathrm{s^{-1}}$ y que puede despreciar la dispersión elástica de neutrones por el blanco.
\end{Ejercicio}


Podemos calcular el número de átomos de $^{10}$B por unidad de superficie 
\begin{equation}
    N(^{10}\text{B}) = \frac{A(^{10}\text{B})}{A(^{10}\text{B})+4A(^{11}\text{B})} \frac{\rho N_A}{A(^{10}\text{B})} \ \unit{cm^{-2}} \qquad  N(^{10}\text{B}) = \SI{1.115e+21}{cm^{-2}}
\end{equation}
siendo $A_1\approx 10$ y $A_2 \approx 11$. El núcleo residual en la reacción $^{10}\text{B}(\text{n},\alpha)^{7}_{3}\text{Li}$ es el litio 7. Podemos suponer, dado que nos da información el ejercicio, que el decaimiento del litio 7 excitado es inmediato, y por tanto la actividad del litio 7 excitado será igual a la tasa de producción del mismo, que es: 

\begin{equation}
    N(\gamma) = \epsilon A = \epsilon N(^{10}\text{B}) \Phi \sigma
\end{equation}
donde $\epsilon=0.3$ es el número de excitaciones que se producen. Así pues: 

\begin{equation}
    N(\gamma) = \SI{1506.65}{s^{-1}}
\end{equation}

%%%%%%%%%%%%%%%%%%%%%%%%%%%%%%%%%%%%%%%%%%%%%%%%%%%%%%%%%%%%%
%%%%%%%%%%%%%%%%%%%%%%%%%%%%%%%%%%%%%%%%%%%%%%%%%%%%%%%%%%%%%
%%%%%%%%%%%%%%%%%%%%%%%%%%%%%%%%%%%%%%%%%%%%%%%%%%%%%%%%%%%%%
%%%%%%%%%%%%%%%%%%%%%%%%%%%%%%%%%%%%%%%%%%%%%%%%%%%%%%%%%%%%%
%%%%%%%%%%%%%%%%%%%%%%%%%%%%%%%%%%%%%%%%%%%%%%%%%%%%%%%%%%%%%
%%%%%%%%%%%%%%%%%%%%%%%%%%%%%%%%%%%%%%%%%%%%%%%%%%%%%%%%%%%%%
%%%%%%%%%%%%%%%%%%%%%%%%%%%%%%%%%%%%%%%%%%%%%%%%%%%%%%%%%%%%%
%%%%%%%%%%%%%%%%%%%%%%%%%%%%%%%%%%%%%%%%%%%%%%%%%%%%%%%%%%%%%z

\begin{Ejercicio}{Generadores de neutrones basados en aceleradores}\label{Ej:09}
Un ejemplo de fuente de neutrones son los generadores basados en aceleradores, en los que aceleradores compactos se utilizan para acelerar deuterio, y generar neutrones a través de las siguientes reacciones:
\[
\mathrm{d}+\mathrm{t}\equiv {}^{2}\mathrm{H}+{}^{3}\mathrm{H}\longrightarrow {}^{4}_{2}\mathrm{He}+\mathrm{n}+Q_{dt},
\]
\[
\mathrm{d}+\mathrm{d}\equiv {}^{2}\mathrm{H}+{}^{2}\mathrm{H}\longrightarrow {}^{3}_{2}\mathrm{He}+\mathrm{n}+Q_{dd}.
\]

\begin{enumerate}
\item[(a)] Calcule el valor $Q$ para cada reacción.
\item[(b)] Determine la energía cinética del neutrón y del ión de helio para cada reacción.
\item[(c)] De las dos reacciones, una de ellas es mucho más común que la otra en los generadores de neutrones. ¿Cuál es la preferente, y por qué?
\end{enumerate}
\end{Ejercicio}


\begin{enumerate}[label=\alph*)]
    \item Como ya hemos visto, $Q$ es el exceso de energía, que se evalua como la suma de las masas iniciales menos las finales, tal que:
    \begin{equation}
        Q =m\pqty{^{2}_{1}\text{H}}+ m \pqty{^{3}_{1}\text{H}} -  m\pqty{^{4}_{2}\text{He}} - \pqty{^{1}_{0}\text{n}} = 
    \end{equation}
    tal que (usando datos del \href{https://www.nndc.bnl.gov/nudat3/}{nndc}):

    \begin{equation}
        Q = 17589  \ \unit{keV}
    \end{equation}
    Mientras que la otra reacción: 
    \begin{equation}
        Q =m\pqty{^{2}_{1}\text{H}}+ m \pqty{^{2}_{1}\text{H}} -  m\pqty{^{3}_{2}\text{He}} - \pqty{^{1}_{0}\text{n}} = 
    \end{equation}
    tal que (usando datos del \href{https://www.nndc.bnl.gov/nudat3/}{nndc}):

    \begin{equation}
        Q = 3268.9 \ \unit{keV}
    \end{equation}
    \item La energía cinética de ambas partículas, suponiendo que se reparte toda la $Q$ en forma de energía cinética $E_k=Q$.
    \item Pues la que genera más energía cinética será la más probable: tiene más estados en el sistema fásico. Pues depende que queramos: si queremos neutrones rápidos está claro que la deuterio-tritio, si queremos neutrones más lentos la del helio-neutrón. 
\end{enumerate}

%%%%%%%%%%%%%%%%%%%%%%%%%%%%%%%%%%%%%%%%%%%%%%%%%%%%%%%%%%%%%
%%%%%%%%%%%%%%%%%%%%%%%%%%%%%%%%%%%%%%%%%%%%%%%%%%%%%%%%%%%%%
%%%%%%%%%%%%%%%%%%%%%%%%%%%%%%%%%%%%%%%%%%%%%%%%%%%%%%%%%%%%%
%%%%%%%%%%%%%%%%%%%%%%%%%%%%%%%%%%%%%%%%%%%%%%%%%%%%%%%%%%%%%
%%%%%%%%%%%%%%%%%%%%%%%%%%%%%%%%%%%%%%%%%%%%%%%%%%%%%%%%%%%%%
%%%%%%%%%%%%%%%%%%%%%%%%%%%%%%%%%%%%%%%%%%%%%%%%%%%%%%%%%%%%%
%%%%%%%%%%%%%%%%%%%%%%%%%%%%%%%%%%%%%%%%%%%%%%%%%%%%%%%%%%%%%
%%%%%%%%%%%%%%%%%%%%%%%%%%%%%%%%%%%%%%%%%%%%%%%%%%%%%%%%%%%%%

\begin{Ejercicio}{Radiación de Cerenkov}\label{Ej:10}
Un núcleo de tritio de $5\,\mathrm{GeV}$ de energía cinética atraviesa un medio transparente cuyo índice de refracción es $n=1{,}5$.  
Calcule el ángulo de emisión de la radiación de Cerenkov.
\end{Ejercicio}

El núcleo de tritio atraviesa un medio con una velocidad $\beta$, tal que el ángulo de Cerenkov viene dado por: 

\begin{equation}
    \cos (\theta_{\text{Cer}}) = \frac{1}{\beta n} \qquad \gamma = 2.78 \qquad \beta = \frac{p}{E} = 0.935
\end{equation}
\begin{equation}
    \theta_{\text{Cer}} = 44.40^{\circ}
\end{equation}


%%%%%%%%%%%%%%%%%%%%%%%%%%%%%%%%%%%%%%%%%%%%%%%%%%%%%%%%%%%%%
%%%%%%%%%%%%%%%%%%%%%%%%%%%%%%%%%%%%%%%%%%%%%%%%%%%%%%%%%%%%%
%%%%%%%%%%%%%%%%%%%%%%%%%%%%%%%%%%%%%%%%%%%%%%%%%%%%%%%%%%%%%
%%%%%%%%%%%%%%%%%%%%%%%%%%%%%%%%%%%%%%%%%%%%%%%%%%%%%%%%%%%%%
%%%%%%%%%%%%%%%%%%%%%%%%%%%%%%%%%%%%%%%%%%%%%%%%%%%%%%%%%%%%%
%%%%%%%%%%%%%%%%%%%%%%%%%%%%%%%%%%%%%%%%%%%%%%%%%%%%%%%%%%%%%
%%%%%%%%%%%%%%%%%%%%%%%%%%%%%%%%%%%%%%%%%%%%%%%%%%%%%%%%%%%%%
%%%%%%%%%%%%%%%%%%%%%%%%%%%%%%%%%%%%%%%%%%%%%%%%%%%%%%%%%%%%%

\begin{Ejercicio}{Umbral de Cerenkov en aire}\label{Ej:11}
Calcule la energía umbral por debajo de la cual no se observa radiación de Cerenkov cuando un electrón se mueve en el aire ($n=1{,}000293$).
\end{Ejercicio}

El valor mínimo de $\beta$ para el cual ya no se observa Cerenkov es: 

\begin{equation}
    \beta_{\min} = \frac{1}{n}  = 0.999707 \to 
\end{equation}
y por tanto la energía será: 

\begin{equation}
    E = 20.6 \ \unit{MeV}
\end{equation}

