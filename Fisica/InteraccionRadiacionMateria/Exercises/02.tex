
\newpage 
\section*{Ejercicios}

%%%%%%%%%%%%%%%%%%%%%%%%%%%%%%%%%%%%%%%%%%%%%%%%%%%%%%%%%%%%%%%%%%%
%%%%%%%%%%%%%%%%%%%%%%%%%%%%%%%%%%%%%%%%%%%%%%%%%%%%%%%%%%%%%%%%%%%
%%%%%%%%%%%%%%%%%%%%%%%%%%%%%%%%%%%%%%%%%%%%%%%%%%%%%%%%%%%%%%%%%%%
%%%%%%%%%%%%%%%%%%%%%%%%%%%%%%%%%%%%%%%%%%%%%%%%%%%%%%%%%%%%%%%%%%%

\addcontentsline{toc}{section}{Ejercicios}
\begin{Ejercicio}{Sección eficaz de Rutherford con modelo atómico de Fermi} \label{Ej:02.01}
    Obten la expresión de la sección de eficaz de Rutherford, ahora con el potencial del modelo atómico estadístico de Fermi, usando la aproximación de Born con ondas planas incidentes y salientes

    \begin{equation*}
        \dv{\sigma_{\Ruth}}{\Omega} =  \pqty{\frac{D_{\alpha-N}}{4}}^2 \frac{1}{\sin^4 \theta/2} \pqty{\frac{1}{1+\frac{1}{K^2 a_{TF}^2}}}
    \end{equation*}
    y luego llegar a la expresión final cuando $\theta\ll 1$: 
\end{Ejercicio}

%%%%%%%%%%%%%%%%%%%%%%%%%%%%%%%%%%%%%%%%%%%%%%%%%%%%%%%%%%%%%%%%%%%
%%%%%%%%%%%%%%%%%%%%%%%%%%%%%%%%%%%%%%%%%%%%%%%%%%%%%%%%%%%%%%%%%%%
%%%%%%%%%%%%%%%%%%%%%%%%%%%%%%%%%%%%%%%%%%%%%%%%%%%%%%%%%%%%%%%%%%%
%%%%%%%%%%%%%%%%%%%%%%%%%%%%%%%%%%%%%%%%%%%%%%%%%%%%%%%%%%%%%%%%%%%
%%%%%%%%%%%%%%%%%%%%%%%%%%%%%%%%%%%%%%%%%%%%%%%%%%%%%%%%%%%%%%%%%%%
%%%%%%%%%%%%%%%%%%%%%%%%%%%%%%%%%%%%%%%%%%%%%%%%%%%%%%%%%%%%%%%%%%%

\begin{Ejercicio}{$\Delta E_{\max}$ y $\Delta p_{\max}$ en colisión con blanco fijo.} \label{Ej:02.02}
    A partir de la conservación del cuadrimomento $p^{\mu}_i = p^{\mu}_f$ encuentra la expresión de máxima trasnferencia de la energía y del momento $\Delta E_{\max}$ y $\Delta p_{\max}$. 
    \begin{equation}
        \Delta E_{\max} = \frac{2(\gamma+1)m_1m_2}{m_1^2 + m_2^2 + 2 \gamma m_1 m_2} E_K^i \qquad 
        \Delta p_{\max} = \frac{2(m_1\gamma+m_2)m_2}{m_1^2 + m_2^2 + 2 \gamma m_1 m_2} p_i 
    \end{equation}
    Véase la solución en \cite{Montaruli201x_Exercise4}.
\end{Ejercicio}

%%%%%%%%%%%%%%%%%%%%%%%%%%%%%%%%%%%%%%%%%%%%%%%%%%%%%%%%%%%%%%%%%%%
%%%%%%%%%%%%%%%%%%%%%%%%%%%%%%%%%%%%%%%%%%%%%%%%%%%%%%%%%%%%%%%%%%%
%%%%%%%%%%%%%%%%%%%%%%%%%%%%%%%%%%%%%%%%%%%%%%%%%%%%%%%%%%%%%%%%%%%
%%%%%%%%%%%%%%%%%%%%%%%%%%%%%%%%%%%%%%%%%%%%%%%%%%%%%%%%%%%%%%%%%%%
%%%%%%%%%%%%%%%%%%%%%%%%%%%%%%%%%%%%%%%%%%%%%%%%%%%%%%%%%%%%%%%%%%%
%%%%%%%%%%%%%%%%%%%%%%%%%%%%%%%%%%%%%%%%%%%%%%%%%%%%%%%%%%%%%%%%%%%
\begin{Ejercicio}{Transferencia de momento y energía por partículas cargadas} \label{Ej:02.03}

Recuerda las relaciones obtenidas para la transferencia $\Delta p$ y $\Delta E$ en una colisión a través de la ley de Coulomb entre una partícula pesada de masa $M$ y carga $+Ze$ y un electrón orbital (carga $-e$ y masa $m_e$), con un valor fijo del parámetro de impacto $b$, teniendo en cuenta las leyes de conservación de $p$, $E$, y del momento angular $L$. En dicha colisión clásica y no relativista, el electrón orbital se sitúa en el foco interno de una hipérbola. Si la partícula cargada es pesada tenemos $M \gg m_e$ y el ángulo de dispersión es nulo $\theta \approx 0$, entonces:

\[ \scriptsize
\Delta p(b) = \frac{2 Z r_e m_e c^2}{v b}, \qquad 
\Delta E(b) = \frac{(\Delta p)^2}{2 m_e} = \frac{2 Z^2 r_e^2 m_e c^2}{(v/c)^2 b^2}, \qquad
b = \frac{2 Z r_e m_e c^2}{v \Delta p} = \frac{Z r_e (v/c)}{\sqrt{\Delta E / 2 m_e c^2 }}
\]

\begin{enumerate}[label=\alph*)]
\item Tanto $\Delta p$ como $\Delta E$ tienen un valor mínimo y un valor máximo. Recuerda cuál es su fundamento, y determina ambos en cada caso $\Delta p_{\min,\max}$ (en KeV/c) y $\Delta E_{\min,\max}$ (en eV), para un \emph{protón} incidente de energía cinética $E_K = 10$ MeV, añadiendo 4 filas más al Cuadro 3, cada una con los 7 elementos.

\item  Debido a su dependencia con el parámetro de impacto $b$, tenemos también un valor mínimo y máximo de éste. Añade al Cuadro 3 los valores de $b_{\min}$ y $b_{\max}$ (en fm) en dos nuevas filas. ¿Es posible la colisión entre una partícula cargada pesada y un electrón orbital del absorbente, estando el parámetro de impacto fuera del rango $b \in (b_{\min}, b_{\max})$? Discute por separado los casos $b > b_{\max}$ y $b < b_{\min}$, intentando caracterizarlos físicamente, si crees que aún son posibles. Aclara el marco de la aproximación realizada en la derivación de las fórmulas anteriores, especialmente para el segundo caso. ¿Son siempre elásticas las colisiones del proyectil con los \emph{núcleos atómicos}? \\[1em]
\end{enumerate}

\begin{center}
\begin{tabular}{lcccccccccccccccccccccccccccc}
\toprule & H & Al & Cu & Ag & Au & Tl & Pb \\
\midrule
Número atómico $Z$ & 1 & 13 & 29 & 47 & 79 & 81 & 82 \\
Potencial de ionización / excitación $I$ (eV) & 19 & 166 & 322 & 470 & 790 & 727 & 823 \\
\bottomrule
\end{tabular}
\end{center}
\captionsetup[table]{justification=centering}
\end{Ejercicio}

Lo primero que vamos a definir son los valores fundamentales. Suponiendo un protón incidente con una energía cinética de $E_K=10$ MeV, tal que $E_K\ll m_p c^2$ y por tanto el momento inicial aproximable a la relación clásica:

\begin{equation*}
    p_i \approx \sqrt{2 m_p E_K} = 136.99 \ \unit{MeV/c^2} \qquad \beta= 0.146  \qquad 
        v_i = \frac{p_i}{m_p} = 43769 \ \unit{km/s}
\end{equation*}

\begin{enumerate}[label=\alph*)]
    \item  Veamos ahroa el momento máximo y la energía máxima transferibles. Dado que estamos en una colisión no relativista, vienen dados por (básicamente $\gamma \to 1$ en las ecuaciones ya vistas): 
    \begin{equation*}
            \Delta E_{\max} \approx \frac{4m_p m_N}{m_p^2 + m_N^2 + 2m_p m_N} E_K^i \qquad 
            \Delta p_{\max} \approx \frac{2(m_p+m_N)m_N}{m_p^2 + m_N^2 + 2m_p m_N} p_i
    \end{equation*}
    donde $m_N$ es la masa del blanco. Dado que interacciona con un \textit{electrón orbital} $M_N=m_e$ y por tanto $\Delta E_{\max}$ para el protón será igual para todos los átomos con los que colisiona. Esta expresión proviene de la conservación de la energía-momento relativista, es decir, es la máxima energía transferible en una interacción relativista: energías mas altas no verificarían la conservación del cuadrimomento. De hecho, por definción es la transferencia de energía que ocurre cuando el ángulo de dispersión tiende a cero: 

    \begin{equation}
        \Delta E_{\max} = \lim_{\theta \to 0} \Delta E(\theta)
    \end{equation}
    
    
    
    La mínima transferencia de energía se da cuando el protón interacciona con la energía suficiente para romper el enlace del electró con el átomo. La manera en la que se tabula esta es directamente relacionado con el potencial de ionización \cite{Brau2014}, tal que 

    \begin{equation*}
        \Delta E_{\min} \approx I \qquad \Delta p_{\min}^2 \approx \frac{1}{c} \sqrt{(\Delta E_{\min}+m_ec^2)^2-m^2 c^4}
    \end{equation*}
    Los resultados numéricos de las \textit{transferencias de energía y momento} los hallamos en la \cref{Tab:02-ex03}.
    
    \item Ahora solo quedaría sustituir los $\Delta E_{\min}$ y $\Delta E_{\max}$ en las definción de $b(\Delta E)$, tal que aplicando la ecuación 
     \[
        b = \frac{2 Z r_e m_e c^2}{v \Delta p} = \frac{Z r_e (v/c)}{\sqrt{\Delta E/2 m_e c^2 }}
    \]
    añadiendo las dos últimas filas. Por definición $b_{\max} (\Delta E_{\min})$ y $b_{\min} (\Delta E_{\max})$. Esta ecuación se deduce a partir de la transferencia de momento $\Delta p$ en la trayectoria hiperbólica que realiza el protón en el \textit{scattering} de Coulomb, y en realidad es: 

    \begin{equation}
            b = \frac{2 Z r_e m_e c^2}{v \Delta p} \cos(\theta/2)
    \end{equation}
    La pregunta es: ¿podemos asumir que $\theta\approx 0$ a la hora de calcular $b_{\min}$ y $b_{\max}$? Lógicamente para calcular $b_{\max}$ nos interesa tener el ángulo más pequeño posible, siendo la única limitación el principio de incertidumbre. Por otro lado, para calcular $b_{\min}$ si podría parecer que nos interesa que $\theta \to \pi$ (aunque ya sabemos que $\theta <\theta_{\max}$ también del ppio. de incertidumbre), sin embargo $\Delta E \propto \cos^2 (\theta)$ (según siecs. relativistas), por lo que si queremos $\Delta E_{\max}$ no podemos tener $\theta \to \theta_{\max}$. 
    
    La conclusión: en ambos casos queremos un ángulo muy pequeño. Ahora bien, aun no hemos respondido al origen de $\theta_{\min}$ ¿Realmente tiende a cero? Según lo que vimos en clase: 

    \begin{equation}
        \theta_{\min} \approx \frac{\lambda }{\cdot a_F}
    \end{equation}
    siendo $\lambda = \hbar/p \approx 1.45$ fm. Como $a_F$ será $\sim 1 \unit{\angstrom}$ (0.5 para el hidrógeno, 4 para el plomo) tendremos $\theta \sim 1.5 \times 10^{-5}$ rad: podemos asumir que es cero. A pesar de sre un parámetro clásico, lo encontramos bien definido.  

    Nos queda responder a las preguntas siguientes.¿Es posible la colisión entre una partícula cargada pesada y un electrón orbital del absorbente, estando el parámetro de impacto fuera del rango $b \in (b_{\min}, b_{\max})$? Veamos. 
    \begin{itemize}
        \item Veamos el caso para $b>b_{\max}$. Cuando esto ocurre tenemos que la transferencia de energía es mínima. ¿Se puede producir una transferencia de energía por debajo de $I$? La respuesta es que un protón (o cualquier otra partícula) no puede interaccionar trasmitiendo energía con un \textit{electrón orbital} si no es capaz de darle más energía que la de ligadura (al menos en una colisión individual). 
        
        El valor $I$ es un promedio de las transiciones, obtenida a partir de diferentes dispersiones. En una dispersión individual (considerando que solo hay un átomo en el vacio con un protón incidente), es posible interacción con un $\Delta E_{\min}$, ya que la energía de ligadura de un electrón (como la del hidrógeno, 13.6 eV es menor que $I$). 
        
        También es posible una interacción elástica, en la que no se transfiera energía al átomo, conservándose la energía. 
        
        \item El caso para $b<b_{\min}$, que sucede para la transferencia de máxima energía, es bastante curioso. Para que $b=0$ la única condición, segń las ecuaciones, es que $\Delta p \to \infty$, lo cual es imposile, ya que en caso de interacción hay una transferencia de momento máxima limitada por las leyes de conservación fundamentales de la física. De hecho, Pogdorsark \cite{Podgorsak2022}: ``Energy transfer larger than $\Delta E_{\max}$ is physically impossible.''. 
    \end{itemize}
    
    Respecto a las colisiones con los núcleos atómicos, no siempre son elásticas: pueden ocurrir procesos inelásticos como excitación nuclear, emisión de partículas secundarias o reacciones nucleares, especialmente para proyectiles de alta energía. En el caso particular de los protones peude ocurrir la fusión (algunas cadenas de fusión incluyen protones colisionando con isótopos del helio). Las interacciones vistas aquí, donde se transfiere energía cinética no son elásticas. Los resultados numéricos de los \textit{parámetros de impacto} los hallamos en la \cref{Tab:02-ex03}.
\end{enumerate}
Siendo el resultado final, la siguiente tabla.

\begingroup
\makeatletter
\let\old@floatboxreset\@floatboxreset
\def\@floatboxreset{\old@floatboxreset\centering} % fuerza centrado dentro del float
\makeatother
\let\normalsize\scriptsize
\begin{table}
\caption{resultados del ejercicio.}
\label{Tab:02-ex03}
\begin{tabular}{cccccccccccccccccc}
\toprule
 & H & Al & Cu & Ag & Au & Tl & Pb \\
\midrule
Z & 1 & 13 & 29 & 47 & 79 & 81 & 82 \\
I [eV] & 19 & 166 & 322 & 470 & 790 & 727 & 823 \\
m [MeV/c$^2$] & $\SI{9.31e+02}{}$ & $\SI{2.51e+04}{}$ & $\SI{5.92e+04}{}$ & $\SI{1.00e+05}{}$ & $\SI{1.83e+05}{}$ & $\SI{1.90e+05}{}$ & $\SI{1.93e+05}{}$ \\
$\Delta E_{\max}$ [keV] & $\SI{2.18e+01}{}$ & $\SI{2.18e+01}{}$ & $\SI{2.18e+01}{}$ & $\SI{2.18e+01}{}$ & $\SI{2.18e+01}{}$ & $\SI{2.18e+01}{}$ & $\SI{2.18e+01}{}$ \\
$\Delta p_{\max}$ [keV/c] & $\SI{1.51e+02}{}$ & $\SI{1.51e+02}{}$ & $\SI{1.51e+02}{}$ & $\SI{1.51e+02}{}$ & $\SI{1.51e+02}{}$ & $\SI{1.51e+02}{}$ & $\SI{1.51e+02}{}$ \\
$\Delta E_{\min}$ [keV] & $\SI{1.90e-02}{}$ & $\SI{1.66e-01}{}$ & $\SI{3.22e-01}{}$ & $\SI{4.70e-01}{}$ & $\SI{7.90e-01}{}$ & $\SI{7.27e-01}{}$ & $\SI{8.23e-01}{}$ \\
$\Delta p_{\min}$ [keV/c] & $\SI{4.41e+00}{}$ & $\SI{1.30e+01}{}$ & $\SI{1.81e+01}{}$ & $\SI{2.19e+01}{}$ & $\SI{2.84e+01}{}$ & $\SI{2.73e+01}{}$ & $\SI{2.90e+01}{}$ \\
$b_{\max}$ [pm] & $\SI{4.46e+00}{}$ & $\SI{1.51e+00}{}$ & $\SI{1.08e+00}{}$ & $\SI{8.98e-01}{}$ & $\SI{6.92e-01}{}$ & $\SI{7.22e-01}{}$ & $\SI{6.78e-01}{}$ \\
$b_{\min}$ [pm] & $\SI{1.32e-01}{}$ & $\SI{1.32e-01}{}$ & $\SI{1.32e-01}{}$ & $\SI{1.32e-01}{}$ & $\SI{1.32e-01}{}$ & $\SI{1.32e-01}{}$ & $\SI{1.32e-01}{}$ \\
\bottomrule
\end{tabular}
\end{table}

\endgroup

%%%%%%%%%%%%%%%%%%%%%%%%%%%%%%%%%%%%%%%%%%%%%%%%%%%%%%%%%%%%%%%%%%%
%%%%%%%%%%%%%%%%%%%%%%%%%%%%%%%%%%%%%%%%%%%%%%%%%%%%%%%%%%%%%%%%%%%
%%%%%%%%%%%%%%%%%%%%%%%%%%%%%%%%%%%%%%%%%%%%%%%%%%%%%%%%%%%%%%%%%%%
%%%%%%%%%%%%%%%%%%%%%%%%%%%%%%%%%%%%%%%%%%%%%%%%%%%%%%%%%%%%%%%%%%%
%%%%%%%%%%%%%%%%%%%%%%%%%%%%%%%%%%%%%%%%%%%%%%%%%%%%%%%%%%%%%%%%%%%
%%%%%%%%%%%%%%%%%%%%%%%%%%%%%%%%%%%%%%%%%%%%%%%%%%%%%%%%%%%%%%%%%%%
%%%%%%%%%%%%%%%%%%%%%%%%%%%%%%%%%%%%%%%%%%%%%%%%%%%%%%%%%%%%%%%%%%%

\begin{Ejercicio}{Dependencias genéricas de la fórmula de Bethe con $Z, M, \beta$ y $z$} \label{Ej:02.04}
    

Se ha visto que el poder de frenado másico $S_\text{\text{col}}$ de un medio absorbente $(Z,A)$ para una 
partícula cargada $(z, M, \beta)$ viene dado por la fórmula cuántica y relativista de Bethe--Bloch, 
reforzada con los factores de Fano $(C,\delta)$:

\[
S_\text{\text{col}} = 4\pi N_e \left( \frac{e^2}{4\pi \epsilon_0} \right)^2 
\frac{z^2}{m_e c^2 \beta^2} 
\left[ \ln \frac{2 m_e c^2}{I} + \ln \frac{\beta^2}{1 - \beta^2} - \beta^2 - \frac{C}{Z} - \delta \right]
\equiv C_1 \frac{N_e z^2}{\beta^2} \, \bar{B}_\text{\text{col}}
\]

donde $N_e \equiv ZN_A/A$ es el número de \emph{electrones por gramo} del medio absorbente.  

Para focalizar y separar adecuadamente las dependencias características de $S_\text{\text{col}}$ con los  parámetros de la partícula y del medio, responde razonadamente a las preguntas siguientes:

\begin{enumerate}[label=\alph*)]

\item ¿Hizo Bethe la hipótesis original de que la velocidad de la partícula era muy superior a la velocidad de los electrones atómicos $v \gg v_\text{orb}$?  
¿Sobreestima esto el potencial de ionización $I$ cuando no lo es?   Explica por qué son los electrones de la capa $K$ los más afectados en el término $C/Z$, y por qué la corrección es negativa.

\item ¿Por qué la corrección por densidad $\delta$ es más importante para las colisiones 
\emph{distantes} (suaves) y por qué es negativa?  
¿Sabes, sin embargo, si dicha corrección es también importante en el límite ultrarrelativista para electrones y positrones?

\item Observa que la dependencia en $Z$ de $S_\text{\text{col}}$ ocurre a través de dos vías distintas: 
una \emph{directa} en $N_e$, y otra \emph{indirecta} a través de $I(Z)$.  
Comenta separadamente sobre ellas.  
¿Empujan ambas en la misma dirección de subir o bajar $S_\text{\text{col}}$?  
¿Por qué, pese a la gran diferencia de los potenciales de ionización (entre $I=19$ eV para H y $I\sim 900$ eV para el Uranio), la dependencia con $I(Z)$ es suave?

\item ¿Puede decirse que, para una velocidad fija $\beta = v/c$ (o energía cinética fija $E_K$), 
$S_\text{\text{col}}$ es \emph{independiente} de la masa del proyectil $M$?  
Comenta sobre esto.

\item Analiza la dependencia de $S_\text{\text{col}}$ con $\beta$, señalando los términos específicos que 
son decisivos en cada una de las \textbf{tres regiones}:  
baja velocidad (con Fano), velocidad relativista intermedia, y velocidad ultrarrelativista.  
Comenta sobre el aumento o disminución de $S_\text{\text{col}}$ con $E_K$ en cada caso.

\item ¿Cómo depende $S_\text{\text{col}}$ de la carga $z$ de la partícula incidente?  
¿Existen también aquí dependencias indirectas?

\end{enumerate}
\end{Ejercicio}


Las solución al ejercicio, apartado por apartado: 
\begin{enumerate}[label=\alph*)]
    \item Si, efecivamente, Bethe supuso que la velocidad de la partícula era muy superior al a velocidad de los electrones atómicos. Esto efectivamente puede entenderse como un sobreestimamiento de la capacidad de ionización (``This effect causes  overestimate in the mean ionization/excitation energy I at low energies and results in an underestimate in Scol calculated from the Bethe equation.'', Pogdorsark \cite{Podgorsak2022}),que se manifiesta por ejemplo en la corrección de las capas $K$, a saber, $C/Z$.
    
    Citando al William R. Leo ``The shell correction accounts for effects which arise when the velocity of the incident particle is comparable or smaller than the orbital velocity of the bound electrons. At such energies, the assumption that the electron is stationary with ;espect to the incident particle is no longer valid and the Bethe-Bloch formula breaks down.'' \cite{Leo1994}.

    Por otro lado, Pogdorsark: ``Orbital electrons stop participating in energy transfer from the charged particle when their  velocity becomes comparable to the charged particle velocity.'' y ``Since the K shell electrons are the fastest of all orbital electrons, they are the first to be affected by low particle velocity with decreasing particle velocity. The shell correction is often addressed as the K shell correction, labeled $C/Z$ and all possible higher shell corrections are usually ignored.'' \cite{Podgorsak2022}.

    Desde un punto de vista individual podemos pensar que este sobreestimamiento puede venir de que la partícula puede ser que no tenga la energía suficiente para ionizar, puede que se supriman ciertas transiciones que $I$ está teniendo en cuenta lo cual se ve en: ``a subset of the included transitions is in fact not physically accessible (in particular, the transitions in the continuous spectrum) , thus the $I$ is then overestimated.''.  $C/Z$ será una función del medio absorbente y la velocidad de la partícula incidente. 

    \item Supongamos una partícula que entra en un medio. Las interacciones de colisiónes serán por Coulomb. En general podemos asumir que el medio es neutro, y por tanto que la partícula interacciona con todas las partículas a su alrededor. Sin embargo esto no es así, ya que si una partícula tiene una energía alta (su velocidad es grande) el campo eléctrico que produce se ``aplana'', i.e. tiene una componente transversal más fuerte que longitudinal (ya no es esféricamente simétrico). Esto induce un campo eléctrico fuerte capaz de generar dipolos a su paso, lo que polariza el medio, generando un campo eléctrico que apantalla a los electrones más alejados de la partícula, lo que disminuye su capacidad de interacción y por tanto disminuye el frenado. 
    
    En palabras del William R. Leo: ``The density effect arises from the fact that the electric field of the particle also tends  to polarize the atoms along its path. Because of this polarization, electrons far from the  path of the particle will be shielded from the full electric field intensity. Collisions with  these outer lying electrons will therefore contribute less to the total energy loss than  predicted by the Bethe-Bloch formula.'' \cite{Leo1994}.

    Debería ser importante para $S_{\text{col}}$ para los $e^-$ y $e^+$, ya que $\delta$ depende únicamente del material y de la velocidad de la partícula incidente (los electrones y positrones suelen ser relativistas), no de su masa ni de su carga. Sin embargo podría despreciarse, al menos tanto en cuanto en el poder de frenado total, ya que en el límite ultrarrelativista $S_{\text{rad}}$ domina frente a $S_{\text{col}}$.  ``For heavy charged particles the density correction is important at relativistic energies and negligible at intermediate and low energies; however, for electrons and positrons it plays a role in stopping power formulas at all energies.'' \cite{Podgorsak2022}.

    \item $N_e \propto Z$ y $I \propto Z$, pero $I$ se encuentra dentro de un logaritmo, $S_{\text{col}} \propto Z \ln (C_0/Z)$, en realidad $S_{\text{col}}$ es proporcional a $Z$. En otras palabras, en realidad $S_{\text{col}}$ depende del logaritmo de la inversa de $I(Z)$, lo que limita mucho el ``poder'' de $I(Z)$ sobre $S_{\text{col}}$, al ser el logaritmo al ser una de las funciones de crecimiento más lento. 
    \item La dependencia con la masa del proyectil se da, siendo precisos, siempre, aunque el cambio de masa cuando ya estamos hablando de partículas másicas incidentes, es despreciable. Sin embargo debería ser notable cuando pasamos de, por ejemplo, protones a electrones. 
    \item Como podemos ver, la dependencia con la velocidad es complicada. Veamos las tres regiones.
    \begin{itemize}
        \item Cuando $\beta$ tiende a cero vemos claramente que la ecuación nos queda como: 
        \[
        S_\text{\text{col}} = \underbrace{4\pi N_e \left( \frac{e^2}{4\pi \epsilon_0} \right)^2 
        z^2}_{\alpha}\frac{1}{m_e c^2 \beta^2} 
        \left[ \ln \frac{2 m_e \beta^2 c^2}{I} - \beta^2 \right]  = \alpha \frac{M}{m_e} \frac{1}{2E_K} \bqty{\ln \frac{4m_e}{M}\frac{E_K}{I}} - C_0
        \] 
        siendo $C_0$ la constante que se sacaría del término $-\beta^2$ dentro del corchete. No depende de la energía cinética. Si hacemos la derivada respecto la energía: 
        \[
        \dv{S_{\text{col}}}{E_K} = \frac{\alpha M}{2m_e E_K^2} \bqty{ 1  - \ln \pqty{\frac{4m_e}{M}\frac{E_K}{I}}}
        \]
        que como podemos ver tiene un crecimiento positivo y exponencial al principio hasta alcanzar el máximo, en: 
        \[
        \left. \dv{S_{\text{col}}}{E_K}  \right|_{\max} =  \dv{S_{\text{col}}}{E_K}  (E_{K\max}); \qquad E_{K\max} =2.71 \cdot  \frac{M}{4m_e} I
        \]
        (donde 2.71 viene del número $e$) que por ejemplo en el caso de $M=m_p$ (partícula incidente protón) tenemos
        \[ 
        \qquad E_{K\max} \approx 1248 \cdot I
        \]
        que en el caso de protones en el aluminio implicaría $E_{K\max} \approx 0.207 \unit{MeV}$, lo cual parece coincidir con los datos, al menos en oreden de magnitud. Según la ecuación de Bethe más básica (aproximación a energía clásica $E_K = \frac{1}{2} M\beta^2 c^2$, sin tener en cuenta más correcciones) deberíamos tener un crecimiento exponencial.
        \item Cuando $E_{K}$ pasa esta barrera, comienza a decrecer rápidamente, hasta un punto en el que alcanzamos un mínimo. Esta región continua dominada principalmente por la parte 
        \[
        S_\text{\text{col}} = 4\pi N_e \left( \frac{e^2}{4\pi \epsilon_0} \right)^2 
        \frac{z^2}{m_e c^2 \beta^2} 
        \left[ \ln \frac{2 m_e c^2}{I}  \right]
        \]
        hasta una región, a partir la cual el término $ \ln \frac{\beta^2}{1 - \beta^2} $ comienza a tomar relevancia. En ese momento aparece un mínimo que marca esta región de decrecimiento proporcional a $S_{\text{col}} \propto \frac{1}{E_K} \ln (E_K)$.        
        
        \item Finalmente, el último término, que se da cuando $\beta \to 1$, una vez pasado el mínimo, en el que se tiene en cuenta
        \[
        S_\text{\text{col}} = 4\pi N_e \left( \frac{e^2}{4\pi \epsilon_0} \right)^2 
        \frac{z^2}{m_e c^2 \beta^2} 
        \left[ \ln \frac{2 m_e c^2}{I} + \ln \frac{\beta^2}{1 - \beta^2} - \beta^2 \right]
        \]
        Cuya derivada respecto a $\beta$ es: 
        \[
        \dv{S_\text{\text{col}}}{\beta} = \alpha^* \bqty{\frac{-2}{\beta^3} \ln \pqty{\frac{\beta^2}{1-\beta^2}} + \frac{2}{\beta^3} + \frac{2\beta}{1-\beta^2} }
        \]
        donde podemos ver que cuando $\beta \to 1$ domina claramente término $ \frac{2\beta}{1-\beta^2} $, con un crecimiento cada vez mayor. Así pues, podemos afirmar que domina claramente 
        \[ 
        S_\text{\text{col}} = 4\pi N_e \left( \frac{e^2}{4\pi \epsilon_0} \right)^2 
        \frac{z^2}{m_e c^2 \beta^2} \ln \pqty{\frac{\beta^2}{1-\beta^2}} 
        \]
        Aunque como ya hemos visto $\delta(\beta)$ tiene una gran importancia a energías relativistas, por lo que podemos consierar una expresión más correcta con este factor (en este régimen). 
    \end{itemize}
    \item Podemos ver una dependencia directa (cuadrática) en $S_{\text{col}}$, sin dependencias indirectas, al menos en la ecuación de Bethe relativista con la corrección de densidad y de capas. Si incluyeramos los términos de Barkas y Bloch tendríamos más dependencias con $z$. 
\end{enumerate}


%%%%%%%%%%%%%%%%%%%%%%%%%%%%%%%%%%%%%%%%%%%%%%%%%%%%%%%%%%%%%%%%%%%
%%%%%%%%%%%%%%%%%%%%%%%%%%%%%%%%%%%%%%%%%%%%%%%%%%%%%%%%%%%%%%%%%%%
%%%%%%%%%%%%%%%%%%%%%%%%%%%%%%%%%%%%%%%%%%%%%%%%%%%%%%%%%%%%%%%%%%%
%%%%%%%%%%%%%%%%%%%%%%%%%%%%%%%%%%%%%%%%%%%%%%%%%%%%%%%%%%%%%%%%%%%
%%%%%%%%%%%%%%%%%%%%%%%%%%%%%%%%%%%%%%%%%%%%%%%%%%%%%%%%%%%%%%%%%%%
%%%%%%%%%%%%%%%%%%%%%%%%%%%%%%%%%%%%%%%%%%%%%%%%%%%%%%%%%%%%%%%%%%%
%%%%%%%%%%%%%%%%%%%%%%%%%%%%%%%%%%%%%%%%%%%%%%%%%%%%%%%%%%%%%%%%%%%


\begin{Ejercicio}{Comparativa del frenado en agua de distintos iones ligeros}

Compara de forma sistemática los resultados para el poder de frenado másico en agua ($S_{\text{col}}$) de distintos haces de iones ligeros, de acuerdo con la fórmula (1) de Bethe, despreciando las correcciones de Fano. Para ello:  

\begin{enumerate}[label=\alph*)]

\item Calcula $S_{\text{col}}$ (en MeV$\cdot$cm/g) para un protón incidente en agua de energía cinética $E_K = 51$ MeV. Toma $I = 75$ eV y la densidad electrónica $N_e = 3.343 \times 10^{23}$ e/g (como se determinó en un Problema anterior).  

\item  Calcula la energía cinética incidente $E_K$ (en MeV) de un deuterón cuyo poder de frenado en agua $S_{\text{col}}$ sea idéntico al del protón del apartado a).  

\item Calcula la energía cinética incidente $E_K$ (en MeV) y el poder de frenado másico en agua $S_{\text{col}}$ de las siguientes partículas: $p, d, \alpha, \, C^6$ y $Ne^{10}$ (iones ligeros), con la \emph{misma velocidad incidente} que el protón del apartado a).  
\end{enumerate}
Como resumen, elabora una Tabla, indicando en 5 columnas (para $p, d, \alpha, \text{C}^6$ y $\text{Ne}^{10}$) los 6 valores siguientes (por filas):  

\begin{enumerate}
\item $Mc^2$ (en MeV)  
\item. $Mc^2/(m_p c^2)$  
\item energía cinética $E_K$ (MeV) con igual poder de frenado en agua que el protón de 51 MeV  
\item Cociente $E_K/A$ (MeV/u)  
\item $z$  
\item $S_{\text{col}}$ (MeV$\cdot$cm/g)  
\end{enumerate}

¿Cuál de las filas resulta tener igual valor aproximado en todos los casos? Indica la magnitud física que es común para todos los iones evaluados. Compara el valor obtenido de $S_{\text{col}}$ en agua para partículas $\alpha$ de $E_K = 202.5$ MeV con el que encuentras en NIST.

\end{Ejercicio}

Nos dicen que calculemos $S_{\text{col}}$ usando la ecuación de Bethe despreciando las corecciones de Fano ($\delta, C/Z$), es decir, usando la ecuación 

\begin{equation*}
S_\text{\text{col}} = 4\pi N_e \left( \frac{e^2}{4\pi \epsilon_0} \right)^2 
\frac{z^2}{m_e c^2 \beta^2} 
\left[ \ln \frac{2 m_e c^2}{I} + \ln \frac{\beta^2}{1 - \beta^2} - \beta^2  \right]
\end{equation*}

\begin{enumerate}[label=\alph*)]
    \item Facilmente podemos obtener que para $E_K=51$ MeV tenemos una $\beta = 0.32$, que sabiendo $z$, $N_e$ e $I$ podemos sustituir en $S_{col}$, obteniendo 
    \[ S_{\text{col}} = 12.264 \ \unit{MeV \cdot cm^2 / g}\]
    \item Dado que $\beta^2$ no es muy grande $(\beta^2 \approx 0.10$), suponemos en este momento que podemos asumir la ecuación de Bethe sin los términos $\ln(\beta^2/(1-\beta^2))$ y $-\beta^2$. Así resulta obvio que la relación $z$ y $\beta$ debe ser
    \begin{equation*}
        S_{\text{col}} (\beta_1,z_1) = S_{\text{col}} (\beta_2,z_2) \Rightarrow \frac{z_1}{\beta_1} = \frac{z_2}{\beta_2} \to \beta_1 = \frac{z_1}{z_2} \beta_2
    \end{equation*}
    dado que el deuterón y el protón tiene el mismo $z$, $\beta_d = \beta_p$, y por tanto la enerǵia cinética vendrá dada por 
    \begin{equation*}
        E_{k,d} = (\gamma(\beta)-1) m_d c^2
    \end{equation*}
    lo qeu resulta en poco más del doble de energía cinética $E_{k,d} = 102.07$ MeV (ya que el neutrón pesa un poco más que el protón). 
    \item Con $\beta$ igual, el único factor que diferencia a los $S_{col}$ es $z$. Obtenemos los resultados en la última fila de la \cref{Tab:02-ex05}.
\end{enumerate}

\begingroup
\makeatletter
\let\old@floatboxreset\@floatboxreset
\def\@floatboxreset{\old@floatboxreset\centering} % fuerza centrado dentro del float
\makeatother
\begin{table}[H] \centering
\caption{resultados del ejercicio.}
\label{Tab:02-ex05}
\begin{tabular}{cccccccccccccccc}
\toprule
 & $p$ & $d$ & $\alpha$ & C$^{6}$ & Ne$^{10}$ \\
\midrule
$Mc^2$ [MeV] & 938.272 & 1877.838 & 3755.675 & 5629.633 & 9382.721 \\
$M/m_p$ & 1.000 & 2.001 & 4.003 & 6.000 & 10.000 \\
$E_k$ [MeV] & 51.000 & 102.070 & 1100.144 & NaN & NaN \\
$E_k/A$ [MeV] & 51.000 & 51.035 & 275.036 & NaN & NaN \\
$z$ & 1 & 1 & 2 & 6 & 10 \\
$S_{\text{col}}$ [MeV$\cdot$ cm$^2$ /g] & 12.264 & 12.264 & 49.058 & 441.522 & 1226.450 \\
\bottomrule
\end{tabular}
\end{table}

\endgroup


