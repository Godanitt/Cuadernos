
%%%%%%%%%%%%%%%%%%%%%%%%%%%%%%%%%%%%%%%%%%%%%%%%%%
%%%%%%%%%%%%%%%%%%%%%%%%%%%%%%%%%%%%%%%%%%%%%%%%%%
%%%%%%%%%%%%%%%%%%%%%%%%%%%%%%%%%%%%%%%%%%%%%%%%%%
%%%%%%%%%%%%%%%%%%%%%%%%%%%%%%%%%%%%%%%%%%%%%%%%%%

\begin{Ejercicio}{Espinores}
	Sean $\psi_\downarrow$ y $\xi_\uparrow$ dos espinores de Weyl zurdos y diestros independientes, respectivamente.  Demuestre que $\xi_\uparrow^\dagger \sigma^\mu \psi_\uparrow$ y $\xi_\downarrow^\dagger \bar{\sigma}^\mu \psi_\downarrow$ son cuatro-vectores contravariantes,  donde $\sigma^\mu \equiv (1,\symbf{\sigma})$ y $\bar{\sigma}^\mu \equiv (1,-\symbf{\sigma})$.
\end{Ejercicio}

Al ser $\sigma^\mu = (1,\sigman)$, la demostración se puede dividir en dos fases, en demostrar que $v^0=\xi_\downarrow^\dagger \psi_\downarrow$ y que $v^i =\xi_\downarrow^\dagger  \bar{\sigma}^i \psi_\downarrow$ (así mismo para los espinores $R$) son efectivamente invariantes Lorentz. Las transformaciones de Lorentz sobre un eje (por ej. el eje x):
\begin{eqnarray}
	V^0 \to V'^0 = \cosh (\eta) V^0 + \sinh (\eta)  V^{1}\\
	V^1 \to V'^1 = \cosh (\eta) V^1 + \sinh (\eta)  V^{0}
\end{eqnarray}
siendo $V'^2=V^2$ y $V'^3=V^3$. Con hacerlo sobre un eje basta, ya que siempre podremos encontrar un sistema de referencia donde el boost sea sobre ese eje x. Si nuestros cuatro-vecotres siguen dichas transformaciones, podemos decir que son contravariantes. Veamos uno por uno los dos casos:

\begin{itemize}
	\item Caso $\xi_\uparrow^\dagger \sigma^\mu \psi_\uparrow$. Una transformación Lorenzt para los espinores de Weyl a derechas es:
	      \begin{equation}
		      \Lambda_\uparrow = \exp \pqty{(-i\thetan  + \etan)\cdot \sigman}
	      \end{equation}
	      y recordemos que
	      \begin{equation}
		      v^0=\xi_\uparrow^\dagger \psi_\uparrow \qquad v^i =\xi_\uparrow^\dagger  \bar{\sigma}^i \psi_\uparrow
	      \end{equation}

	      Si el boost lo hacemos en el eje $x$, el único valor no nulo de $\etan$ es $\eta_1=\eta$. Así pues:
	      \begin{equation}
		      \Lambda_\uparrow = \exp \pqty{\eta  \sigma_1/2}
	      \end{equation}
	      Ahora tenemos que aplicar esto a nuestro $v_0$:

	      \begin{equation}
		      v^0 \to v^{\prime 0}= \xi_\uparrow^{\prime \dagger} \psi_\uparrow' = (\Lambda_\uparrow \xi_\uparrow^\dagger) (\Lambda_\uparrow \psi_\uparrow) = e^{\pqty{ \eta  \sigma^1}}\xi_\uparrow^\dagger \psi_\uparrow
	      \end{equation}
	      Usando que $e^{\pqty{ \eta  \sigma_1}} = \cosh (\eta)+\sigma^1 \eta $, tenemos que:
	      \begin{equation}
		      v^{\prime 0}= \cosh (\eta) \xi_\uparrow^\dagger \psi_\uparrow  + \sinh (\eta) \xi_\uparrow^\dagger  \sigma^1 \psi_\uparrow = \cosh v^0 + \sinh \eta v^1 =  \Lambda v^0
	      \end{equation}
	      q.e.d. Por otro lado, nos queda demostrar para $v^i$:

	      \begin{equation}
		      v^1 \to v^{\prime 1}= \xi_\uparrow^{\prime \dagger} \sigma_1 \psi_\uparrow' = \cosh (\eta) \xi_\uparrow^\dagger \sigma^1 \psi_\uparrow  + \sinh (\eta) \xi_\uparrow^\dagger \psi_\uparrow = \cosh (\eta) v^1 + \sinh (\eta) v^0
	      \end{equation}
	      donde nos hemos saltados algunos de los pasos. Dado que $\eta_2=\eta_3=0$, es trivial que $v^{\prime 2} = v^2$ y $v^{\prime 3} = v^3$, de lo que se deduce que efectivamente $xi_\uparrow^{\prime \dagger} \sigma^\mu \psi_\uparrow'$ transforma como un \text{4-vector contravariante}.
	\item El caso para la izquierda es análogo, aunque un poco didferente. En este caso
	      \begin{equation}
		      \Lambda_\downarrow = \exp \pqty{-\eta  \sigma_1/2} \qquad \exp \pqty{-\eta  \sigma_1} = \cosh (\eta) - \sinh (\eta) \eta_1
	      \end{equation}
	      lo cual hace que aparezca un signo menos, pero que debido a $\bar{\sigma}^\mu = (1,-\sigman)$, desaparece. Como hemos dicho, es repetir pasos.
\end{itemize}
pág. 267-268 del Maggiore \cite{Maggiore:2005qv}.

%%%%%%%%%%%%%%%%%%%%%%%%%%%%%%%%%%%%%%%%%%%%%%%%%%
%%%%%%%%%%%%%%%%%%%%%%%%%%%%%%%%%%%%%%%%%%%%%%%%%%
%%%%%%%%%%%%%%%%%%%%%%%%%%%%%%%%%%%%%%%%%%%%%%%%%%
%%%%%%%%%%%%%%%%%%%%%%%%%%%%%%%%%%%%%%%%%%%%%%%%%%

\begin{Ejercicio}{Transformaciones de Lorentz}
	Usando la representación quiral, demuestre que las transformaciones de Lorentz de los espinores de Dirac pueden escribirse en términos de las matrices $\gamma$ como
	\[
		\Psi \;\longrightarrow\; \Psi' = \exp\!\left(-\tfrac{i}{4}\omega_{\mu\nu}\sigma^{\mu\nu}\right)\Psi.
	\]
	Esta ecuación nos dice que $S^{\mu\nu} = \sigma^{\mu\nu}/2$ proporciona una representación de dimensión compleja cuatro del álgebra de Lorentz.
	Compruébelo directamente usando la definición de $\sigma^{\mu\nu}$ en términos de las matrices $\gamma$ y la relación $\{\gamma^\mu,\gamma^\nu\}=2g^{\mu\nu}$, es decir, verifique la siguiente relación de conmutación
	\[
		[S^{\mu\nu},S^{\rho\sigma}] = i\Big[g^{\nu\rho}S^{\mu\sigma} + g^{\mu\sigma}S^{\nu\rho} - g^{\nu\sigma}S^{\mu\rho} - g^{\mu\rho}S^{\nu\sigma}\Big].
	\]
\end{Ejercicio}

%%%%%%%%%%%%%%%%%%%%%%%%%%%%%%%%%%%%%%%%%%%%%%%%%%
%%%%%%%%%%%%%%%%%%%%%%%%%%%%%%%%%%%%%%%%%%%%%%%%%%
%%%%%%%%%%%%%%%%%%%%%%%%%%%%%%%%%%%%%%%%%%%%%%%%%%
%%%%%%%%%%%%%%%%%%%%%%%%%%%%%%%%%%%%%%%%%%%%%%%%%%


\begin{Ejercicio}{Identidad de Dirac}
	Demuestre la relación
	\[
		\frac{i}{\slashed{q}-m} = \frac{i(\slashed{q}+m)}{q^2-m^2}.
	\]
\end{Ejercicio}

%%%%%%%%%%%%%%%%%%%%%%%%%%%%%%%%%%%%%%%%%%%%%%%%%%
%%%%%%%%%%%%%%%%%%%%%%%%%%%%%%%%%%%%%%%%%%%%%%%%%%
%%%%%%%%%%%%%%%%%%%%%%%%%%%%%%%%%%%%%%%%%%%%%%%%%%
%%%%%%%%%%%%%%%%%%%%%%%%%%%%%%%%%%%%%%%%%%%%%%%%%%


\begin{Ejercicio}{Flujo invariante}
	Demuestre la siguiente identidad referente al flujo invariante de Lorentz,
	\[
		F = 4E_aE_b(v_a+v_b) = 4\sqrt{(p_a\cdot p_b)^2 - m_a^2 m_b^2}.
	\]
\end{Ejercicio}

%%%%%%%%%%%%%%%%%%%%%%%%%%%%%%%%%%%%%%%%%%%%%%%%%%
%%%%%%%%%%%%%%%%%%%%%%%%%%%%%%%%%%%%%%%%%%%%%%%%%%
%%%%%%%%%%%%%%%%%%%%%%%%%%%%%%%%%%%%%%%%%%%%%%%%%%
%%%%%%%%%%%%%%%%%%%%%%%%%%%%%%%%%%%%%%%%%%%%%%%%%%


\begin{Ejercicio}{Helicidad y Hamiltoniano de Dirac}
	Demuestre que el operador de helicidad conmuta con el hamiltoniano de Dirac, $[\hat{h},H_D]=0$, donde
	\[
		\hat{h} = \frac{\symbf{\Sigma}\cdot \hat{\mathbf{p}}}{2p}
		= \frac{1}{2p}\begin{pmatrix} \symbf{\sigma}\cdot\hat{\mathbf{p}} & 0 \\ 0 & \symbf{\sigma}\cdot\hat{\mathbf{p}} \end{pmatrix},
		\quad
		\hat{H}_D = \symbf{\alpha}\cdot\hat{\mathbf{p}} + \beta m,
	\]
	con
	\[
		\alpha_k = \gamma^0 \gamma^k, \qquad \beta = \gamma^0.
	\]
\end{Ejercicio}

%%%%%%%%%%%%%%%%%%%%%%%%%%%%%%%%%%%%%%%%%%%%%%%%%%
%%%%%%%%%%%%%%%%%%%%%%%%%%%%%%%%%%%%%%%%%%%%%%%%%%
%%%%%%%%%%%%%%%%%%%%%%%%%%%%%%%%%%%%%%%%%%%%%%%%%%
%%%%%%%%%%%%%%%%%%%%%%%%%%%%%%%%%%%%%%%%%%%%%%%%%%


\begin{Ejercicio}{Sección eficaz diferencial $e^-\mu^-$}\label{Ej:12}
	Usando amplitudes de helicidad, calcule la sección eficaz diferencial para el proceso
	\[
		e^- \mu^- \;\to\; e^- \mu^-,
	\]
	en el límite relativista ($m_e \approx 0$, $m_\mu \approx 0$), en el sistema del centro de masas.
\end{Ejercicio}

Nos piden la sección eficaz diferencial del scattering elástico $e^- \mu^- \;\to\; e^- \mu^-,$. Para ello tenemos que usar la ecuación:

\begin{equation}
	\dv{\sigma}{\Omega} = \frac{1}{64 \pi^2 s} \frac{p_f}{p_i} \vqty{\Mcal_{fi}}^2
\end{equation}
donde $\Mcal$ podemos hallarlo a través del siguiente diagrama de Feynmann a primer orden
\begin{figure}[h]
	\centering
	\feynmandiagram [vertical'=a to b] {
	i1 [particle=\(e^-\)] -- [anti fermion, rmomentum=\(p^\mu\)] a -- [anti fermion, rmomentum=\(p'^\mu\)] f1 [particle=\(e^-\)],
	a -- [photon, edge label=\(\gamma\)] b,
	i2 [particle=\(\mu^-\)] -- [fermion, momentum=\(k^\mu\)] b -- [fermion, momentum=\(k'^\mu\)] f2 [particle=\(\mu^-\)],
	};

	\caption{Diagrama de Feynman para \(e^- \mu^- \to e^- \mu^-\).}
\end{figure}
Nuestro elemento de matriz
\begin{equation}
	i \Mcal =e^2 \underbrace{\pqty{\bar{u}^{s'}(p') Q_e  \gamma^\mu u^{s}(p)}}_{j_e^\mu}  \frac{-ig_{\mu \nu}}{q^2}  \underbrace{\pqty{\bar{u}^{r'}(k') Q_\mu e \gamma^\nu u^{r}(k)}}_{{j_\mu^\nu} }
\end{equation}
tal que
\begin{equation}
	i \Mcal =\frac{e^2}{t} (j_e \cdot j_\mu)
\end{equation}
con $Q=-1$ en ambos casos y $q^2=t$ el invariange de Maddelstam ($t=|p'^\mu-p^\mu|^2$). Dado que no nos especifican los espines de entrada, podemos considerar que el elemento de matriz total es el no polarizado:
\begin{equation}
	\vqty{\Mcal_{fi}}^2 = \langle \vqty{\Mcal_{fi}}^2 \rangle = \frac{1}{4} \sum_{ss',r,r'} \vqty{\Mcal_{s,r}}
\end{equation}
O alternativamente con \textit{las amplitudes de helicidad}:
\begin{equation}
	\vqty{\Mcal_{fi}}^2 = \frac{1}{4} \pqty{\vqty{\Mcal_{\text{RR}}}^2+\vqty{\Mcal_{\text{RL}}}^2
		+\vqty{\Mcal_{\text{LR}}}^2+\vqty{\Mcal_{\text{LL}}}^2}
\end{equation}
Solo tenemos que considerar los siguientes espinores (al no haber antipartículas con consideramos $v$ y $\bar{v}$) con esta forma ($m_e\approx m_\mu\approx 0$), aunque hay que tener en cuenta que al tener momentso $p_z$ y $-p_z$ cambian algunos signos:
\begin{equation}
	u_\uparrow = \sqrt{E} \begin{pmatrix}
		c \\ se^{i\phi} \\ c \\ se^{i\phi}
	\end{pmatrix} \qquad
	u_\downarrow= \sqrt{E} \begin{pmatrix}
		-s \\ ce^{i\phi} \\ s \\ -ce^{i\phi}
	\end{pmatrix}
\end{equation}
con $c=\cos (\theta/2)$ y $s=\sin(\theta/2)$, siendo este $\theta$ el ángulo de salida de las partículas. Los momentos de nuestras partículas (caso de masa despreciable $m_e\approx m_\mu\approx 0$)
\begin{equation}
    p^\mu= (E,0,0,E) \qquad 
    p^{\prime\mu}= (E,E\sin \theta,0,E \cos \theta) \qquad 
\end{equation}
\begin{equation}
    k^\mu= (E,0,0,-E) \qquad 
    k^{\prime\mu}= (E,-E\sin \theta,0,-E \cos \theta)
\end{equation}
Son así poruqePor lo que, por ejemplo, para $p$:
\begin{equation}
	u_\uparrow(p) = \sqrt{E} \begin{pmatrix}
		1 \\ 0 \\ 1 \\ 0
	\end{pmatrix} \quad
	u_\downarrow(p)= \sqrt{E} \begin{pmatrix}
		0 \\ 1 \\ 0 \\ -1
	\end{pmatrix}\quad
	u_\uparrow(p') = \sqrt{E} \begin{pmatrix}
		c \\ s \\ c \\ s
	\end{pmatrix} \quad
	u_\downarrow(p')= \sqrt{E} \begin{pmatrix}
		-s \\ c \\ s \\ -c
	\end{pmatrix}
\end{equation}
\begin{equation}
	u_\uparrow(k) = \sqrt{E} \begin{pmatrix}
		1 \\ 0 \\ -1 \\ 0
	\end{pmatrix} \quad
	u_\downarrow(k)= \sqrt{E} \begin{pmatrix}
		0 \\ 1 \\ 0 \\ 1
	\end{pmatrix}  \quad
	u_\uparrow(k') = \sqrt{E} \begin{pmatrix}
		c \\ s \\ -c \\ -s
	\end{pmatrix} \quad
	u_\downarrow(k')= \sqrt{E} \begin{pmatrix}
		-s \\ c \\ -s \\ c
	\end{pmatrix}
\end{equation}
para $k'$ y $p'$ usaremos la expresión general. Usando las relaciones del Thomson \cite{thomson_modern_physics}
\begin{align}
	\overline{\psi}\gamma^0\phi & = \psi^\dagger \gamma^0 \gamma^0 \phi
	= \psi_1^*\phi_1 + \psi_2^*\phi_2 + \psi_3^*\phi_3 + \psi_4^*\phi_4,     \\
	\overline{\psi}\gamma^1\phi & = \psi^\dagger \gamma^0 \gamma^1 \phi
	= \psi_1^*\phi_4 + \psi_2^*\phi_3 + \psi_3^*\phi_2 + \psi_4^*\phi_1,     \\
	\overline{\psi}\gamma^2\phi & = \psi^\dagger \gamma^0 \gamma^2 \phi
	= -i(\psi_1^*\phi_4 - \psi_2^*\phi_3 + \psi_3^*\phi_2 - \psi_4^*\phi_1), \\
	\overline{\psi}\gamma^3\phi & = \psi^\dagger \gamma^0 \gamma^3 \phi
	= \psi_1^*\phi_3 - \psi_2^*\phi_4 + \psi_3^*\phi_1 - \psi_4^*\phi_2.
\end{align}
Tenemos que:


\begin{minipage}[t]{0.48\textwidth}
\begin{align*}
    \bar{u}_\uparrow (p') \gamma_0 {u}_\uparrow (p) & = 2Ec \\
    \bar{u}_\uparrow (p') \gamma_1 {u}_\uparrow (p) & = 2Es \\
    \bar{u}_\uparrow (p') \gamma_2 {u}_\uparrow (p) & = -i2Es \\
    \bar{u}_\uparrow (p') \gamma_3 {u}_\uparrow (p) & = 2Ec
\end{align*}
\begin{align*}
    \bar{u}_\uparrow(p') \gamma_0 {u}_\downarrow (p) & = 0 \\
    \bar{u}_\uparrow(p') \gamma_1 {u}_\downarrow (p) & = 0 \\
    \bar{u}_\uparrow(p') \gamma_2 {u}_\downarrow (p) & = 0 \\
    \bar{u}_\uparrow(p') \gamma_3 {u}_\downarrow (p) & = 0
\end{align*}
\end{minipage}\hfill
\begin{minipage}[t]{0.48\textwidth}
\begin{align*}
    \bar{u}_\downarrow (p') \gamma_0 {u}_\uparrow (p) & = 0 \\
    \bar{u}_\downarrow (p') \gamma_1 {u}_\uparrow (p) & = 0 \\
    \bar{u}_\downarrow (p') \gamma_2 {u}_\uparrow (p) & = 0 \\
    \bar{u}_\downarrow (p') \gamma_3 {u}_\uparrow (p) & = 0
\end{align*}
\begin{align*}
    \bar{u}_\downarrow(p') \gamma_0 {u}_\downarrow (p) & = 2Ec \\
    \bar{u}_\downarrow(p') \gamma_1 {u}_\downarrow (p) & = 2Es \\
    \bar{u}_\downarrow(p') \gamma_2 {u}_\downarrow (p) & = -i2Es \\
    \bar{u}_\downarrow(p') \gamma_3 {u}_\downarrow (p) & = 2Ec
\end{align*}
\end{minipage}
%%%%%%%%%%%%%%%%%%%%%%%%%%%%%%%%%%%%%%%%%%%%%%%5
\begin{minipage}[t]{0.48\textwidth}
\begin{align*}
    \bar{u}_\uparrow (k') \gamma_0 {u}_\uparrow (k) & = 2Ec \\
    \bar{u}_\uparrow (k') \gamma_1 {u}_\uparrow (k) & = -2Es \\
    \bar{u}_\uparrow (k') \gamma_2 {u}_\uparrow (k) & = i2Es \\
    \bar{u}_\uparrow (k') \gamma_3 {u}_\uparrow (k) & = -2Ec
\end{align*}
\begin{align*}
    \bar{u}_\uparrow(k') \gamma_0 {u}_\downarrow (k) & = 0 \\
    \bar{u}_\uparrow(k') \gamma_1 {u}_\downarrow (k) & = 0 \\
    \bar{u}_\uparrow(k') \gamma_2 {u}_\downarrow (k) & = 0 \\
    \bar{u}_\uparrow(k') \gamma_3 {u}_\downarrow (k) & = 0
\end{align*}
\end{minipage}\hfill
\begin{minipage}[t]{0.48\textwidth}
\begin{align*}
    \bar{u}_\downarrow (k') \gamma_0 {u}_\uparrow (k) & = 0 \\
    \bar{u}_\downarrow (k') \gamma_1 {u}_\uparrow (k) & = 0 \\
    \bar{u}_\downarrow (k') \gamma_2 {u}_\uparrow (k) & = 0 \\
    \bar{u}_\downarrow (k') \gamma_3 {u}_\uparrow (k) & = 0
\end{align*}
\begin{align*}
    \bar{u}_\downarrow(k') \gamma_0 {u}_\downarrow (k) & = 2Ec \\
    \bar{u}_\downarrow(k') \gamma_1 {u}_\downarrow (k) & = -2Es \\
    \bar{u}_\downarrow(k') \gamma_2 {u}_\downarrow (k) & = -i2Es \\
    \bar{u}_\downarrow(k') \gamma_3 {u}_\downarrow (k) & = -2Ec
\end{align*}
\end{minipage}
Es decir, \textit{cada vértice debe conservar la helicidad} (que básicamente implica conservar espín). Ahora tenemos que calcular cada término. Lo que está claro que es que, si en un vértice no se conseva helicidad la amplitud de probabilidad es nula. Así pues: 

\begin{equation}
    \vqty{\Mcal_{\text{LL}}}^2 = \vqty{\Mcal_{\text{LL}\to\text{LL}}}^2 = \frac{e^4}{t^2}\vqty{4E^2c^2+4E^2c^2+4E^2s^2+4E^2s^2} = \frac{e^4}{t^2}  64 E^4 = \frac{e^4}{t^2}  4s^2
\end{equation}
\begin{equation}
    \vqty{\Mcal_{\text{RR}}}^2 = \vqty{\Mcal_{\text{RR}\to\text{RR}}}^2 = \frac{e^4}{t^2}\vqty{4E^2c^2+4E^2c^2} = \frac{e^4}{t^2}  64 c^4 E^4 = 16 E^4 (1+\cos \theta)^2 =  \frac{e^4}{t^2} 4 u^2
\end{equation}
\begin{equation}
    \vqty{\Mcal_{\text{RL}}}^2 = \vqty{\Mcal_{\text{RL\to\text{RL}}}}^2 = \frac{e^4}{t^2}\vqty{4E^2c^2+4E^2c^2+4E^2s^2+4E^2s^2} = \frac{e^4}{t^2}  64 E^4 = \frac{e^4}{t^2}  4s^2
\end{equation}
\begin{equation}
    \vqty{\Mcal_{\text{LR}}}^2 = \vqty{\Mcal_{\text{LR}\to\text{LR}}}^2 = \frac{e^4}{t^2}\vqty{4E^2c^2+4E^2c^2} = \frac{e^4}{t^2}  64 c^4 E^4 = 16 E^4 (1+\cos \theta)^2 =  \frac{e^4}{t^2} 4 u^2
\end{equation}
donde hemos usado que $s=(p+k)^2=4E^2$ y $u=(p-k')^2=-2(p\cdot k')=-2E(1+\cos \theta)$. Luego el elemento de matriz total cuando $m_e \approx m_\mu \approx 0$ es: 

\begin{equation}
    \vqty{\Mcal_{ij}} = \frac{2e^4}{t^2} (s^2+u^2)
\end{equation}
que es lo que queríamos demostrar. De hecho, podemos hacer el cálculo con trazas para comprobar si da lo mismo. Así pues, tenemos que:

\begin{equation}
	\vqty{\Mcal_{s,r}}^2 = e^4 \bqty{\pqty{\bar{u}^{s'}(p') \gamma^\mu u^{s}(p)}\pqty{\bar{u}^{r'}(k') \gamma_\mu u^{r}(k)}} \bqty{\pqty{\bar{u}^{s'}(p') \gamma^\nu u^{s}(p)}\pqty{\bar{u}^{r'}(k')\gamma_\nu u^{r}(k)}}^*
\end{equation}
Tal que usando el truco de Casimir (Williams \cite{Williams_2022})

\[
	\sum_{s_1,s_2}
	\bigl[\bar{u}_1^{s_1}(p_1)\,\Gamma\,u_2^{s_2}(p_2)\bigr]
	\bigl[\bar{u}_1^{s_1}(p_1)\,\Gamma' u_2^{s_2}(p_2)\bigr]^*
	= \mathrm{tr}\!\left[ \Gamma \,(\slashed{p}_2+m_2)\,\bar{\Gamma}'\,(\slashed{p}_1+m_1) \right].
\]
tenemos que

\begin{equation}
	\vqty{\Mcal_{s,r}}^2 =\frac{2e^4}{t^2} \tr \pqty{\gamma^\mu (\pslash+m_e)\gamma^\nu (\pslash+m_e)}\tr \pqty{\gamma^\mu (\kslash+m_\mu)\gamma^\nu (\kslash+m_\mu)}
\end{equation}
tal que:

\begin{itemize}
	\item La primera traza:
	      \begin{equation}
		      \tr \pqty{\gamma^\mu (\pslash+m_e)\gamma^\nu (\pslash'+m_e)} = \tr (\gamma^\mu \pslash \gamma^\nu \pslash') + 4g^{\mu \nu}m_e^2
	      \end{equation}
	      Ahora la traza más compleja
	      \begin{align}
		      \tr (\gamma^\mu \pslash \gamma^\nu \pslash')
		       & = 4 g^{\mu \sigma} p_\sigma g^{\nu \rho} p_{\rho}^{\prime}
		      + 4 g^{\mu \rho} p_{\rho}^{\prime } g^{\nu \sigma} p_{\sigma}
		      - 4 p_{\mu \nu} (p \cdot p')                                                    \\
		       & = 4 p^\mu p^{\prime \nu} + 4 p^\nu p^{\prime\mu} - 4 g^{\mu \nu} (p\cdot p')
	      \end{align}
	\item La segunda traza:
	      \begin{equation}
		      \tr \pqty{\gamma_\mu (\kslash+m_\mu)\gamma_\nu (\kslash'+m_\mu)} = \tr (\gamma_\mu \kslash \gamma_\nu \kslash') + 4g_{\mu \nu}m_\mu^2
	      \end{equation}
	      Ahora la traza más compleja
	      \begin{align}
		      \tr (\gamma_\mu \kslash \gamma_\nu \kslash')
		       & = 4 g_{\mu \sigma} k^\sigma g_{\nu \rho} k^{\prime \rho}
		      + 4 g_{\mu \rho} k^{\prime \rho} g_{\nu \sigma} k^{\sigma}
		      - 4 g_{\mu \nu} (k \cdot k')                                     \\
		       & = 4 k_\mu k'_\nu + 4 k_\nu k'_\mu - 4 g_{\mu \nu} (k\cdot k')
	      \end{align}
\end{itemize}
Directamente  multiplicando:
\begin{equation}
	\vqty{\Mcal_{s,r}}^2 = 32 (p\cdot k)(p'\cdot k') + 32 (p \cdot k')(p'\cdot k)- 32(p\cdot p')m_\mu^2 - 32(k\cdot k')m_e^2 + 64 m_e^2 m_\mu^2
\end{equation}
Ahora solo tenemos que agrupar y dividir entre $1/4t$.
\begin{equation}
	\vqty{\Mcal_{fi}}^2 = \frac{8}{t} \pqty{(p\cdot k)(p'\cdot k') +  (p \cdot k')(p'\cdot k)- (p\cdot p')m_\mu^2 - (k\cdot k')m_e^2 + 2 m_e^2 m_\mu^2}
\end{equation}
Usando los invariantes de Maddelstam:
\begin{equation}
	s = (p+k)^2 = (p'+k')^2 \qquad t = (p-p')^2 = (k-k')^2 \qquad u = (p-k')^2 = (p'-k)^2
\end{equation}
tal que:
\begin{equation}
	s = m_e^2 + m_\mu^2 +2 (p \cdot k) = m_e^2 + m_\mu^2 + 2 (p' \cdot k')
\end{equation}
\begin{equation}
	t = 2m_e^2 + m_\mu^2 -2 (p \cdot p') = 2m_\mu^2 - 2 (k \cdot k')
\end{equation}
\begin{equation}
	u = m_e^2 + m_\mu^2 -2 (p \cdot k') = m_e^2 + m_\mu^2 - 2 (p'\cdot k)
\end{equation}
Con la relación:
\begin{equation}
	s+t+u = 2(m_e^2 + m_\mu^2)
\end{equation}
Ahora:
\begin{align}
	\vqty{\Mcal_{fi}}^2 & = \frac{2e^4}{t^2} \pqty{(s-m_\mu^2-m_e^2)^2 + (u-m_e^2-m_\mu^2)^2  +  2 t(m_e^2 + m_\mu^2)}        \\
	    & = \frac{2e^4}{t^2} \pqty{s^2 + u^2 - 2(s+\mu)(m_\mu^2 +m_e^2) +  2 t(m_e^2 + m_\mu^2) + 2(m_\mu^2 +m_e^2)^2}
\end{align}
que aplicando $ s+u = 2(m_e^2 + m_\mu^2) -t$ tenemos:
\begin{align}
	\vqty{\Mcal_{fi}}^2 & = \frac{2e^4}{t^2} (s^2 + u^2 + 4t(m_\mu^2 + m_e^4) - 2(m_\mu^2+m_e^2)^2)
\end{align}
que efectivamente si $m_e \approx m_\mu \approx 0$ tenemos

\begin{align}
	\vqty{\Mcal_{fi}}^2 & = \frac{2e^4}{t^2} (s^2 + u^2 )
\end{align}
que es lo mismo que obtuvimos antes \cite{Williams_2022}. Lo que implica que, la sección eficaz diferencial es, en este caso:

\begin{equation}
    \dv{\sigma}{\Omega} = \frac{1}{64\pi^2 s} \frac{e^4 (s^2 + u^2)}{t^2}
\end{equation}
que si queremos expresar con $E$ y $\cos \theta$ explícitamente, tenemos que $t=-2(p_1\cdot p_3)=-2E(1-\cos \theta)$

\begin{equation}
    \dv{\sigma}{\Omega} = \frac{16}{64\pi^2 s} \frac{ e^4 E^2 \pqty{+\frac{1}{4}(1+\cos \theta)}}{4E^2(1-\cos \theta)^2}
\end{equation}
y la expresión final, usando la constante de estructura fina $2\alpha^2 = 2(e^4/16\pi^2)$ tenemos: 
\begin{equation}
    \dv{\sigma}{\Omega} = \frac{2\alpha^2}{s} \frac{{+\frac{1}{4}(1+\cos \theta)}}{(1-\cos \theta)^2}
\end{equation}
que coincide con la del Thomson \cite{thomson_modern_physics}.

%%%%%%%%%%%%%%%%%%%%%%%%%%%%%%%%%%%%%%%%%%%%%%%%%%
%%%%%%%%%%%%%%%%%%%%%%%%%%%%%%%%%%%%%%%%%%%%%%%%%%
%%%%%%%%%%%%%%%%%%%%%%%%%%%%%%%%%%%%%%%%%%%%%%%%%%
%%%%%%%%%%%%%%%%%%%%%%%%%%%%%%%%%%%%%%%%%%%%%%%%%%


\begin{Ejercicio}{Elemento de matriz en QED con trazas}\label{Ej:07}
	Use el formalismo de trazas para calcular el elemento de matriz al cuadrado, promediado en espines, de la electrodinámica cuántica (QED) para
	\[
		e^+ e^- \;\to\; f \bar{f},
	\]
	sin despreciar ninguna de las masas de las partículas.
\end{Ejercicio}

\begin{figure}[h]
	\centering
	\feynmandiagram [horizontal'=a to b] {
	i1 [particle=\(e^-\)] -- [fermion, momentum=\(p^\mu\)] a -- [fermion,rmomentum=\(k^\mu\)] i2 [particle=\(e^+\)],
	a -- [photon, edge label=\(\gamma\)] b,
	f1 [particle=\( f^+\)] -- [fermion,rmomentum=\(k'^\mu\)] b -- [fermion,momentum=\(p'^\mu\)] f2 [particle=\(f^-\)],
	};
	\caption{Diagrama de Feynman para \(e^+ e^- \to f^+ f^-\).}
\end{figure}
Nuestro elemento de matriz
con $Q=-1$ en ambos casos y $q^2=s$ el invariange de Maddelstam ($s=|k^\mu+p^\mu|^2$). Dado que no nos especifican los espines de entrada, podemos considerar que el elemento de matriz total es el no polarizado:
\begin{equation}
	\vqty{\Mcal_{fi}}^2 = \langle \vqty{\Mcal_{fi}}^2 \rangle = \frac{1}{4} \sum_{s,s',r,r'} \vqty{\Mcal_{s,r}}
\end{equation}
Así pues, tenemos que:

\begin{equation}
	\vqty{\Mcal_{s,r}}^2 = e^4 \bqty{\pqty{\bar{v}^{s'}(k) \gamma^\mu u^{s}(p)}\pqty{\bar{u}^{r'}(p') \gamma_\mu v^{r}(k')}} \bqty{\pqty{\bar{v}^{s'}(k) \gamma^\nu u^{s}(p)}\pqty{\bar{u}^{r'}(p')\gamma_\nu v^{r}(k')}}^*
\end{equation}
Tal que usando el truco de Casimir (Williams \cite{Williams_2022})

\[
	\sum_{s_1,s_2}
	\bigl[\bar{u}_1^{s_1}(p_1)\,\Gamma\,u_2^{s_2}(p_2)\bigr]
	\bigl[\bar{u}_1^{s_1}(p_1)\,\Gamma' u_2^{s_2}(p_2)\bigr]^*
	= \mathrm{tr}\!\left[ \Gamma \,(\slashed{p}_2+m_2)\,\bar{\Gamma}'\,(\slashed{p}_1+m_1) \right].
\]
tenemos que
\begin{equation}
	\vqty{\Mcal_{s,r}}^2 =\frac{2e^4}{t^2} \tr \pqty{\gamma^\mu (\pslash+m_e)\gamma^\nu (\kslash-m_e)}\tr \pqty{\gamma^\mu (\pslash'+m_\mu)\gamma^\nu (\kslash'-m_\mu)}
\end{equation}
tal que:

\begin{itemize}
	\item La primera traza:
	      \begin{equation}
		      \tr \pqty{\gamma^\mu (\pslash+m_e)\gamma^\nu (\kslash-m_e)} = \tr (\gamma^\mu \pslash \gamma^\nu \kslash) - 4g^{\mu \nu}m_e^2
	      \end{equation}
	      Ahora la traza más compleja
	      \begin{align}
		      \tr (\gamma^\mu \pslash \gamma^\nu \kslash)
		       & = 4 g^{\mu \sigma} p_\sigma g^{\nu \rho} k_{\rho}
		      + 4 g^{\mu \rho} k_{\rho} g^{\nu \sigma} p_{\sigma}
		      - 4 p_{\mu \nu} (p \cdot k)                                                    \\
		       & = 4 p^\mu p^{\prime \nu} + 4 p^\nu k^{\mu} - 4 g^{\mu \nu} (p\cdot k)
	      \end{align}
	\item La segunda traza:
	      \begin{equation}
		      \tr \pqty{\gamma_\mu (\pslash'+m_f)\gamma_\nu (\kslash'-m_f)} = \tr (\gamma_\mu \pslash' \gamma_\nu \kslash') - 4g_{\mu \nu}m_f^2
	      \end{equation}
	      Ahora la traza más compleja
	      \begin{align}
		      \tr (\gamma_\mu \pslash' \gamma_\nu \kslash')
		       & = 4 g_{\mu \sigma} p^{\prime\sigma} g_{\nu \rho} k^{\prime \rho}
		      + 4 g_{\mu \rho} k^{\prime \rho} g_{\nu \sigma} p^{\prime \sigma}
		      - 4 g_{\mu \nu} (p' \cdot k')                                     \\
		       & = 4 p'_\mu k'_\nu + 4 p'_\nu k'_\mu - 4 g_{\mu \nu} (p'\cdot k')
	      \end{align}
\end{itemize}
Directamente  multiplicando:
\begin{equation}
	\frac{s^2}{e^4}\vqty{\Mcal_{s,r}}^2 = 32 (p\cdot p')(k\cdot k') + 32 (p \cdot k')(k\cdot p') + 32(p\cdot k)m_f^2 + 32(p'\cdot k')m_e^2 + 64 m_e^2 m_f^2
\end{equation}
Ahora solo tenemos que agrupar y dividir entre $1/4t$.
\begin{equation}
	\vqty{\Mcal_{fi}}^2 = \frac{8e^4}{s^2} \pqty{(p\cdot k)(p'\cdot k') +  (p \cdot k')(p'\cdot k)- (p\cdot p')m_f^2 - (k\cdot k')m_e^2 + 2 m_e^2 m_f^2}
\end{equation}
Usando los invariantes de Maddelstam:
\begin{equation}
	s = (p+k)^2 = (p'+k')^2 \qquad t = (p-p')^2 = (k-k')^2 \qquad u = (p-k')^2 = (p'-k)^2
\end{equation}
tal que:
\begin{equation}
	s = m_e^2 + m_f^2 +2 (p \cdot k) = m_e^2 + m_f^2 + 2 (p' \cdot k')
\end{equation}
\begin{equation}
	t = 2m_e^2 + m_f^2 -2 (p \cdot p') = 2m_f^2 - 2 (k \cdot k')
\end{equation}
\begin{equation}
	u = m_e^2 + m_f^2 -2 (p \cdot k') = m_e^2 + m_f^2 - 2 (p'\cdot k)
\end{equation}
Con la relación:
\begin{equation}
	s+t+u = 2(m_e^2 + m_f^2)
\end{equation}
Ahora:
\begin{align}
	\vqty{\Mcal_{fi}}^2 & = \frac{2e^4}{s^2} \pqty{(t-m_f^2-m_e^2)^2 + (u-m_e^2-m_f^2)^2  +  2 s(m_e^2 + m_f^2)}        \\
	& = \frac{2e^4}{s^2} \pqty{t^2 + u^2 - 2(-s+2(m_e^2 + m_f^2))(m_f^2 +m_e^2) +  2 s(m_e^2 + m_f^2) + 2(m_f^2 +m_e^2)^2}
\end{align}
que aplicando $ t+u = 2(m_e^2 + m_f^2) -s$ tenemos:
\begin{align}
	\vqty{\Mcal_{fi}}^2 & = \frac{2e^4}{s^2} (t^2 + u^2 + 4s(m_f^2 + m_e^4) - 2(m_f^2+m_e^2)^2)
\end{align}
que es el resultado final \cite{Williams_2022}, y que como podemos ver es igual a la dispersión $e^- f^- \to e^- f^-$ pero haciendo el cambio $s \leftrightarrow t$. 

%%%%%%%%%%%%%%%%%%%%%%%%%%%%%%%%%%%%%%%%%%%%%%%%%%
%%%%%%%%%%%%%%%%%%%%%%%%%%%%%%%%%%%%%%%%%%%%%%%%%%
%%%%%%%%%%%%%%%%%%%%%%%%%%%%%%%%%%%%%%%%s%%%%%%%%%%
%%%%%%%%%%%%%%%%%%%%%%%%%%%%%%%%%%%%%%%%%%%%%%%%%%


\begin{Ejercicio}{Dispersión $e^-p$ y transferencia de momento}\label{Ej:14}
	En un experimento de dispersión $e^-p$, la energía del electrón incidente es $E_1 = 529.5\,\mathrm{MeV}$ y los electrones dispersados se detectan a un ángulo $\theta = 75^\circ$ con respecto al haz incidente.
	A este ángulo, casi todos los electrones dispersados se miden con una energía de $E_3 \approx 373\,\mathrm{MeV}$.

	\medskip
	¿Qué se puede concluir a partir de esta observación?
	Encuentre el valor correspondiente de
	\[
		Q^2 = -q^2.
	\]
\end{Ejercicio}

En este ejercicio tenemos que usar los cuadrimomentos para ver la energía y momento del protón para el ángulo con el que salen los electrones. 

%%%%%%%%%%%%%%%%%%%%%%%%%%%%%%%%%%%%%%%%%%%%%%%%%%
%%%%%%%%%%%%%%%%%%%%%%%%%%%%%%%%%%%%%%%%%%%%%%%%%%
%%%%%%%%%%%%%%%%%%%%%%%%%%%%%%%%%%%%%%%%%%%%%%%%%%
%%%%%%%%%%%%%%%%%%%%%%%%%%%%%%%%%%%%%%%%%%%%%%%%%%


\begin{Ejercicio}{Tasa de decaimiento del pión en interacción escalar}\label{Ej:15}
	Calcule la tasa de decaimiento del pión para una interacción puramente escalar y muestre que la razón predicha de tasas de decaimiento es
	\[
		\frac{\Gamma(\pi^- \to e^- \bar{\nu}_e)}{\Gamma(\pi^- \to \mu^- \bar{\nu}_\mu)} \;\approx\; 5.5.
	\]

	\textit{Sugerencia:} suponga que la interacción está mediada por una partícula escalar masiva $X$, de modo que la cantidad escalar asociada al vértice $\ell-\nu_\ell$ con constante de acoplo $g_X$ sería
	\[
		j_\ell = g_X \, \bar{u} v.
	\]
	Puede trabajar también en el régimen $q^2 = m_\pi^2 \ll m_X^2$ y tomar el propagador como $1/m_X^2$ (interacción tipo Fermi puntual).
\end{Ejercicio}

Tenemos que la tasa de decaimiento viene dada por la expresión: 

\begin{equation}
    \Gamma = \frac{p^* }{8\pi m_a^2} |\Mcal_{fi}|^2
\end{equation}
siendo $p^*$ el momento en el centro de masas de las partículas salientes ($p^*=|\pn|=|\kn|$), con los siguientes momentos:
\begin{equation}
    p_\pi^\mu = (m_\pi,0,0,0) \qquad p^\mu =(E_1,0,0,p_1) \qquad  k^\mu =(E_2,0,0,-p_1) 
\end{equation}
Así pues, solo tenemos que calcular el elemento de matriz $\Mcal_{fi}$, que se calcula a través del siguiente diagrama:


\begin{figure}[h]
	\centering
	\feynmandiagram [horizontal'=a to b] {
		i1 [particle=\(\pi^-\)] -- [fermion] a
			-- [boson, edge label=\(X\)] b,
		b -- [fermion, momentum'=\(p\)] f1 [particle=\(\ell^-\)],
		b -- [anti fermion, momentum=\(k\)] f2 [particle=\(\bar{\nu}_\ell\)],
	};
	\caption{Diagrama de Feynman para \(\pi^- \to \ell^- \bar{\nu}_\ell\).}
\end{figure}
Al hacer un acoplo puramente escalar tenemos que: 
\begin{equation}
    \Mcal_{fi} = f_\pi p_\pi \frac{g_X}{M_X^2} \bar{u}(p) v(k)
\end{equation}
Luego, la única dificultad para calcular $\vqty{\Mcal_{ij}}^2$ es hacer: 
\begin{equation}
    \sum_{\text{espines}} \pqty{\bar{u}(p) v(k)}  \pqty{\bar{u}(p) v(k)}^\dagger = \tr \pqty{(\pslash+m_l)(\kslash-m_\nu)} = 4 p^\mu k_\mu - 4 m_l m_\nu 
\end{equation}
que es, según los cálculos anteriores, y suponiendo que $m_\nu \to 0$: 
\begin{equation}
    |\Mcal_{fi}|^2 = \bqty{f_\pi m_\pi \frac{g_X}{M_X^2} 2 (p \cdot k)}^2
\end{equation}
tal que 

\begin{equation}
    p_\pi^\mu = p^\mu + k^\mu \to m_\pi^2 = m_l^2 + 2 (p \cdot k) \to (p\cdot k) = \frac{1}{2} (m_\pi^2 - m_l^2)
\end{equation}
y por tanto: 

\begin{equation}
    \Gamma = \frac{p^* }{8\pi m_\pi^2} \bqty{f_\pi m_\pi \frac{g_X}{M_X^2} (m_\pi^2 - m_l^2)}^2
\end{equation}
tal que: 

\begin{equation}
    \Gamma = \frac{f_\pi^2 g_X^2}{8 \pi m_X^2} (m_\pi^2 - m_l^2)^2
\end{equation}
De lo que se deduce que 

\begin{equation}
   \frac{\Gamma(\pi^- \to e^- \bar{\nu}_e)}{\Gamma(\pi^- \to \mu^- \bar{\nu}_\mu)} = \frac{(m_\pi^2 - m_e^2)^2}{(m_\pi^2 - m_\mu^2)^2} = 5.5
\end{equation}


%%%%%%%%%%%%%%%%%%%%%%%%%%%%%%%%%%%%%%%%%%%%%%%%%%
%%%%%%%%%%%%%%%%%%%%%%%%%%%%%%%%%%%%%%%%%%%%%%%%%%
%%%%%%%%%%%%%%%%%%%%%%%%%%%%%%%%%%%%%%%%%%%%%%%%%%
%%%%%%%%%%%%%%%%%%%%%%%%%%%%%%%%%%%%%%%%%%%%%%%%%%

\begin{Ejercicio}{Vértices gauge en la parte cinética del lagrangiano electrodébil}\label{Ej:16}
Muestre explícitamente que la parte cinética del lagrangiano electrodébil contiene dos vértices triple–gauge
\[
\gamma W^+W^-,\qquad ZW^+W^-,
\]
y cuatro vértices cuárticos gauge
\[
W^+W^-W^+W^-,\qquad W^+W^-ZZ,\qquad W^+W^-\gamma\gamma,\qquad W^+W^-Z\gamma.
\]
\end{Ejercicio}

La parte cinética del lagrangiano es: 

\begin{equation}
	\Lcal_{\text{kin}} = B^{\mu \nu} B_{\mu \nu} + W^{\mu \nu i} W_{\mu \nu}^i
\end{equation}
Queremos llegar a obtener las relaciones entre $W_{\mu}$, $W_{\mu \dagger}$, $Z_\mu$ y $A_\mu$. Tenemos que usar la que relaciona los elementos 1 y 2  de $W^i$ con los bosones $W$  y $W^\dagger$: 

\begin{equation}
	W_{\mu} = \frac{1}{\sqrt{2}} \pqty{W_{\mu}^1+iW_{\mu}^2} \qquad 
	W_{\mu}^\dagger = \frac{1}{\sqrt{2}} \pqty{W_{\mu}^1-iW_{\mu}^2}
\end{equation}
\begin{equation}
	W^1_{\mu} = \frac{1}{\sqrt{2}} (W_\mu + W_\mu^\dagger) \qquad 
	W^2_{\mu} = \frac{-i}{\sqrt{2}} (W_\mu + - W_\mu^\dagger) 
\end{equation}
y la que relciona $B^\mu$ y $W^3$ con $Z$ y $A$:

\begin{equation}
\begin{pmatrix}
W_\mu^{3} \\[4pt]
B_\mu
\end{pmatrix}
=
\begin{pmatrix}
\cos\theta_W & \sin\theta_W \\[4pt]
-\sin\theta_W & \cos\theta_W
\end{pmatrix}
\begin{pmatrix}
Z_\mu \\[4pt]
A_\mu
\end{pmatrix}.
\end{equation}
\begin{equation}
	W^3_{\mu} =  \cos \theta_W W_\mu +\sin \theta_W W_\mu^\dagger \qquad 
	B^\mu_{\mu} = -\sin \theta_W W_\mu +\cos \theta_W W_\mu^\dagger
\end{equation}
Lo siguiente que tenemos que hacer es aplicar lo siguiente a nuestro $\Lcal_{\text{kin}}$:

\begin{equation}
\widetilde{W}_{\mu\nu} \equiv \frac{\sigma^i}{2} W^i_{\mu\nu}, 
\qquad 
W^i_{\mu\nu} = \partial_\mu W^i_\nu - \partial_\nu W^i_\mu - g_W \epsilon^{ijk} W^j_\mu W^k_\nu
\end{equation}
\begin{equation}
B_{\mu\nu} \equiv \partial_\mu B_\nu - \partial_\nu B_\mu
\end{equation}
Una vez tenemos eto, solo es sutituir. Lo que está claro qes que el término $B^{\mu \nu} B_{\mu \nu} $ no va a contribuir a nuestra triple o cuádruple interacción, ya que es evidente que solo contendrá términos de $\gamma$ y $Z$. Por otro lado, solo nos intersan los términos que incluyan al menos un $g_W \epsilon^{ijk} W_{\mu}^j W_{\mu}^k$, ya que otros términos no contendrán términos triples y cuádruples. Cuando este esté relacionado con otro $g_W \epsilon^{ijk} W_{\mu}^j W_{\mu}^k$ tendremos los vértices cuádruples, y cuando esté relacionado con $\partial_\mu W^i_\nu-\partial_\nu W_\mu^i$ los vértices triples. Así: 

\begin{equation}
	W^{\mu \nu i} W_{\mu \nu}^i = 
	W^{\mu \nu 1} W_{\mu \nu}^1 +
	W^{\mu \nu 2} W_{\mu \nu}^2 +
	W^{\mu \nu 3} W_{\mu \nu}^3
\end{equation}
Veamos término a término: 

\begin{itemize}
	\item El primer término: 
	\begin{align}  
		W^{\mu \nu 1} W_{\mu \nu}^1 = & \bqty{\pqty{\partial_\mu W^{\nu 1} - \partial_\nu W^{\mu 1}} - g_W \epsilon^{1ij} W^{\mu j} W^{\nu i}} \times \nonumber \\
		& \bqty{\pqty{\partial_\mu W^1_\nu - \partial_\nu W^1_\mu} - g_W 	\epsilon^{1kl} W^j_\mu W^k_\nu} 
	\end{align}
	Es evidente que el producto entre los dos primeros términos de cada corchete no generan un vértice triple o cuádruple, ya que $W^1$ solo va con los bosones $W$ y $W^\dagger$. Los vértices triples se producirán cuando los primers términos interaccionan con el término que va con la levi-civita se producen los vértices triples. Cuando interaccionan los dos términos de la levi-civita se producen los términos que representan los vértices cuárticos. Así pues, tenemos que: 

\end{itemize}



%%%%%%%%%%%%%%%%%%%%%%%%%%%%%%%%%%%%%%%%%%%%%%%%%%
%%%%%%%%%%%%%%%%%%%%%%%%%%%%%%%%%%%%%%%%%%%%%%%%%%
%%%%%%%%%%%%%%%%%%%%%%%%%%%%%%%%%%%%%%%%%%%%%%%%%%
%%%%%%%%%%%%%%%%%%%%%%%%%%%%%%%%%%%%%%%%%%%%%%%%%%

\begin{Ejercicio}{Expansión del potencial alrededor del vacío}\label{Ej:17}
Dados
\[
\mathcal{U}=\mu^2\phi^2+\lambda \phi^4,\qquad \mu^2=-\lambda v^2,\qquad 
\phi=\frac{1}{\sqrt{2}}\,(v+\eta+i\xi),
\]
verifique la ecuación (2.108), a saber:
\[
\mathcal{U}(\eta,\xi)= -\frac{1}{2}\lambda v^2\Big[(v+\eta)^2+\xi^2\Big]
+ \frac{1}{4}\lambda\Big[(v+\eta)^2+\xi^2\Big]^2
\]
\[
= -\frac{1}{4}\lambda v^4 + \lambda v^2\eta^2 + \lambda v\eta^3 + \frac{1}{4}\lambda\eta^4
+ \frac{1}{4}\lambda\xi^4 + \lambda v\eta\,\xi^2 + \frac{1}{2}\lambda \eta^2\xi^2
= \lambda v^2\eta^2 + \mathcal{U}_{\text{int}} - \frac{1}{4}\lambda v^4.
\]
\end{Ejercicio}

%%%%%%%%%%%%%%%%%%%%%%%%%%%%%%%%%%%%%%%%%%%%%%%%%%
%%%%%%%%%%%%%%%%%%%%%%%%%%%%%%%%%%%%%%%%%%%%%%%%%%
%%%%%%%%%%%%%%%%%%%%%%%%%%%%%%%%%%%%%%%%%%%%%%%%%%
%%%%%%%%%%%%%%%%%%%%%%%%%%%%%%%%%%%%%%%%%%%%%%%%%%

\begin{Ejercicio}{Invariancia gauge de un escalar complejo}\label{Ej:18}
Verifique que el lagrangiano
\[
\mathcal{L}=(D_\mu\phi)^\ast(D^\mu\phi)-\mathcal{U}(\phi^2),
\]
donde $\phi$ es un campo escalar complejo, es invariante bajo transformaciones de norma (gauge) del tipo
\[
\phi \;\to\; \phi' = e^{\,i g\,\alpha(x)}\,\phi,
\]
siempre que el campo gauge se transforme como
\[
B_\mu \;\to\; B'_\mu = B_\mu - \partial_\mu \alpha(x),
\]
con el derivado covariante $D_\mu=\partial_\mu + i g B_\mu$.
\end{Ejercicio}

%%%%%%%%%%%%%%%%%%%%%%%%%%%%%%%%%%%%%%%%%%%%%%%%%%
%%%%%%%%%%%%%%%%%%%%%%%%%%%%%%%%%%%%%%%%%%%%%%%%%%
%%%%%%%%%%%%%%%%%%%%%%%%%%%%%%%%%%%%%%%%%%%%%%%%%%
%%%%%%%%%%%%%%%%%%%%%%%%%%%%%%%%%%%%%%%%%%%%%%%%%%
\begin{Ejercicio}{Lagrangiano con campo escalar complejo}\label{Ej:19}
Considere el siguiente lagrangiano para un campo escalar complejo:
\[
\mathcal{L} = -\frac{1}{4}F_{\mu\nu}F^{\mu\nu}
+ (\partial_\mu \phi)^\ast (\partial^\mu \phi)
- \mu^2 \phi^2 - \lambda \phi^4
- i g B_\mu \phi^\ast (\partial^\mu \phi)
+ i g (\partial_\mu \phi^\ast) B^\mu \phi
+ g^2 B_\mu B^\mu \phi^\ast \phi,
\]
donde $B_\mu$ es un campo gauge y 
\[
F^{\mu\nu} = \partial^\mu B^\nu - \partial^\nu B^\mu.
\]

\medskip
Haga la sustitución 
\[
\phi(x) = \frac{1}{\sqrt{2}}\big(v + \eta(x) + i\xi(x)\big),
\]
y verifique que se obtiene:
\[
\mathcal{L} =
\frac{1}{2}(\partial_\mu \eta)(\partial^\mu \eta)
- \lambda v^2 \eta^2
+ \frac{1}{2}(\partial_\mu \xi)(\partial^\mu \xi)
- \frac{1}{4}F_{\mu\nu}F^{\mu\nu}
+ \frac{1}{2}g^2 v^2 B_\mu B^\mu
- \mathcal{U}_{\text{int}}
+ g v B_\mu (\partial^\mu \xi),
\]
junto con la expresión explícita para $\mathcal{U}_{\text{int}}$.
\end{Ejercicio}

%%%%%%%%%%%%%%%%%%%%%%%%%%%%%%%%%%%%%%%%%%%%%%%%%%
%%%%%%%%%%%%%%%%%%%%%%%%%%%%%%%%%%%%%%%%%%%%%%%%%%
%%%%%%%%%%%%%%%%%%%%%%%%%%%%%%%%%%%%%%%%%%%%%%%%%%
%%%%%%%%%%%%%%%%%%%%%%%%%%%%%%%%%%%%%%%%%%%%%%%%%%
\begin{Ejercicio}{Término del lagrangiano del Modelo Estándar con derivadas covariantes}\label{Ej:20}
Verifique que el término del lagrangiano del Modelo Estándar que contiene las derivadas covariantes del campo de Higgs puede escribirse como:
\[
(D_\mu \phi)^\dagger (D^\mu \phi)
= \frac{1}{2}(\partial_\mu h)(\partial^\mu h)
+ \frac{1}{8} g_W^2 (W_\mu^{(1)} + iW_\mu^{(2)})(W^{(1)\mu} - iW^{(2)\mu})(v+h)^2
\]
\[
+ \frac{1}{8}(g_W W_\mu^{(3)} - g' B_\mu)(g_W W^{(3)\mu} - g' B^\mu)(v+h)^2.
\]
\end{Ejercicio}

%%%%%%%%%%%%%%%%%%%%%%%%%%%%%%%%%%%%%%%%%%%%%%%%%%
%%%%%%%%%%%%%%%%%%%%%%%%%%%%%%%%%%%%%%%%%%%%%%%%%%
%%%%%%%%%%%%%%%%%%%%%%%%%%%%%%%%%%%%%%%%%%%%%%%%%%
%%%%%%%%%%%%%%%%%%%%%%%%%%%%%%%%%%%%%%%%%%%%%%%%%%
\begin{Ejercicio}{Transformación del doblete de Higgs bajo $SU(2)$}\label{Ej:21}
Demuestre que el campo $\phi_c$ transforma bajo $SU(2)$ de la misma forma que $\phi$ (el doblete de Higgs), donde
\[ \small
\phi =
\begin{pmatrix}
\phi^+ \\[4pt]
\phi^0
\end{pmatrix},
\qquad
\phi_c = -i\sigma_2 \phi^\ast
= -\frac{i}{\sqrt{2}}
\begin{pmatrix}
0 & -i\\[4pt]
i & 0
\end{pmatrix}
\begin{pmatrix}
\phi_1 - i\phi_2\\[4pt]
\phi_3 - i\phi_4
\end{pmatrix}
=
\frac{1}{\sqrt{2}}
\begin{pmatrix}
-\phi_3 + i\phi_4\\[4pt]
\phi_1 - i\phi_2
\end{pmatrix}
=
\begin{pmatrix}
-\phi^{0\ast}\\[4pt]
\phi^{+\ast}
\end{pmatrix}.
\]
\end{Ejercicio}
