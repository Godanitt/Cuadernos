
%%%%%%%%%%%%%%%%%%%%%%%%%%%%%%%%%%%%%%%%%%%%%%%%%%
%%%%%%%%%%%%%%%%%%%%%%%%%%%%%%%%%%%%%%%%%%%%%%%%%%
%%%%%%%%%%%%%%%%%%%%%%%%%%%%%%%%%%%%%%%%%%%%%%%%%%
%%%%%%%%%%%%%%%%%%%%%%%%%%%%%%%%%%%%%%%%%%%%%%%%%%

\begin{Ejercicio}{Espinores}
	Sean $\psi_\downarrow$ y $\xi_\uparrow$ dos espinores de Weyl zurdos y diestros independientes, respectivamente.  Demuestre que $\xi_\uparrow^\dagger \sigma^\mu \psi_\uparrow$ y $\xi_\downarrow^\dagger \bar{\sigma}^\mu \psi_\downarrow$ son cuatro-vectores contravariantes,  donde $\sigma^\mu \equiv (1,\symbf{\sigma})$ y $\bar{\sigma}^\mu \equiv (1,-\symbf{\sigma})$.
\end{Ejercicio}

Al ser $\sigma^\mu = (1,\sigman)$, la demostración se puede dividir en dos fases, en demostrar que $v^0=\xi_\downarrow^\dagger \psi_\downarrow$ y que $v^i =\xi_\downarrow^\dagger  \bar{\sigma}^i \psi_\downarrow$ (así mismo para los espinores $R$) son efectivamente invariantes Lorentz. Las transformaciones de Lorentz sobre un eje (por ej. el eje x):
\begin{eqnarray}
	V^0 \to V'^0 = \cosh (\eta) V^0 + \sinh (\eta)  V^{1}\\
	V^1 \to V'^1 = \cosh (\eta) V^1 + \sinh (\eta)  V^{0}
\end{eqnarray}
siendo $V'^2=V^2$ y $V'^3=V^3$. Con hacerlo sobre un eje basta, ya que siempre podremos encontrar un sistema de referencia donde el boost sea sobre ese eje x. Si nuestros cuatro-vecotres siguen dichas transformaciones, podemos decir que son contravariantes. Veamos uno por uno los dos casos:

\begin{itemize}
	\item Caso $\xi_\uparrow^\dagger \sigma^\mu \psi_\uparrow$. Una transformación Lorenzt para los espinores de Weyl a derechas es:
	      \begin{equation}
		      \Lambda_\uparrow = \exp \pqty{(-i\thetan  + \etan)\cdot \sigman}
	      \end{equation}
	      y recordemos que
	      \begin{equation}
		      v^0=\xi_\uparrow^\dagger \psi_\uparrow \qquad v^i =\xi_\uparrow^\dagger  \bar{\sigma}^i \psi_\uparrow
	      \end{equation}

	      Si el boost lo hacemos en el eje $x$, el único valor no nulo de $\etan$ es $\eta_1=\eta$. Así pues:
	      \begin{equation}
		      \Lambda_\uparrow = \exp \pqty{\eta  \sigma_1/2}
	      \end{equation}
	      Ahora tenemos que aplicar esto a nuestro $v_0$:

	      \begin{equation}
		      v^0 \to v^{\prime 0}= \xi_\uparrow^{\prime \dagger} \psi_\uparrow' = (\Lambda_\uparrow \xi_\uparrow^\dagger) (\Lambda_\uparrow \psi_\uparrow) = e^{\pqty{ \eta  \sigma^1}}\xi_\uparrow^\dagger \psi_\uparrow
	      \end{equation}
	      Usando que $e^{\pqty{ \eta  \sigma_1}} = \cosh (\eta)+\sigma^1 \eta $, tenemos que:
	      \begin{equation}
		      v^{\prime 0}= \cosh (\eta) \xi_\uparrow^\dagger \psi_\uparrow  + \sinh (\eta) \xi_\uparrow^\dagger  \sigma^1 \psi_\uparrow = \cosh v^0 + \sinh \eta v^1 =  \Lambda v^0
	      \end{equation}
	      q.e.d. Por otro lado, nos queda demostrar para $v^i$:

	      \begin{equation}
		      v^1 \to v^{\prime 1}= \xi_\uparrow^{\prime \dagger} \sigma_1 \psi_\uparrow' = \cosh (\eta) \xi_\uparrow^\dagger \sigma^1 \psi_\uparrow  + \sinh (\eta) \xi_\uparrow^\dagger \psi_\uparrow = \cosh (\eta) v^1 + \sinh (\eta) v^0
	      \end{equation}
	      donde nos hemos saltados algunos de los pasos. Dado que $\eta_2=\eta_3=0$, es trivial que $v^{\prime 2} = v^2$ y $v^{\prime 3} = v^3$, de lo que se deduce que efectivamente $xi_\uparrow^{\prime \dagger} \sigma^\mu \psi_\uparrow'$ transforma como un \text{4-vector contravariante}.
	\item El caso para la izquierda es análogo, aunque un poco didferente. En este caso
	      \begin{equation}
		      \Lambda_\downarrow = \exp \pqty{-\eta  \sigma_1/2} \qquad \exp \pqty{-\eta  \sigma_1} = \cosh (\eta) - \sinh (\eta) \eta_1
	      \end{equation}
	      lo cual hace que aparezca un signo menos, pero que debido a $\bar{\sigma}^\mu = (1,-\sigman)$, desaparece. Como hemos dicho, es repetir pasos.
\end{itemize}
pág. 267-268 del Maggiore \cite{Maggiore:2005qv}.

%%%%%%%%%%%%%%%%%%%%%%%%%%%%%%%%%%%%%%%%%%%%%%%%%%
%%%%%%%%%%%%%%%%%%%%%%%%%%%%%%%%%%%%%%%%%%%%%%%%%%
%%%%%%%%%%%%%%%%%%%%%%%%%%%%%%%%%%%%%%%%%%%%%%%%%%
%%%%%%%%%%%%%%%%%%%%%%%%%%%%%%%%%%%%%%%%%%%%%%%%%%

\begin{Ejercicio}{Transformaciones de Lorentz}
	Usando la representación quiral, demuestre que las transformaciones de Lorentz de los espinores de Dirac pueden escribirse en términos de las matrices $\gamma$ como
	\[
		\Psi \;\longrightarrow\; \Psi' = \exp\!\left(-\tfrac{i}{4}\omega_{\mu\nu}\sigma^{\mu\nu}\right)\Psi.
	\]
	Esta ecuación nos dice que $S^{\mu\nu} = \sigma^{\mu\nu}/2$ proporciona una representación de dimensión compleja cuatro del álgebra de Lorentz.
	Compruébelo directamente usando la definición de $\sigma^{\mu\nu}$ en términos de las matrices $\gamma$ y la relación $\{\gamma^\mu,\gamma^\nu\}=2g^{\mu\nu}$, es decir, verifique la siguiente relación de conmutación
	\[
		[S^{\mu\nu},S^{\rho\sigma}] = i\Big[g^{\nu\rho}S^{\mu\sigma} + g^{\mu\sigma}S^{\nu\rho} - g^{\nu\sigma}S^{\mu\rho} - g^{\mu\rho}S^{\nu\sigma}\Big].
	\]
\end{Ejercicio}

%%%%%%%%%%%%%%%%%%%%%%%%%%%%%%%%%%%%%%%%%%%%%%%%%%
%%%%%%%%%%%%%%%%%%%%%%%%%%%%%%%%%%%%%%%%%%%%%%%%%%
%%%%%%%%%%%%%%%%%%%%%%%%%%%%%%%%%%%%%%%%%%%%%%%%%%
%%%%%%%%%%%%%%%%%%%%%%%%%%%%%%%%%%%%%%%%%%%%%%%%%%


\begin{Ejercicio}{Identidad de Dirac}
	Demuestre la relación
	\[
		\frac{i}{\slashed{q}-m} = \frac{i(\slashed{q}+m)}{q^2-m^2}.
	\]
\end{Ejercicio}

%%%%%%%%%%%%%%%%%%%%%%%%%%%%%%%%%%%%%%%%%%%%%%%%%%
%%%%%%%%%%%%%%%%%%%%%%%%%%%%%%%%%%%%%%%%%%%%%%%%%%
%%%%%%%%%%%%%%%%%%%%%%%%%%%%%%%%%%%%%%%%%%%%%%%%%%
%%%%%%%%%%%%%%%%%%%%%%%%%%%%%%%%%%%%%%%%%%%%%%%%%%


\begin{Ejercicio}{Flujo invariante}
	Demuestre la siguiente identidad referente al flujo invariante de Lorentz,
	\[
		F = 4E_aE_b(v_a+v_b) = 4\sqrt{(p_a\cdot p_b)^2 - m_a^2 m_b^2}.
	\]
\end{Ejercicio}

%%%%%%%%%%%%%%%%%%%%%%%%%%%%%%%%%%%%%%%%%%%%%%%%%%
%%%%%%%%%%%%%%%%%%%%%%%%%%%%%%%%%%%%%%%%%%%%%%%%%%
%%%%%%%%%%%%%%%%%%%%%%%%%%%%%%%%%%%%%%%%%%%%%%%%%%
%%%%%%%%%%%%%%%%%%%%%%%%%%%%%%%%%%%%%%%%%%%%%%%%%%


\begin{Ejercicio}{Helicidad y Hamiltoniano de Dirac}
	Demuestre que el operador de helicidad conmuta con el hamiltoniano de Dirac, $[\hat{h},H_D]=0$, donde
	\[
		\hat{h} = \frac{\symbf{\Sigma}\cdot \hat{\mathbf{p}}}{2p}
		= \frac{1}{2p}\begin{pmatrix} \symbf{\sigma}\cdot\hat{\mathbf{p}} & 0 \\ 0 & \symbf{\sigma}\cdot\hat{\mathbf{p}} \end{pmatrix},
		\quad
		\hat{H}_D = \symbf{\alpha}\cdot\hat{\mathbf{p}} + \beta m,
	\]
	con
	\[
		\alpha_k = \gamma^0 \gamma^k, \qquad \beta = \gamma^0.
	\]
\end{Ejercicio}

%%%%%%%%%%%%%%%%%%%%%%%%%%%%%%%%%%%%%%%%%%%%%%%%%%
%%%%%%%%%%%%%%%%%%%%%%%%%%%%%%%%%%%%%%%%%%%%%%%%%%
%%%%%%%%%%%%%%%%%%%%%%%%%%%%%%%%%%%%%%%%%%%%%%%%%%
%%%%%%%%%%%%%%%%%%%%%%%%%%%%%%%%%%%%%%%%%%%%%%%%%%


\begin{Ejercicio}{Sección eficaz diferencial $e^-\mu^-$}\label{Ej:12}
	Usando amplitudes de helicidad, calcule la sección eficaz diferencial para el proceso
	\[
		e^- \mu^- \;\to\; e^- \mu^-,
	\]
	en el límite relativista ($m_e \approx 0$, $m_\mu \approx 0$), en el sistema del centro de masas.
\end{Ejercicio}

Nos piden la sección eficaz diferencial del scattering elástico $e^- \mu^- \;\to\; e^- \mu^-,$. Para ello tenemos que usar la ecuación:

\begin{equation}
	\dv{\sigma}{\Omega} = \frac{1}{64 \pi^2 s} \frac{p_f}{p_i} \vqty{\Mcal_{fi}}^2
\end{equation}
donde $\Mcal$ podemos hallarlo a través del siguiente diagrama de Feynmann a primer orden
\begin{figure}[h]
	\centering
	\feynmandiagram [vertical'=a to b] {
	i1 [particle=\(e^-\)] -- [anti fermion, rmomentum=\(p^\mu\)] a -- [anti fermion, rmomentum=\(p'^\mu\)] f1 [particle=\(e^-\)],
	a -- [photon, edge label=\(\gamma\)] b,
	i2 [particle=\(\mu^-\)] -- [fermion, momentum=\(k^\mu\)] b -- [fermion, momentum=\(k'^\mu\)] f2 [particle=\(\mu^-\)],
	};
	\caption{Diagrama de Feynman para \(e^- \mu^- \to e^- \mu^-\).}
\end{figure}
Nuestro elemento de matriz
\begin{equation}
	i \Mcal =e^2 \underbrace{\pqty{\bar{u}^{s'}(p') Q_e  \gamma^\mu u^{s}(p)}}_{j_e^\mu}  \frac{-ig_{\mu \nu}}{q^2}  \underbrace{\pqty{\bar{u}^{r'}(k') Q_\mu e \gamma^\nu u^{r}(k)}}_{{j_\mu^\nu} }
\end{equation}
tal que
\begin{equation}
	i \Mcal =\frac{e^2}{t} (j_e \cdot j_\mu)
\end{equation}
con $Q=-1$ en ambos casos y $q^2=t$ el invariange de Maddelstam ($t=|p'^\mu-p^\mu|^2$). Dado que no nos especifican los espines de entrada, podemos considerar que el elemento de matriz total es el no polarizado:
\begin{equation}
	\vqty{\Mcal_{fi}}^2 = \langle \vqty{\Mcal_{fi}}^2 \rangle = \frac{1}{4} \sum_{ss',r,r'} \vqty{\Mcal_{s,r}}
\end{equation}
O alternativamente con \textit{las amplitudes de helicidad}:
\begin{equation}
	\vqty{\Mcal_{fi}}^2 = \frac{1}{4} \pqty{\vqty{\Mcal_{\text{RR}}}^2+\vqty{\Mcal_{\text{RL}}}^2
		+\vqty{\Mcal_{\text{LR}}}^2+\vqty{\Mcal_{\text{LL}}}^2}
\end{equation}
Solo tenemos que considerar los siguientes espinores (al no haber antipartículas con consideramos $v$ y $\bar{v}$) con esta forma ($m_e\approx m_\mu\approx 0$), aunque hay que tener en cuenta que al tener momentso $p_z$ y $-p_z$ cambian algunos signos:
\begin{equation}
	u_\uparrow = \sqrt{E} \begin{pmatrix}
		c \\ se^{i\phi} \\ c \\ se^{i\phi}
	\end{pmatrix} \qquad
	u_\downarrow= \sqrt{E} \begin{pmatrix}
		-s \\ ce^{i\phi} \\ s \\ -ce^{i\phi}
	\end{pmatrix}
\end{equation}
con $c=\cos (\theta/2)$ y $s=\sin(\theta/2)$, siendo este $\theta$ el ángulo de salida de las partículas. Los momentos de nuestras partículas (caso de masa despreciable $m_e\approx m_\mu\approx 0$)
\begin{equation}
    p^\mu= (E,0,0,E) \qquad 
    p^{\prime\mu}= (E,E\sin \theta,0,E \cos \theta) \qquad 
\end{equation}
\begin{equation}
    k^\mu= (E,0,0,-E) \qquad 
    k^{\prime\mu}= (E,-E\sin \theta,0,-E \cos \theta)
\end{equation}
Dado que $\pn(e^-)=-\pn(\mu^-)$, tenemos que algunos $u_\uparrow (p) \neq u_\uparrow (k)$. Usando que 

\[
u_{\uparrow} = \sqrt{E + m}\,
\begin{pmatrix}
c \\[6pt]
s e^{i\phi} \\[6pt]
\dfrac{p}{E + m}\, c \\[6pt]
\dfrac{p}{E + m}\, s e^{i\phi}
\end{pmatrix},
\qquad
u_{\downarrow} = \sqrt{E + m}\,
\begin{pmatrix}
-\,s \\[6pt]
c e^{i\phi} \\[6pt]
\dfrac{p}{E + m}\, s \\[6pt]
-\dfrac{p}{E + m}\, c e^{i\phi}
\end{pmatrix},
\]
de Saborido \cite{SaboridoSilva2025ParticlePhysicsI}, podemos ver las siguientes ecuaciones: 
\begin{equation}
	u_\uparrow(p) = \sqrt{E} \begin{pmatrix}
		1 \\ 0 \\ 1 \\ 0
	\end{pmatrix} \quad
	u_\downarrow(p)= \sqrt{E} \begin{pmatrix}
		0 \\ 1 \\ 0 \\ -1
	\end{pmatrix}\quad
	u_\uparrow(p') = \sqrt{E} \begin{pmatrix}
		c \\ s \\ c \\ s
	\end{pmatrix} \quad
	u_\downarrow(p')= \sqrt{E} \begin{pmatrix}
		-s \\ c \\ s \\ -c
	\end{pmatrix}
\end{equation}
\begin{equation}
	u_\uparrow(k) = \sqrt{E} \begin{pmatrix}
		1 \\ 0 \\ -1 \\ 0
	\end{pmatrix} \quad
	u_\downarrow(k)= \sqrt{E} \begin{pmatrix}
		0 \\ 1 \\ 0 \\ 1
	\end{pmatrix}  \quad
	u_\uparrow(k') = \sqrt{E} \begin{pmatrix}
		c \\ s \\ -c \\ -s
	\end{pmatrix} \quad
	u_\downarrow(k')= \sqrt{E} \begin{pmatrix}
		-s \\ c \\ -s \\ c
	\end{pmatrix}
\end{equation}
para $k'$ y $p'$ usaremos la expresión general. Usando las relaciones del Thomson \cite{thomson_modern_physics}
\begin{align}
	\overline{\psi}\gamma^0\phi & = \psi^\dagger \gamma^0 \gamma^0 \phi
	= \psi_1^*\phi_1 + \psi_2^*\phi_2 + \psi_3^*\phi_3 + \psi_4^*\phi_4,     \\
	\overline{\psi}\gamma^1\phi & = \psi^\dagger \gamma^0 \gamma^1 \phi
	= \psi_1^*\phi_4 + \psi_2^*\phi_3 + \psi_3^*\phi_2 + \psi_4^*\phi_1,     \\
	\overline{\psi}\gamma^2\phi & = \psi^\dagger \gamma^0 \gamma^2 \phi
	= -i(\psi_1^*\phi_4 - \psi_2^*\phi_3 + \psi_3^*\phi_2 - \psi_4^*\phi_1), \\
	\overline{\psi}\gamma^3\phi & = \psi^\dagger \gamma^0 \gamma^3 \phi
	= \psi_1^*\phi_3 - \psi_2^*\phi_4 + \psi_3^*\phi_1 - \psi_4^*\phi_2.
\end{align}
Tenemos que:


\begin{minipage}[t]{0.48\textwidth}
\begin{align*}
    \bar{u}_\uparrow (p') \gamma_0 {u}_\uparrow (p) & = 2Ec \\
    \bar{u}_\uparrow (p') \gamma_1 {u}_\uparrow (p) & = 2Es \\
    \bar{u}_\uparrow (p') \gamma_2 {u}_\uparrow (p) & = -i2Es \\
    \bar{u}_\uparrow (p') \gamma_3 {u}_\uparrow (p) & = 2Ec
\end{align*}
\begin{align*}
    \bar{u}_\uparrow(p') \gamma_0 {u}_\downarrow (p) & = 0 \\
    \bar{u}_\uparrow(p') \gamma_1 {u}_\downarrow (p) & = 0 \\
    \bar{u}_\uparrow(p') \gamma_2 {u}_\downarrow (p) & = 0 \\
    \bar{u}_\uparrow(p') \gamma_3 {u}_\downarrow (p) & = 0
\end{align*}
\end{minipage}\hfill
\begin{minipage}[t]{0.48\textwidth}
\begin{align*}
    \bar{u}_\downarrow (p') \gamma_0 {u}_\uparrow (p) & = 0 \\
    \bar{u}_\downarrow (p') \gamma_1 {u}_\uparrow (p) & = 0 \\
    \bar{u}_\downarrow (p') \gamma_2 {u}_\uparrow (p) & = 0 \\
    \bar{u}_\downarrow (p') \gamma_3 {u}_\uparrow (p) & = 0
\end{align*}
\begin{align*}
    \bar{u}_\downarrow(p') \gamma_0 {u}_\downarrow (p) & = 2Ec \\
    \bar{u}_\downarrow(p') \gamma_1 {u}_\downarrow (p) & = 2Es \\
    \bar{u}_\downarrow(p') \gamma_2 {u}_\downarrow (p) & = -i2Es \\
    \bar{u}_\downarrow(p') \gamma_3 {u}_\downarrow (p) & = 2Ec
\end{align*}
\end{minipage}
%%%%%%%%%%%%%%%%%%%%%%%%%%%%%%%%%%%%%%%%%%%%%%%5
\begin{minipage}[t]{0.48\textwidth}
\begin{align*}
    \bar{u}_\uparrow (k') \gamma_0 {u}_\uparrow (k) & = 2Ec \\
    \bar{u}_\uparrow (k') \gamma_1 {u}_\uparrow (k) & = -2Es \\
    \bar{u}_\uparrow (k') \gamma_2 {u}_\uparrow (k) & = i2Es \\
    \bar{u}_\uparrow (k') \gamma_3 {u}_\uparrow (k) & = -2Ec
\end{align*}
\begin{align*}
    \bar{u}_\uparrow(k') \gamma_0 {u}_\downarrow (k) & = 0 \\
    \bar{u}_\uparrow(k') \gamma_1 {u}_\downarrow (k) & = 0 \\
    \bar{u}_\uparrow(k') \gamma_2 {u}_\downarrow (k) & = 0 \\
    \bar{u}_\uparrow(k') \gamma_3 {u}_\downarrow (k) & = 0
\end{align*}
\end{minipage}\hfill
\begin{minipage}[t]{0.48\textwidth}
\begin{align*}
    \bar{u}_\downarrow (k') \gamma_0 {u}_\uparrow (k) & = 0 \\
    \bar{u}_\downarrow (k') \gamma_1 {u}_\uparrow (k) & = 0 \\
    \bar{u}_\downarrow (k') \gamma_2 {u}_\uparrow (k) & = 0 \\
    \bar{u}_\downarrow (k') \gamma_3 {u}_\uparrow (k) & = 0
\end{align*}
\begin{align*}
    \bar{u}_\downarrow(k') \gamma_0 {u}_\downarrow (k) & = 2Ec \\
    \bar{u}_\downarrow(k') \gamma_1 {u}_\downarrow (k) & = -2Es \\
    \bar{u}_\downarrow(k') \gamma_2 {u}_\downarrow (k) & = -i2Es \\
    \bar{u}_\downarrow(k') \gamma_3 {u}_\downarrow (k) & = -2Ec
\end{align*}
\end{minipage}
Es decir, \textit{cada vértice debe conservar la helicidad} (que básicamente implica conservar espín). Ahora tenemos que calcular cada término. Lo que está claro que es que, si en un vértice no se conseva helicidad la amplitud de probabilidad es nula. Así pues: 

\begin{equation}
    \vqty{\Mcal_{\text{LL}}}^2 = \vqty{\Mcal_{\text{LL}\to\text{LL}}}^2 = \frac{e^4}{t^2}\vqty{4E^2c^2+4E^2c^2+4E^2s^2+4E^2s^2} = \frac{e^4}{t^2}  64 E^4 = \frac{e^4}{t^2}  4s^2
\end{equation}
\begin{equation}
    \vqty{\Mcal_{\text{RR}}}^2 = \vqty{\Mcal_{\text{RR}\to\text{RR}}}^2 = \frac{e^4}{t^2}\vqty{4E^2c^2+4E^2c^2} = \frac{e^4}{t^2}  64 c^4 E^4 = 16 E^4 (1+\cos \theta)^2 =  \frac{e^4}{t^2} 4 u^2
\end{equation}
\begin{equation}
    \vqty{\Mcal_{\text{RL}}}^2 = \vqty{\Mcal_{\text{RL\to\text{RL}}}}^2 = \frac{e^4}{t^2}\vqty{4E^2c^2+4E^2c^2+4E^2s^2+4E^2s^2} = \frac{e^4}{t^2}  64 E^4 = \frac{e^4}{t^2}  4s^2
\end{equation}
\begin{equation}
    \vqty{\Mcal_{\text{LR}}}^2 = \vqty{\Mcal_{\text{LR}\to\text{LR}}}^2 = \frac{e^4}{t^2}\vqty{4E^2c^2+4E^2c^2} = \frac{e^4}{t^2}  64 c^4 E^4 = 16 E^4 (1+\cos \theta)^2 =  \frac{e^4}{t^2} 4 u^2
\end{equation}
donde hemos usado que $s=(p+k)^2=4E^2$ y $u=(p-k')^2=-2(p\cdot k')=-2E(1+\cos \theta)$. Luego el elemento de matriz total cuando $m_e \approx m_\mu \approx 0$ es: 

\begin{equation}
    \vqty{\Mcal_{ij}} = \frac{2e^4}{t^2} (s^2+u^2)
\end{equation}
que es lo que queríamos demostrar. De hecho, podemos hacer el cálculo con trazas para comprobar si da lo mismo. Así pues, tenemos que:

\begin{equation}
	\vqty{\Mcal_{s,r}}^2 = e^4 \bqty{\pqty{\bar{u}^{s'}(p') \gamma^\mu u^{s}(p)}\pqty{\bar{u}^{r'}(k') \gamma_\mu u^{r}(k)}} \bqty{\pqty{\bar{u}^{s'}(p') \gamma^\nu u^{s}(p)}\pqty{\bar{u}^{r'}(k')\gamma_\nu u^{r}(k)}}^*
\end{equation}
Tal que usando el truco de Casimir (Williams \cite{Williams_2022})

\[
	\sum_{s_1,s_2}
	\bigl[\bar{u}_1^{s_1}(p_1)\,\Gamma\,u_2^{s_2}(p_2)\bigr]
	\bigl[\bar{u}_1^{s_1}(p_1)\,\Gamma' u_2^{s_2}(p_2)\bigr]^*
	= \mathrm{tr}\!\left[ \Gamma \,(\slashed{p}_2+m_2)\,\bar{\Gamma}'\,(\slashed{p}_1+m_1) \right].
\]
tenemos que

\begin{equation}
	\vqty{\Mcal_{s,r}}^2 =\frac{2e^4}{t^2} \tr \pqty{\gamma^\mu (\pslash+m_e)\gamma^\nu (\pslash+m_e)}\tr \pqty{\gamma^\mu (\kslash+m_\mu)\gamma^\nu (\kslash+m_\mu)}
\end{equation}
tal que:

\begin{itemize}
	\item La primera traza:
	      \begin{equation}
		      \tr \pqty{\gamma^\mu (\pslash+m_e)\gamma^\nu (\pslash'+m_e)} = \tr (\gamma^\mu \pslash \gamma^\nu \pslash') + 4g^{\mu \nu}m_e^2
	      \end{equation}
	      Ahora la traza más compleja
	      \begin{align}
		      \tr (\gamma^\mu \pslash \gamma^\nu \pslash')
		       & = 4 g^{\mu \sigma} p_\sigma g^{\nu \rho} p_{\rho}^{\prime}
		      + 4 g^{\mu \rho} p_{\rho}^{\prime } g^{\nu \sigma} p_{\sigma}
		      - 4 p_{\mu \nu} (p \cdot p')                                                    \\
		       & = 4 p^\mu p^{\prime \nu} + 4 p^\nu p^{\prime\mu} - 4 g^{\mu \nu} (p\cdot p')
	      \end{align}
	\item La segunda traza:
	      \begin{equation}
		      \tr \pqty{\gamma_\mu (\kslash+m_\mu)\gamma_\nu (\kslash'+m_\mu)} = \tr (\gamma_\mu \kslash \gamma_\nu \kslash') + 4g_{\mu \nu}m_\mu^2
	      \end{equation}
	      Ahora la traza más compleja
	      \begin{align}
		      \tr (\gamma_\mu \kslash \gamma_\nu \kslash')
		       & = 4 g_{\mu \sigma} k^\sigma g_{\nu \rho} k^{\prime \rho}
		      + 4 g_{\mu \rho} k^{\prime \rho} g_{\nu \sigma} k^{\sigma}
		      - 4 g_{\mu \nu} (k \cdot k')                                     \\
		       & = 4 k_\mu k'_\nu + 4 k_\nu k'_\mu - 4 g_{\mu \nu} (k\cdot k')
	      \end{align}
\end{itemize}
Directamente  multiplicando:
\begin{equation}
	\vqty{\Mcal_{s,r}}^2 = 32 (p\cdot k)(p'\cdot k') + 32 (p \cdot k')(p'\cdot k)- 32(p\cdot p')m_\mu^2 - 32(k\cdot k')m_e^2 + 64 m_e^2 m_\mu^2
\end{equation}
Ahora solo tenemos que agrupar y dividir entre $1/4t$.
\begin{equation}
	\vqty{\Mcal_{fi}}^2 = \frac{8}{t} \pqty{(p\cdot k)(p'\cdot k') +  (p \cdot k')(p'\cdot k)- (p\cdot p')m_\mu^2 - (k\cdot k')m_e^2 + 2 m_e^2 m_\mu^2}
\end{equation}
Usando los invariantes de Maddelstam:
\begin{equation}
	s = (p+k)^2 = (p'+k')^2 \qquad t = (p-p')^2 = (k-k')^2 \qquad u = (p-k')^2 = (p'-k)^2
\end{equation}
tal que:
\begin{equation}
	s = m_e^2 + m_\mu^2 +2 (p \cdot k) = m_e^2 + m_\mu^2 + 2 (p' \cdot k')
\end{equation}
\begin{equation}
	t = 2m_e^2 + m_\mu^2 -2 (p \cdot p') = 2m_\mu^2 - 2 (k \cdot k')
\end{equation}
\begin{equation}
	u = m_e^2 + m_\mu^2 -2 (p \cdot k') = m_e^2 + m_\mu^2 - 2 (p'\cdot k)
\end{equation}
Con la relación:
\begin{equation}
	s+t+u = 2(m_e^2 + m_\mu^2)
\end{equation}
Ahora:
\begin{align}
	\vqty{\Mcal_{fi}}^2 & = \frac{2e^4}{t^2} \pqty{(s-m_\mu^2-m_e^2)^2 + (u-m_e^2-m_\mu^2)^2  +  2 t(m_e^2 + m_\mu^2)}        \\
	    & = \frac{2e^4}{t^2} \pqty{s^2 + u^2 - 2(s+\mu)(m_\mu^2 +m_e^2) +  2 t(m_e^2 + m_\mu^2) + 2(m_\mu^2 +m_e^2)^2}
\end{align}
que aplicando $ s+u = 2(m_e^2 + m_\mu^2) -t$ tenemos:
\begin{align}
	\vqty{\Mcal_{fi}}^2 & = \frac{2e^4}{t^2} (s^2 + u^2 + 4t(m_\mu^2 + m_e^4) - 2(m_\mu^2+m_e^2)^2)
\end{align}
que efectivamente si $m_e \approx m_\mu \approx 0$ tenemos

\begin{align}
	\vqty{\Mcal_{fi}}^2 & = \frac{2e^4}{t^2} (s^2 + u^2 )
\end{align}
que es lo mismo que obtuvimos antes \cite{Williams_2022}. Lo que implica que, la sección eficaz diferencial es, en este caso:

\begin{equation}
    \dv{\sigma}{\Omega} = \frac{1}{64\pi^2 s} \frac{e^4 (s^2 + u^2)}{t^2}
\end{equation}
que si queremos expresar con $E$ y $\cos \theta$ explícitamente, tenemos que $t=-2(p_1\cdot p_3)=-2E(1-\cos \theta)$

\begin{equation}
    \dv{\sigma}{\Omega} = \frac{16}{64\pi^2 s} \frac{ e^4 E^2 \pqty{+\frac{1}{4}(1+\cos \theta)}}{4E^2(1-\cos \theta)^2}
\end{equation}
y la expresión final, usando la constante de estructura fina $2\alpha^2 = 2(e^4/16\pi^2)$ tenemos: 
\begin{equation}
    \dv{\sigma}{\Omega} = \frac{2\alpha^2}{s} \frac{{+\frac{1}{4}(1+\cos \theta)}}{(1-\cos \theta)^2}
\end{equation}
que coincide con la del Thomson \cite{thomson_modern_physics}.

%%%%%%%%%%%%%%%%%%%%%%%%%%%%%%%%%%%%%%%%%%%%%%%%%%
%%%%%%%%%%%%%%%%%%%%%%%%%%%%%%%%%%%%%%%%%%%%%%%%%%
%%%%%%%%%%%%%%%%%%%%%%%%%%%%%%%%%%%%%%%%%%%%%%%%%%
%%%%%%%%%%%%%%%%%%%%%%%%%%%%%%%%%%%%%%%%%%%%%%%%%%


\begin{Ejercicio}{Elemento de matriz en QED con trazas}\label{Ej:07}
	Use el formalismo de trazas para calcular el elemento de matriz al cuadrado, promediado en espines, de la electrodinámica cuántica (QED) para
	\[
		e^+ e^- \;\to\; f \bar{f},
	\]
	sin despreciar ninguna de las masas de las partículas.
\end{Ejercicio}

\begin{figure}[h]
	\centering
	\feynmandiagram [horizontal'=a to b] {
	i1 [particle=\(e^-\)] -- [fermion, momentum=\(p^\mu\)] a -- [fermion,rmomentum=\(k^\mu\)] i2 [particle=\(e^+\)],
	a -- [photon, edge label=\(\gamma\)] b,
	f1 [particle=\( f^+\)] -- [fermion,rmomentum=\(k'^\mu\)] b -- [fermion,momentum=\(p'^\mu\)] f2 [particle=\(f^-\)],
	};
	\caption{Diagrama de Feynman para \(e^+ e^- \to f^+ f^-\).}
\end{figure}
Nuestro elemento de matriz
con $Q=-1$ en ambos casos y $q^2=s$ el invariange de Maddelstam ($s=|k^\mu+p^\mu|^2$). Dado que no nos especifican los espines de entrada, podemos considerar que el elemento de matriz total es el no polarizado:
\begin{equation}
	\vqty{\Mcal_{fi}}^2 = \langle \vqty{\Mcal_{fi}}^2 \rangle = \frac{1}{4} \sum_{s,s',r,r'} \vqty{\Mcal_{s,r}}
\end{equation}
Así pues, tenemos que:

\begin{equation}
	\vqty{\Mcal_{s,r}}^2 = e^4 \bqty{\pqty{\bar{v}^{s'}(k) \gamma^\mu u^{s}(p)}\pqty{\bar{u}^{r'}(p') \gamma_\mu v^{r}(k')}} \bqty{\pqty{\bar{v}^{s'}(k) \gamma^\nu u^{s}(p)}\pqty{\bar{u}^{r'}(p')\gamma_\nu v^{r}(k')}}^*
\end{equation}
Tal que usando el truco de Casimir (Williams \cite{Williams_2022})

\[
	\sum_{s_1,s_2}
	\bigl[\bar{u}_1^{s_1}(p_1)\,\Gamma\,u_2^{s_2}(p_2)\bigr]
	\bigl[\bar{u}_1^{s_1}(p_1)\,\Gamma' u_2^{s_2}(p_2)\bigr]^*
	= \mathrm{tr}\!\left[ \Gamma \,(\slashed{p}_2+m_2)\,\bar{\Gamma}'\,(\slashed{p}_1+m_1) \right].
\]
tenemos que
\begin{equation}
	\vqty{\Mcal_{s,r}}^2 =\frac{2e^4}{t^2} \tr \pqty{\gamma^\mu (\pslash+m_e)\gamma^\nu (\kslash-m_e)}\tr \pqty{\gamma^\mu (\pslash'+m_\mu)\gamma^\nu (\kslash'-m_\mu)}
\end{equation}
tal que:

\begin{itemize}
	\item La primera traza:
	      \begin{equation}
		      \tr \pqty{\gamma^\mu (\pslash+m_e)\gamma^\nu (\kslash-m_e)} = \tr (\gamma^\mu \pslash \gamma^\nu \kslash) - 4g^{\mu \nu}m_e^2
	      \end{equation}
	      Ahora la traza más compleja
	      \begin{align}
		      \tr (\gamma^\mu \pslash \gamma^\nu \kslash)
		       & = 4 g^{\mu \sigma} p_\sigma g^{\nu \rho} k_{\rho}
		      + 4 g^{\mu \rho} k_{\rho} g^{\nu \sigma} p_{\sigma}
		      - 4 p_{\mu \nu} (p \cdot k)                                                    \\
		       & = 4 p^\mu p^{\prime \nu} + 4 p^\nu k^{\mu} - 4 g^{\mu \nu} (p\cdot k)
	      \end{align}
	\item La segunda traza:
	      \begin{equation}
		      \tr \pqty{\gamma_\mu (\pslash'+m_f)\gamma_\nu (\kslash'-m_f)} = \tr (\gamma_\mu \pslash' \gamma_\nu \kslash') - 4g_{\mu \nu}m_f^2
	      \end{equation}
	      Ahora la traza más compleja
	      \begin{align}
		      \tr (\gamma_\mu \pslash' \gamma_\nu \kslash')
		       & = 4 g_{\mu \sigma} p^{\prime\sigma} g_{\nu \rho} k^{\prime \rho}
		      + 4 g_{\mu \rho} k^{\prime \rho} g_{\nu \sigma} p^{\prime \sigma}
		      - 4 g_{\mu \nu} (p' \cdot k')                                     \\
		       & = 4 p'_\mu k'_\nu + 4 p'_\nu k'_\mu - 4 g_{\mu \nu} (p'\cdot k')
	      \end{align}
\end{itemize}
Directamente  multiplicando:
\begin{equation}
	\frac{s^2}{e^4}\vqty{\Mcal_{s,r}}^2 = 32 (p\cdot p')(k\cdot k') + 32 (p \cdot k')(k\cdot p') + 32(p\cdot k)m_f^2 + 32(p'\cdot k')m_e^2 + 64 m_e^2 m_f^2
\end{equation}
Ahora solo tenemos que agrupar y dividir entre $1/4t$.
\begin{equation}
	\vqty{\Mcal_{fi}}^2 = \frac{8e^4}{s^2} \pqty{(p\cdot k)(p'\cdot k') +  (p \cdot k')(p'\cdot k)- (p\cdot p')m_f^2 - (k\cdot k')m_e^2 + 2 m_e^2 m_f^2}
\end{equation}
Usando los invariantes de Maddelstam:
\begin{equation}
	s = (p+k)^2 = (p'+k')^2 \qquad t = (p-p')^2 = (k-k')^2 \qquad u = (p-k')^2 = (p'-k)^2
\end{equation}
tal que:
\begin{equation}
	s = m_e^2 + m_f^2 +2 (p \cdot k) = m_e^2 + m_f^2 + 2 (p' \cdot k')
\end{equation}
\begin{equation}
	t = 2m_e^2 + m_f^2 -2 (p \cdot p') = 2m_f^2 - 2 (k \cdot k')
\end{equation}
\begin{equation}
	u = m_e^2 + m_f^2 -2 (p \cdot k') = m_e^2 + m_f^2 - 2 (p'\cdot k)
\end{equation}
Con la relación:
\begin{equation}
	s+t+u = 2(m_e^2 + m_f^2)
\end{equation}
Ahora:
\begin{align}
	\vqty{\Mcal_{fi}}^2 & = \frac{2e^4}{s^2} \pqty{(t-m_f^2-m_e^2)^2 + (u-m_e^2-m_f^2)^2  +  2 s(m_e^2 + m_f^2)}        \\
	& = \frac{2e^4}{s^2} \pqty{t^2 + u^2 - 2(-s+2(m_e^2 + m_f^2))(m_f^2 +m_e^2) +  2 s(m_e^2 + m_f^2) + 2(m_f^2 +m_e^2)^2}
\end{align}
que aplicando $ t+u = 2(m_e^2 + m_f^2) -s$ tenemos:
\begin{align}
	\vqty{\Mcal_{fi}}^2 & = \frac{2e^4}{s^2} (t^2 + u^2 + 4s(m_f^2 + m_e^4) - 2(m_f^2+m_e^2)^2)
\end{align}
que es el resultado final \cite{Williams_2022}, y que como podemos ver es igual a la dispersión $e^- f^- \to e^- f^-$ pero haciendo el cambio $s \leftrightarrow t$. 

%%%%%%%%%%%%%%%%%%%%%%%%%%%%%%%%%%%%%%%%%%%%%%%%%%
%%%%%%%%%%%%%%%%%%%%%%%%%%%%%%%%%%%%%%%%%%%%%%%%%%
%%%%%%%%%%%%%%%%%%%%%%%%%%%%%%%%%%%%%%%%s%%%%%%%%%%
%%%%%%%%%%%%%%%%%%%%%%%%%%%%%%%%%%%%%%%%%%%%%%%%%%


\begin{Ejercicio}{Dispersión $e^-p$ y transferencia de momento}\label{Ej:14}
	En un experimento de dispersión $e^-p$, la energía del electrón incidente es $E_1 = 529.5\,\mathrm{MeV}$ y los electrones dispersados se detectan a un ángulo $\theta = 75^\circ$ con respecto al haz incidente.
	A este ángulo, casi todos los electrones dispersados se miden con una energía de $E_3 \approx 373\,\mathrm{MeV}$.

	\medskip
	¿Qué se puede concluir a partir de esta observación?
	Encuentre el valor correspondiente de
	\[
		Q^2 = -q^2.
	\]
\end{Ejercicio}

En este ejercicio tenemos que usar los cuadrimomentos para ver la energía y momento del protón para el ángulo con el que salen los electrones. 

%%%%%%%%%%%%%%%%%%%%%%%%%%%%%%%%%%%%%%%%%%%%%%%%%%
%%%%%%%%%%%%%%%%%%%%%%%%%%%%%%%%%%%%%%%%%%%%%%%%%%
%%%%%%%%%%%%%%%%%%%%%%%%%%%%%%%%%%%%%%%%%%%%%%%%%%
%%%%%%%%%%%%%%%%%%%%%%%%%%%%%%%%%%%%%%%%%%%%%%%%%%


\begin{Ejercicio}{Tasa de decaimiento del pión en interacción escalar}\label{Ej:15}
	Calcule la tasa de decaimiento del pión para una interacción puramente escalar y muestre que la razón predicha de tasas de decaimiento es
	\[
		\frac{\Gamma(\pi^- \to e^- \bar{\nu}_e)}{\Gamma(\pi^- \to \mu^- \bar{\nu}_\mu)} \;\approx\; 5.5.
	\]

	\textit{Sugerencia:} suponga que la interacción está mediada por una partícula escalar masiva $X$, de modo que la cantidad escalar asociada al vértice $\ell-\nu_\ell$ con constante de acoplo $g_X$ sería
	\[
		j_\ell = g_X \, \bar{u} v.
	\]
	Puede trabajar también en el régimen $q^2 = m_\pi^2 \ll m_X^2$ y tomar el propagador como $1/m_X^2$ (interacción tipo Fermi puntual).
\end{Ejercicio}

Tenemos que la tasa de decaimiento viene dada por la expresión: 

\begin{equation}
    \Gamma = \frac{p^* }{8\pi m_a^2} |\Mcal_{fi}|^2
\end{equation}
siendo $p^*$ el momento en el centro de masas de las partículas salientes ($p^*=|\pn|=|\kn|$), con los siguientes momentos:
\begin{equation}
    p_\pi^\mu = (m_\pi,0,0,0) \qquad p^\mu =(E_1,0,0,p_1) \qquad  k^\mu =(E_2,0,0,-p_1) 
\end{equation}
Así pues, solo tenemos que calcular el elemento de matriz $\Mcal_{fi}$, que se calcula a través del siguiente diagrama:


\begin{figure}[h]
	\centering
	\feynmandiagram [horizontal'=a to b] {
		i1 [particle=\(\pi^-\)] -- [fermion] a
			-- [boson, edge label=\(X\)] b,
		b -- [fermion, momentum'=\(p\)] f1 [particle=\(\ell^-\)],
		b -- [anti fermion, momentum=\(k\)] f2 [particle=\(\bar{\nu}_\ell\)],
	};
	\caption{Diagrama de Feynman para \(\pi^- \to \ell^- \bar{\nu}_\ell\).}
\end{figure}
Al hacer un acoplo puramente escalar tenemos que: 
\begin{equation}
    \Mcal_{fi} = f_\pi p_\pi \frac{g_X}{M_X^2} \bar{u}(p) v(k)
\end{equation}
Luego, la única dificultad para calcular $\vqty{\Mcal_{ij}}^2$ es hacer: 
\begin{equation}
    \sum_{\text{espines}} \pqty{\bar{u}(p) v(k)}  \pqty{\bar{u}(p) v(k)}^\dagger = \tr \pqty{(\pslash+m_l)(\kslash-m_\nu)} = 4 p^\mu k_\mu - 4 m_l m_\nu 
\end{equation}
que es, según los cálculos anteriores, y suponiendo que $m_\nu \to 0$: 
\begin{equation}
    |\Mcal_{fi}|^2 = \bqty{f_\pi m_\pi \frac{g_X}{M_X^2} 2 (p \cdot k)}^2
\end{equation}
tal que 

\begin{equation}
    p_\pi^\mu = p^\mu + k^\mu \to m_\pi^2 = m_l^2 + 2 (p \cdot k) \to (p\cdot k) = \frac{1}{2} (m_\pi^2 - m_l^2)
\end{equation}
y por tanto: 

\begin{equation}
    \Gamma = \frac{p^* }{8\pi m_\pi^2} \bqty{f_\pi m_\pi \frac{g_X}{M_X^2} (m_\pi^2 - m_l^2)}^2
\end{equation}
tal que: 

\begin{equation}
    \Gamma = \frac{f_\pi^2 g_X^2}{8 \pi m_X^2} (m_\pi^2 - m_l^2)^2
\end{equation}
De lo que se deduce que 

\begin{equation}
   \frac{\Gamma(\pi^- \to e^- \bar{\nu}_e)}{\Gamma(\pi^- \to \mu^- \bar{\nu}_\mu)} = \frac{(m_\pi^2 - m_e^2)^2}{(m_\pi^2 - m_\mu^2)^2} = 5.5
\end{equation}


%%%%%%%%%%%%%%%%%%%%%%%%%%%%%%%%%%%%%%%%%%%%%%%%%%
%%%%%%%%%%%%%%%%%%%%%%%%%%%%%%%%%%%%%%%%%%%%%%%%%%
%%%%%%%%%%%%%%%%%%%%%%%%%%%%%%%%%%%%%%%%%%%%%%%%%%
%%%%%%%%%%%%%%%%%%%%%%%%%%%%%%%%%%%%%%%%%%%%%%%%%%

\begin{Ejercicio}{Vértices gauge en la parte cinética del lagrangiano electrodébil}\label{Ej:16}
Muestre explícitamente que la parte cinética del lagrangiano electrodébil contiene dos vértices triple–gauge
\[
\gamma W^+W^-,\qquad ZW^+W^-,
\]
y cuatro vértices cuárticos gauge
\[
W^+W^-W^+W^-,\qquad W^+W^-ZZ,\qquad W^+W^-\gamma\gamma,\qquad W^+W^-Z\gamma.
\]
\end{Ejercicio}
L manera mas rápida de resolver el ejercicio sería aplicar la siguiente ecuación \cite{thomson_modern_physics}, 

\begin{equation}
	\mathcal{L}_{\text{int}} 
	= +\frac{1}{2} g_W \, \epsilon_{ijk} 
	\left( \partial^{\mu} W_i^{\nu} - \partial^{\nu} W_i^{\mu} \right)
	W_{j\mu} W_{k\nu}
	- \frac{1}{4} g_W^2 \, \epsilon_{ijk} \epsilon_{imn}
	W_j^{\mu} W_k^{\nu} W_{m\mu} W_{n\nu}.
\end{equation}
con las permutaciones necesarias, de una en una. Aquí decidimos hacerlo por partes, probablemente una decisión equivocada. En cualquier caso ya está resuelto. Veamos que la parte cinética del lagrangiano es: 

\begin{equation}
	\Lcal_{\text{kin}} = -\frac{1}{4}B^{\mu \nu} B_{\mu \nu} - \frac{1}{4}W^{\mu \nu i} W_{\mu \nu}^i
\end{equation}
Queremos llegar a obtener las relaciones entre $W_{\mu}$, $W_{\mu \dagger}$, $Z_\mu$ y $A_\mu$. Tenemos que usar la que relaciona los elementos 1 y 2  de $W^i$ con los bosones $W$  y $W^\dagger$: 

\begin{equation}
	W_{\mu} = \frac{1}{\sqrt{2}} \pqty{W_{\mu}^1+iW_{\mu}^2} \qquad 
	W_{\mu}^\dagger = \frac{1}{\sqrt{2}} \pqty{W_{\mu}^1-iW_{\mu}^2}
\end{equation}
\begin{equation}
	W^1_{\mu} = \frac{1}{\sqrt{2}} (W_\mu + W_\mu^\dagger) \qquad 
	W^2_{\mu} = \frac{-i}{\sqrt{2}} (W_\mu - W_\mu^\dagger) \label{Ec:ej10-01}
\end{equation}
y la que relciona $B^\mu$ y $W^3$ con $Z$ y $A$:

\begin{equation}
\begin{pmatrix}
W_\mu^{3} \\[4pt]
B_\mu
\end{pmatrix}
=
\begin{pmatrix}
\cos\theta_W & \sin\theta_W \\[4pt]
-\sin\theta_W & \cos\theta_W
\end{pmatrix}
\begin{pmatrix}
Z_\mu \\[4pt]
A_\mu
\end{pmatrix}.
\end{equation}
\begin{equation}
	W^3_{\mu} =  \cos \theta_W W_\mu +\sin \theta_W W_\mu^\dagger \qquad 
	B^\mu_{\mu} = -\sin \theta_W W_\mu +\cos \theta_W W_\mu^\dagger
\end{equation}
Lo siguiente que tenemos que hacer es aplicar lo siguiente a nuestro $\Lcal_{\text{kin}}$:

\begin{equation}
\widetilde{W}_{\mu\nu} \equiv \frac{\sigma^i}{2} W^i_{\mu\nu}, 
\qquad 
W^i_{\mu\nu} = \partial_\mu W^i_\nu - \partial_\nu W^i_\mu - g_W \epsilon^{ijk} W^j_\mu W^k_\nu
\end{equation}
\begin{equation}
B_{\mu\nu} \equiv \partial_\mu B_\nu - \partial_\nu B_\mu
\end{equation}
Una vez tenemos eto, solo es sutituir. Lo que está claro qes que el término $B^{\mu \nu} B_{\mu \nu} $ no va a contribuir a nuestra triple o cuádruple interacción, ya que es evidente que solo contendrá términos de $\gamma$ y $Z$. Por otro lado, solo nos intersan los términos que incluyan al menos un $g_W \epsilon^{ijk} W_{\mu}^j W_{\mu}^k$, ya que otros términos no contendrán términos triples y cuádruples. Cuando este esté relacionado con otro $g_W \epsilon^{ijk} W_{\mu}^j W_{\mu}^k$ tendremos los vértices cuádruples, y cuando esté relacionado con $\partial_\mu W^i_\nu-\partial_\nu W_\mu^i$ los vértices triples. Así: 

\begin{equation}
	W^{\mu \nu i} W_{\mu \nu}^i = 
	W^{\mu \nu 1} W_{\mu \nu}^1 +
	W^{\mu \nu 2} W_{\mu \nu}^2 +
	W^{\mu \nu 3} W_{\mu \nu}^3
\end{equation}
Veamos término a término: 

\begin{itemize}
	\item El \textbf{primer término}: 
	\begin{align}  
		W^{\mu \nu 1} W_{\mu \nu}^1 = & \bqty{\pqty{\partial^\mu W^{\nu 1} - \partial^\nu W^{\mu 1}} - g_W \epsilon^{1ij} W^{\mu i} W^{\nu j}} \times \nonumber \\
		& \bqty{\pqty{\partial_\mu W^1_\nu - \partial_\nu W^1_\mu} - g_W 	\epsilon^{1kl} W^j_\mu W^k_\nu} \label{Ec:ej10-02}
	\end{align}
	Es evidente que el producto entre los doss primeros términos de cada corchete no generan un vértice triple o cuádruple, ya que $W^1$ solo va con los bosones $W$ y $W^\dagger$. Los vértices triples $\Ocal(3)$ se producirán cuando los primeros términos interaccionan con el término que va con la levi-civita. Cuando interaccionan los dos términos de la levi-civita se producen los términos que representan los vértices cuárticos $\Ocal(4)$. Así pues, tenemos que, los términos que representen vértices triples son (digamos que $\Ocal(n)$ significa ``términos que contienen vérices con $n$ partículas''):

	\begin{equation} \small
		\Ocal(3) = - g_W \bqty{\pqty{\partial^\mu W^{\nu 1} - \partial^\nu W^{\mu 1}}\pqty{\epsilon^{1kl} W^k_\mu W^l_\nu} + \pqty{ \epsilon^{1ij} W^{\mu i} W^{\nu j}} \pqty{\partial_\mu W^1_\nu - \partial_\nu W^1_\mu}}
	\end{equation}
	Fijemonos en particular en la primera parte: 
	\begin{align}
		\pqty{\partial^\mu W^{\nu 1} - \partial^\nu W^{\mu 1}}\pqty{\epsilon^{1kl} W^k_\mu W^l_\nu} &  =  
		\pqty{\partial^\mu W^{\nu 1} - \partial^\nu W^{\mu 1}}\pqty{ W^2_\mu W^3_\nu} \nonumber \\ & -
		\pqty{\partial^\mu W^{\nu 1} - \partial^\nu W^{\mu 1}}\pqty{ W^3_\mu W^2_\nu} 
	\end{align}
	Trivialmente, si intercambiamos los nombres $\mu \leftrightarrow \nu$ en el segundo término nos podemos dar cuenta que es igual al primero (hay dos signos $-$, uno procedente de la Levi-Civita y otra del intercambio de las posiciones de las parciales que se cancelan), por lo que: 

	\begin{equation}
		\pqty{\partial^\mu W^{\nu 1} - \partial^\nu W^{\mu 1}}\pqty{\epsilon^{1kl} W^k_\mu W^l_\nu} = 2
		\pqty{\partial^\mu W^{\nu 1} - \partial^\nu W^{\mu 1 }}\pqty{ W^2_\mu W^3_\nu}
	\end{equation}
	donde hemos obvidado los términos que no generan vértices dobles (contenidos en $\Ocal(2)$, y que contiene los términos que multiplica $W-W$ y $W^\dagger-W^\dagger$). Luego esto nos lleva a:

	\begin{align}
		\pqty{\partial^\mu W^{\nu 1} - \partial^\nu W^{\mu 1}}\pqty{ W^2_\mu W^3_\nu} & =  \frac{i}{2}\pqty{\partial^\mu W^{\nu } - \partial^\nu W^{\mu }}\pqty{ W^\dagger_\mu W^3_\nu} \nonumber \\
		& - \frac{i}{2}\pqty{\partial^\mu W^{\nu \dagger } - \partial^\nu W^{\mu \dagger}}\pqty{ W_\mu W^3_\nu} \nonumber \\
		& + \Ocal(2)
	\end{align}
	Donde aparece un signo menos y el número imaginario procedente de la definición de $W^2$ (\ref{Ec:ej10-01}). El divisor 2 procede de la multiplicación de las raíces cuadradas. Finalmente nos quedaría expandir $W_{\nu}^3$, de lo que podemos obtener: 


	\begin{align}
		\pqty{\partial^\mu W^{\nu 1} - \partial^\nu W^{\mu 1}}\pqty{ W^2_\mu W^3_\nu} & =  \frac{i\cos \theta_W}{2}\left[ \pqty{\partial^\mu W^{\nu } - \partial^\nu W^{\mu }}\pqty{ W^\dagger_\mu } \nonumber  \right. \\
		& \left. - \pqty{\partial^\mu W^{\nu \dagger } - \partial^\nu W^{\mu \dagger}}\pqty{ W_\mu } \right] Z_\nu \nonumber \\ & + 
		 \frac{i\sin \theta_W}{2}\left[ \pqty{\partial^\mu W^{\nu } - \partial^\nu W^{\mu }}\pqty{ W^\dagger_\mu } \nonumber  \right. \\
		& \left. - \pqty{\partial^\mu W^{\nu \dagger } - \partial^\nu W^{\mu \dagger}}\pqty{ W_\mu} \right] A_\nu \nonumber \\
		& + \Ocal(2) \label{Ec:ej10-03}
	\end{align}
	Sin embargo aun tenemos que fijarnos en la segunda parte del $\Ocal(3)$ (\ref{Ec:ej10-02}). Aunque no parezca trivial, realmente el problema es simétrico respecto lo que acabamos de decir. Así pues, tenemos que el resultado final será dos veces el anterior (\ref{Ec:ej10-03}), por lo que, para el primer término, los \textit{vértices triples} serán: 


	\begin{align}
		\Ocal(3)  = &  - {i2 g_W \cos \theta_W}\left[ \pqty{\partial^\mu W^{\nu } - \partial^\nu W^{\mu }}\pqty{ W^\dagger_\mu } \nonumber  \right. 
		\left. - \pqty{\partial^\mu W^{\nu \dagger } - \partial^\nu W^{\mu \dagger}}\pqty{ W_\mu } \right] Z_\nu \nonumber \\ & - 
		 {i2 g_W\sin \theta_W}\left[ \pqty{\partial^\mu W^{\nu } - \partial^\nu W^{\mu }}\pqty{ W^\dagger_\mu } \nonumber  \right. 
		 \left. - \pqty{\partial^\mu W^{\nu \dagger } - \partial^\nu W^{\mu \dagger}}\pqty{ W_\mu} \right] A_\nu  \label{Ec:ej10-03}
	\end{align}
	Ahora queda el término cuártico producido, que es, tal y como dijimos antes, el término producto de multiplicar los dos levi-civita. Así pues: 
	\begin{equation}
		\Ocal(4) = g_W^2 \pqty{\epsilon^{1ij} W^{\mu i} W^{\nu j}}\pqty{\epsilon^{1kl} W^k_\mu W^l_\nu} 
	\end{equation}
	que usando la propiedad 
	\begin{equation}
		\epsilon_{1ij} \epsilon^{1kl}
		= \delta_i^{\,k} \delta_j^{\,l}
		- \delta_i^{\,l} \delta_j^{\,k}
	\end{equation}
	nos lleva a: 

	\begin{equation}
		\Ocal(4) = g_W^2 \bqty{W^{\mu  2} W^{\nu 3} \pqty{W_{\mu}^2 W_{\nu}^3-W_{\mu}^3 W_{\nu}^2}+W^{\mu  3} W^{\nu 2} \pqty{W_{\mu}^3 W_{\nu}^2-W_{\mu}^2 W_{\nu}^3} }
	\end{equation}
	si en el segundo término hacemos el cambio trivial de $\mu \leftrightarrow \nu$, podemos ver que tenemos el primer término, tal que: 
	
	\begin{equation}
		\Ocal(4) = 2g_W^2 \bqty{W^{\mu  2} W^{\nu 3} \pqty{W_{\mu}^2 W_{\nu}^3-W_{\mu}^3 W_{\nu}^2}}
	\end{equation}
	Denemos que desarrollar tres términos para obtener esto en función de $W,W^\dagger,Z$ y $A$: 
	\begin{itemize}
		\item El primer término: 
		\begin{equation}
			W^{\mu  2} W_{\mu}^2 =  \frac{-1}{2} (W^{\mu \dagger} - W^{\mu})(W_{\mu}^{\dagger} - W_{\mu}) \Longrightarrow
  		\end{equation}
		\begin{equation}
			W^{\mu  2} W_{\mu}^2 =  \frac{-1}{2} \bqty{W^{\mu \dagger} W_{\mu}^{\dagger} + W^{\mu } W_{\mu} -  2(W^{\mu \dagger} W_{\mu})  }
		\end{equation}
		\item El segundo término: 
		\begin{equation}
			W^{\mu  2} W_{\mu}^3 =  \frac{1}{\sqrt{2}} (W^{\mu \dagger} - W^{\mu})( \cos \theta_W Z_{\mu} + \sin \theta_W A_\mu) \Longrightarrow
  		\end{equation}
		\begin{equation}
			W^{\mu  2} W_{\mu}^3 =  \frac{1}{\sqrt{2}} \bqty{\pqty{W^{\mu \dagger}-W^{\mu}} \cos \theta_W Z_\mu + \pqty{W^{\mu \dagger} - W^{\mu}} \sin \theta_W A_\mu  }
		\end{equation}
		\item El tercer término: 
		\begin{equation}
			W^{\mu  3} W_{\mu}^3 =   (\cos  \theta_W Z^{\mu } + \sin \theta_W A^{\mu})(\cos  \theta_W Z_{\mu } + \sin \theta_W A_{\mu})
  		\end{equation}
		\begin{equation}
			W^{\mu  3} W_{\mu}^3=  \cos^2  \theta_W  Z^\mu Z_\mu + 2 (\cos  \theta_W \sin \theta_W ) Z^\mu A_\mu + \sin^2 \theta_W  A^\mu A_\mu
		\end{equation}
	\end{itemize}
	Una vez tenemos esto queda claro que: 
	\begin{align}
		\Ocal(4)/2g_W^2 = &  - \cos^2 \theta_W \Bqty{\bqty{W^{\mu \dagger} W_{\mu}^{\dagger} + W^{\mu } W_{\mu} -  2(W^{\mu \dagger} W_{\mu})  }Z^\mu Z_\mu} \nonumber \\
		& -  \sin^2 \theta_W \Bqty{\bqty{W^{\mu \dagger} W_{\mu}^{\dagger} + W^{\mu } W_{\mu} -  2(W^{\mu \dagger} W_{\mu})  }A^\mu A_\mu} \nonumber \\ 
		& - 2 \sin \theta_W \cos \theta_W \Bqty{\bqty{W^{\mu \dagger} W_{\mu}^{\dagger} + W^{\mu } W_{\mu} -  2(W^{\mu \dagger} W_{\mu})  }Z^\mu A_\mu} 
	\end{align}

	\item Aún queda evaluar el \textbf{segundo término}, que será: 
	\begin{align}  
		W^{\mu \nu 2} W_{\mu \nu}^2 = & \bqty{\pqty{\partial^\mu W^{\nu 2} - \partial^\nu W^{\mu 2}} - g_W \epsilon^{2ij} W^{\mu i} W^{\nu j}} \times \nonumber \\
		& \bqty{\pqty{\partial_\mu W^2_\nu - \partial_\nu W^2_\mu} - g_W 	\epsilon^{2kl} W^j_\mu W^k_\nu} \label{Ec:ej10-02}
	\end{align}
	Trvialmente, los términos de \textit{vértices triples} generados serán exactamente igual a los anteriores, ya que estos contienen igual que antes un producto de $W^\dagger$ y $W$, por lo que da igual que en los términos de tensor de  campo esté $W^2$ y que con la levi-civita este $W^1$, el resultado será el mismo: 
	\begin{align}
		\Ocal(3)  = &  - {i2 g_W \cos \theta_W}\left[ \pqty{\partial^\mu W^{\nu } - \partial^\nu W^{\mu }}\pqty{ W^\dagger_\mu } \nonumber  \right. 
		\left. - \pqty{\partial^\mu W^{\nu \dagger } - \partial^\nu W^{\mu \dagger}}\pqty{ W_\mu } \right] Z_\nu \nonumber \\ & - 
		 {i2 g_W\sin \theta_W}\left[ \pqty{\partial^\mu W^{\nu } - \partial^\nu W^{\mu }}\pqty{ W^\dagger_\mu } \nonumber  \right. 
		 \left. - \pqty{\partial^\mu W^{\nu \dagger } - \partial^\nu W^{\mu \dagger}}\pqty{ W_\mu} \right] A_\nu 
	\end{align}
	No igual será el término que representa los vértices cuárticos, ya que es diferente $W^{\mu 2}W_{\mu}^2$ a $W^{\mu 1}W_{\mu}^1$. Así pues, veamos que: 
	\begin{itemize}
		\item El término
		\begin{equation}
			W^{\mu  1} W_{\mu}^1 =  \frac{1}{2} (W^{\mu \dagger} + W^{\mu})(W_{\mu}^{\dagger} + W_{\mu}) \Longrightarrow
  		\end{equation}
		\begin{equation}
			W^{\mu  1} W_{\mu}^1 =  \frac{1}{2} \bqty{W^{\mu \dagger} W_{\mu}^{\dagger} + W^{\mu } W_{\mu} +  2(W^{\mu \dagger} W_{\mu})  }
		\end{equation}
		\item El otro término: 
		\begin{equation}
			W^{\mu  1} W_{\mu}^3 =  \frac{1}{\sqrt{2}} (W^{\mu \dagger} + W^{\mu})( \cos \theta_W Z_{\mu} + \sin \theta_W A_\mu) \Longrightarrow
  		\end{equation}
		\begin{equation}
			W^{\mu  1} W_{\mu}^3 =  \frac{1}{\sqrt{2}} \bqty{\pqty{W^{\mu \dagger}+W^{\mu}} \cos \theta_W Z_\mu + \pqty{W^{\mu \dagger} + W^{\mu}} \sin \theta_W A_\mu  }
		\end{equation}
	\end{itemize}
	Lo que claramente nos lleva al siguiente resultado: 
	\begin{align}
		\Ocal(4)  /2g_W^2 & = \cos^2 \theta_W \Bqty{\bqty{W^{\mu \dagger} W_{\mu}^{\dagger} + W^{\mu } W_{\mu}  + 2(W^{\mu \dagger} W_{\mu})  }Z^\mu Z_\mu} \nonumber \\
		& + \sin^2 \theta_W \Bqty{\bqty{W^{\mu \dagger} W_{\mu}^{\dagger} + W^{\mu } W_{\mu} +  2(W^{\mu \dagger} W_{\mu})  }A^\mu A_\mu} \nonumber \\ 
		& + 2 \sin \theta_W \cos \theta_W \Bqty{\bqty{W^{\mu \dagger} W_{\mu}^{\dagger} + W^{\mu } W_{\mu} +  2(W^{\mu \dagger} W_{\mu})  }Z^\mu A_\mu} 
	\end{align}
	\item Lo último que tenemos que calcular es el \textbf{tercer término}, con un vértice triple representado por tensor de campo con $Z^\mu$ y $A^\mu$ y términos cuárticos $WWWW$. Como antes: 
	
	\begin{equation} \small
		\Ocal(3) = - g_W \bqty{\pqty{\partial^\mu W^{\nu 3} - \partial^\nu W^{\mu 3}}\pqty{\epsilon^{3kl} W^k_\mu W^l_\nu} + \pqty{ \epsilon^{3ij} W^{\mu i} W^{\nu j}} \pqty{\partial_\mu W^3_\nu - \partial_\nu W^3_\mu}}
	\end{equation}
	Dado que el proceso para obtener los resultados finales es igual al del primer término, nos tomamos algunas licencias y vamos directamente a ver que: 
	\begin{equation} \small
		\Ocal(3) = - i 4 g_W \Bqty{\cos\theta_W\bqty{\pqty{\partial^\mu Z^{\nu} - \partial^\nu Z^{\mu }} W_\mu W_\nu^\dagger} + \sin \theta_W\bqty{\pqty{\partial^\mu A^{\nu} - \partial^\nu A^{\mu}}W_\mu W_\nu^\dagger}}
	\end{equation}
	donde ha hemos descartado todos los términos de orden 2. Aparece un término 2 debido a que ahora debemos tener en cuenta el producto $W^1 W^2$ y no $W^1 W^3$ o $W^2  W^3$. El término cuártico 
	\begin{equation}
		\Ocal(4) = g_W^2 \bqty{W^{\mu  1} W^{\nu 2} \pqty{W_{\mu}^1 W_{\nu}^2-W_{\mu}^2 W_{\nu}^1}+W^{\mu 2} W^{\nu 1} \pqty{W_{\mu}^2 W_{\nu}^1-W_{\mu}^1 W_{\nu}^2} } \Longrightarrow
	\end{equation}
	\begin{align}
		\Ocal(4)  /2g_W^2 = \pqty{ (W^{\mu 1}W_{\mu}^1)(W^{\nu 2} W_\nu^2)-(W^{\mu 1}W^{2}_{\mu})(W^{\nu 1} W_\nu^2)}
	\end{align}
	ahora contiene el producto de los términos $W^{\mu 1}W_{\mu 1}$, $W^{\mu 2}W_{\mu}^2$ y  $W^{\mu 1}W_{\mu}^2$. Evaluamos este último: 

	\begin{itemize}
		\item El término
		\begin{equation}
			W^{\mu  1} W_{\mu}^2 =  \frac{i}{2} (W^{\mu \dagger} + W^{\mu})(W_{\mu}^{\dagger} - W_{\mu}) \Longrightarrow
  		\end{equation}
		\begin{equation}
			W^{\mu  1} W_{\mu}^2 =  \frac{i}{2} \bqty{W^{\mu \dagger} W_{\mu}^{\dagger} + W^{\mu } W_{\mu} -  2(W^{\mu \dagger} W_{\mu})  }
		\end{equation}
	\end{itemize}
	Por lo que 

	
	\begin{align}
		\Ocal(4) /2g_W^2 & =   \frac{-1}{4} \bqty{W^{\mu \dagger} W_{\mu}^{\dagger} + W^{\mu } W_{\mu} -  2(W^{\mu \dagger} W_{\mu})  }\bqty{W^{\mu \dagger} W_{\mu}^{\dagger} + W^{\mu } W_{\mu} +  2(W^{\mu \dagger} W_{\mu})  }  \nonumber \\
		& +  \frac{1}{4} \bqty{W^{\mu \dagger} W_{\mu}^{\dagger} + W^{\mu } W_{\mu} -  2(W^{\mu \dagger} W_{\mu})  }^2 
	\end{align}

	Finalmente
	\begin{align}
		\Ocal(4) /2g_W^2 = \pqty{ (W^{\mu}W_{\mu})(W^{\nu \dagger} W_\nu^\dagger)-(W^{\mu}W^{\dagger}_{\mu})(W^{\nu} W_\nu^\dagger)}
	\end{align}
\end{itemize}
Una vez tenemos estos resultados, podemos llegar a los resultados de los apuntes. Cabe destacar dos cosas. La primera es que hay términos de $\Ocal(3)$ qeu se anulan entre los generados por el primer término y el segundo, los cuales ya restamos. Lo segundo que hay que destacar es que hemos sustituido ya estas relaciones:
\begin{equation}
	g_W = \frac{e}{\sin \theta_W} \qquad \cot \theta_W = \frac{\cos \theta_W}{\sin \theta_W}
\end{equation}
Además recordemos que los valores de $\Lcal$ tienen un término 1/4 delante del término $W^{\mu \nu i}W_{\mu \nu}^i$. Ahora sí: 


\begin{align*}
\mathcal{L}_3
&= {i e \cot\theta_W}
  \Big\{
    (\partial^\mu W^\nu - \partial^\nu W^\mu) W_\mu^\dagger Z_\nu
    - (\partial^\mu W^{\nu\dagger} - \partial^\nu W^{\mu\dagger}) W_\mu Z_\nu
    + W_\mu W_\nu^\dagger (\partial^\mu Z^\nu - \partial^\nu Z^\mu)
  \Big\} \notag\\[4pt]
&\quad
  + { i e}
  \Big\{
    (\partial^\mu W^\nu - \partial^\nu W^\mu) W_\mu^\dagger A_\nu
    - (\partial^\mu W^{\nu\dagger} - \partial^\nu W^{\mu\dagger}) W_\mu A_\nu
    + W_\mu W_\nu^\dagger (\partial^\mu A^\nu - \partial^\nu A^\mu)
  \Big\}.
\end{align*}
\begin{align*}
\mathcal{L}_4
&= -\,\frac{e^2}{2 \sin^2 \theta_W}
  \Big\{
    (W_\mu^\dagger W^\mu)^2
    - W_\mu^\dagger W^{\mu\dagger} W_\nu W^\nu
  \Big\} \notag\\[4pt]
&\quad
  - {e^2 \cot^2 \theta_W}
  \Big\{
    W_\mu^\dagger W^\mu Z_\nu Z^\nu
    - W_\mu^\dagger Z_\mu W_\nu Z^\nu
  \Big\} \notag\\[4pt]
&\quad
  - { e^2 \cot \theta_W}
  \Big\{
    2 W_\mu^\dagger W^\mu Z_\nu A^\nu
    - W_\mu^\dagger Z_\mu W_\nu A^\nu
    - W_\mu^\dagger A_\mu W_\nu Z^\nu
  \Big\} \notag\\[4pt]
&\quad
  - { e^2}
  \Big\{
    W_\mu^\dagger W^\mu A_\nu A^\nu
    - W_\mu^\dagger A_\mu W_\nu A^\nu
  \Big\}.
\end{align*}





%%%%%%%%%%%%%%%%%%%%%%%%%%%%%%%%%%%%%%%%%%%%%%%%%%
%%%%%%%%%%%%%%%%%%%%%%%%%%%%%%%%%%%%%%%%%%%%%%%%%%
%%%%%%%%%%%%%%%%%%%%%%%%%%%%%%%%%%%%%%%%%%%%%%%%%%
%%%%%%%%%%%%%%%%%%%%%%%%%%%%%%%%%%%%%%%%%%%%%%%%%%

\begin{Ejercicio}{Expansión del potencial alrededor del vacío}\label{Ej:17}
Dados
\[
\mathcal{U}=\mu^2\phi^2+\lambda \phi^4,\qquad \mu^2=-\lambda v^2,\qquad 
\phi=\frac{1}{\sqrt{2}}\,(v+\eta+i\xi),
\]
verifique la ecuación (2.108), a saber:
\[
\mathcal{U}(\eta,\xi)= -\frac{1}{2}\lambda v^2\Big[(v+\eta)^2+\xi^2\Big]
+ \frac{1}{4}\lambda\Big[(v+\eta)^2+\xi^2\Big]^2
\]
\[
= -\frac{1}{4}\lambda v^4 + \lambda v^2\eta^2 + \lambda v\eta^3 + \frac{1}{4}\lambda\eta^4
+ \frac{1}{4}\lambda\xi^4 + \lambda v\eta\,\xi^2 + \frac{1}{2}\lambda \eta^2\xi^2
= \lambda v^2\eta^2 + \mathcal{U}_{\text{int}} - \frac{1}{4}\lambda v^4.
\]
\end{Ejercicio}

Este es aplicar directamente la ecuaciónon. El \textit{primer término} es 
\begin{equation}
- \mu^2 \phi^2 =  \lambda v^2 \phi^2
\end{equation}
Veamos que sustituyendo tenemos: 

\begin{equation}
	- \mu^2 \phi^2 = \frac{\lambda v^2}{2} \bqty{(v + \eta)^2 + \xi^2} =
	\frac{\lambda v^2}{2} \bqty{v^2 + 2v\eta + \eta^2 + \xi^2}  
\end{equation}
mientra que el \textit{segundo término} es 
\begin{equation}
	-\lambda \phi^4 
\end{equation}
Veamos que sustituyendo tenemos: 

\begin{equation}
	-\lambda \phi^4  = -\frac{\lambda}{4} \bqty{(v + \eta)^2 + \xi^2}^2 = -
		\frac{\lambda}{4} \bqty{v^4 + 8v^2\eta^2 + \eta^4 + \xi^4 + 2 v^2 \xi^2 + 4 v \eta \xi^2 + 2 \eta^2 \xi^2}  
\end{equation}
Sumando el segundo y tercer término llegamos a: 

\begin{equation}
	- \Ucal = - \mu^2 \phi^2 -\lambda \phi^4  = \lambda\bqty{\frac{v^4}{4}+ v^3 \eta - {v^2 \eta^2} -   \frac{1}{4}  \eta^4 -  \frac{1}{4} \xi^4 -   v \eta \xi^2 +  \frac{1}{2}  \eta^2 \xi^2 }
\end{equation}
o lo que es lo mismo: 

\begin{equation}
	\Ucal = -\frac{1}{4}\lambda v^4 + \lambda v^2\eta^2 + \lambda v\eta^3 + \frac{1}{4}\lambda\eta^4
+ \frac{1}{4}\lambda\xi^4 + \lambda v\eta\,\xi^2 + \frac{1}{2}\lambda \eta^2\xi^2
\end{equation}
tal y como queríamos demostrar, tal que 

\begin{equation}
	\Ucal_{\text{int}} = + \lambda v\eta^3 + \frac{1}{4}\lambda\eta^4
+ \frac{1}{4}\lambda\xi^4 + \lambda v\eta\,\xi^2 + \frac{1}{2}\lambda \eta^2\xi^2
\end{equation}
		
%%%%%%%%%%%%%%%%%%%%%%%%%%%%%%%%%%%%%%%%%%%%%%%%%%
%%%%%%%%%%%%%%%%%%%%%%%%%%%%%%%%%%%%%%%%%%%%%%%%%%
%%%%%%%%%%%%%%%%%%%%%%%%%%%%%%%%%%%%%%%%%%%%%%%%%%
%%%%%%%%%%%%%%%%%%%%%%%%%%%%%%%%%%%%%%%%%%%%%%%%%%

\begin{Ejercicio}{Invariancia gauge de un escalar complejo}\label{Ej:18}
Verifique que el lagrangiano
\[
\mathcal{L}=(D_\mu\phi)^\ast(D^\mu\phi)-\mathcal{U}(\phi^2),
\]
donde $\phi$ es un campo escalar complejo, es invariante bajo transformaciones de norma (gauge) del tipo
\[
\phi \;\to\; \phi' = e^{\,i g\,\alpha(x)}\,\phi,
\]
siempre que el campo gauge se transforme como
\[
B_\mu \;\to\; B'_\mu = B_\mu - \partial_\mu \alpha(x),
\]
con el derivado covariante $D_\mu=\partial_\mu + i g B_\mu$.
\end{Ejercicio}
Si aplicamos la transformación gauge a nuestro complejo es: 
\begin{equation}
	\Lcal' = (D'_\mu \phi')^* (D^{\prime \mu} \phi') - \Ucal (\phi^2) \label{Ec:ej12-02}
\end{equation}
Tenemos que: 
\begin{equation}
	D_\mu' \phi' = (\partial_\mu + i g  B_\mu') e^{ig\alpha(x)} \phi =  (\partial_\mu - ig\partial_\mu(\alpha(x))+ i g  B_\mu) e^{ig\alpha(x)} \phi \label{Ec:ej12-01}
\end{equation}
Por otro lado, tenemos que:
\begin{equation}
	\partial_\mu e^{ig\alpha(x)} \phi = e^{ig\alpha(x)} \pqty{\partial_\mu\phi + ig \partial_\mu(\alpha(x))}
\end{equation}
Sustituyendo esto en (\ref{Ec:ej12-01}) tenemos que: 

\begin{equation}
	D_\mu' \phi' =e^{ig\alpha(x)}  \pqty{\partial_\mu + ig\partial_\mu(\alpha(x)) - ig\partial_\mu(\alpha(x)) + i gB_\mu  } \phi = e^{ig\alpha(x)} D_\mu \phi
\end{equation}
y dado que
\begin{equation}
	(D_\mu' \phi')^* = (e^{ig\alpha(x)}D_\mu \phi)^* = e^{-ig\alpha(x)} (D_\mu \phi)^*
\end{equation}
si lo sustituimos en (\ref{Ec:ej12-02}): 

\begin{equation}
	\Lcal' = e^{-ig\alpha(x)} e^{ig\alpha(x)} (D_\mu\phi)^*(D_\mu \phi) + \Ucal(\phi^2)= \Lcal 
\end{equation}
y por tanto es  $\Lcal$ \textit{es invariante bajo transformaciones gauge}.

%%%%%%%%%%%%%%%%%%%%%%%%%%%%%%%%%%%%%%%%%%%%%%%%%%
%%%%%%%%%%%%%%%%%%%%%%%%%%%%%%%%%%%%%%%%%%%%%%%%%%
%%%%%%%%%%%%%%%%%%%%%%%%%%%%%%%%%%%%%%%%%%%%%%%%%%
%%%%%%%%%%%%%%%%%%%%%%%%%%%%%%%%%%%%%%%%%%%%%%%%%%
\begin{Ejercicio}{Lagrangiano con campo escalar complejo}\label{Ej:19}
Considere el siguiente lagrangiano para un campo escalar complejo:
\[
\mathcal{L} = -\frac{1}{4}F_{\mu\nu}F^{\mu\nu}
+ (\partial_\mu \phi)^\ast (\partial^\mu \phi)
- \mu^2 \phi^2 - \lambda \phi^4
- i g B_\mu \phi^\ast (\partial^\mu \phi)
+ i g (\partial_\mu \phi^\ast) B^\mu \phi
+ g^2 B_\mu B^\mu \phi^\ast \phi,
\]
donde $B_\mu$ es un campo gauge y 
\[
F^{\mu\nu} = \partial^\mu B^\nu - \partial^\nu B^\mu.
\]

\medskip
Haga la sustitución 
\[
\phi(x) = \frac{1}{\sqrt{2}}\big(v + \eta(x) + i\xi(x)\big),
\]
y verifique que se obtiene:
\[
\mathcal{L} =
\frac{1}{2}(\partial_\mu \eta)(\partial^\mu \eta)
- \lambda v^2 \eta^2
+ \frac{1}{2}(\partial_\mu \xi)(\partial^\mu \xi)
- \frac{1}{4}F_{\mu\nu}F^{\mu\nu}
+ \frac{1}{2}g^2 v^2 B_\mu B^\mu
- \mathcal{U}_{\text{int}}
+ g v B_\mu (\partial^\mu \xi),
\]
junto con la expresión explícita para $\mathcal{U}_{\text{int}}$.
\end{Ejercicio}

Basta con sustituir en el lagrangiano 

\[
\mathcal{L} = -\frac{1}{4}F_{\mu\nu}F^{\mu\nu}
+ (\partial_\mu \phi)^\ast (\partial^\mu \phi)
- \mu^2 \phi^2 - \lambda \phi^4
- i g B_\mu \phi^\ast (\partial^\mu \phi)
+ i g (\partial_\mu \phi^\ast) B^\mu \phi
+ g^2 B_\mu B^\mu \phi^\ast \phi,
\]
el término 

\[
\phi(x) = \frac{1}{\sqrt{2}}\big(v + \eta(x) + i\xi(x)\big),
\]
Iremos término por término. 

\begin{itemize}
	\item El \textit{primer término} es 
	\begin{equation}
		(\partial_\mu \phi)^* (\partial^\mu \phi)
	\end{equation}
	Veamos que sustituyendo tenemos: 

	\begin{equation}
		\frac{1}{2}(\partial_\mu (v + \eta(x) + i\xi(x)))^* (\partial^\mu (v + \eta(x) + i\xi(x))) = \frac{1}{2}  \bqty{(\partial_\mu \eta)(\partial^\mu \eta) + (\partial_\mu \xi)(\partial^\mu \xi) } 
	\end{equation}
	\item El \textit{segundo término} es 
	\begin{equation}
		- \mu^2 \phi^2 =  \lambda v^2 \phi^2
	\end{equation}
	Veamos que sustituyendo tenemos: 

	\begin{equation}
		- \mu^2 \phi^2 = \frac{\lambda v^2}{2} \bqty{(v + \eta)^2 + \xi^2} =
		\frac{\lambda v^2}{2} \bqty{v^2 + 2v\eta + \eta^2 + \xi^2}  
	\end{equation}
	mientra que el \textit{tercer término} es 
	\begin{equation}
		-\lambda \phi^4 
	\end{equation}
	Veamos que sustituyendo tenemos: 

	\begin{equation}
		-\lambda \phi^4  = -\frac{\lambda}{4} \bqty{(v + \eta)^2 + \xi^2}^2 = -
		\frac{\lambda}{4} \bqty{v^4 + 8v^2\eta^2 + \eta^4 + \xi^4 + 2 v^2 \xi^2 + 4 v \eta \xi^2 + 2 \eta^2 \xi^2}  
	\end{equation}
	Sumando el segundo y tercer término llegamos a: 

	\begin{equation}
		- \mu^2 \phi^2 -\lambda \phi^4  = \lambda\bqty{\frac{v^4}{2}+ v^3 \eta - {v^2 \eta^2} - \eta^4 - \xi^4 - v \eta \xi^2 + 2 \eta^2 \xi^2 }
	\end{equation}
	Usando que 

	\begin{equation}
		\Ucal^1_{\text{int}} =  \lambda v\eta^3 + \frac{1}{4}\lambda\eta^4
	+ \frac{1}{4}\lambda\xi^4 + \lambda v\eta\,\xi^2 + \frac{1}{2}\lambda \eta^2\xi^2 + \frac{1}{4}\lambda v^4
	\end{equation}
	que es diferente al del ejercicio 11. Usando esto, tenemos que el segundo y tercer término juntos: 
	\begin{equation}
		- \mu^2 \phi^2 -\lambda \phi^4   = - \lambda v^2 \eta^2  - \Ucal_{\text{int}}
	\end{equation}
	\textit{aún faltan términso por añadir a esta $\Ucal_{\text{int}}$, por eso lo denotamos por con un índice superior}. 
	\item El \textit{cuarto término} viene dado por: 
	\begin{equation}
		- i g B_\mu \phi^* \partial^\mu \phi = - \frac{i g}{2} B_\mu (v + \eta(x) - i \xi(x)) \partial^{\mu}(\eta(x) + i \xi(x))
	\end{equation}
	y el \textit{quinto término} por 
	\begin{equation}
		i g B^\mu \phi \partial_\mu \phi^* = \frac{i g}{2} B^\mu  (v + \eta(x) + i \xi(x)) \partial_{\mu}(\eta(x) - i \xi(x)) 
	\end{equation}
	que al sumarse: 
	\begin{equation}
		- i g B_\mu \phi^* \partial^\mu \phi + i g B^\mu \phi \partial_\mu \phi^* = - i g  B^\mu \eta \partial_\mu \xi + i g B^\mu \xi \partial_\mu \eta+ i g B^\mu v \partial_\mu \eta
	\end{equation}
	tal que el término de interacción $\Ucal_{\text{int}}$ generado por estos dos términos: 
	\begin{equation}
		\Ucal^2_{\text{int}} =  - i g  B^\mu \eta \partial_\mu \xi + i g B^\mu \xi \partial_\mu \eta
	\end{equation}
	denotando el superíndice 2 que es un $\Ucal_{\text{int}}$ no es este término, si no la suma de otros.

	\item El \textit{sexto término}
	\begin{equation}
		g^2 B_\mu B^\mu \phi^* \phi  =  
		\frac{1}{2}g^2 v^2 B_\mu B^\mu  +
		\frac{1}{2}g^2  \eta^2 B_\mu B^\mu +
		\frac{1}{2}g^2 v \eta  B_\mu B^\mu +
		\frac{1}{2}g^2  \xi^2 B_\mu B^\mu 
	\end{equation}
	nos lleva a un término de interacción
	\begin{equation}
		\Ucal_{\text{int}}^3 = \frac{1}{2}g^2  \eta^2 B_\mu B^\mu +
		\frac{1}{2}g^2 v \eta  B_\mu B^\mu +
		\frac{1}{2}g^2  \xi^2 B_\mu B^\mu 
	\end{equation}
\end{itemize}
Finalmente, la suma nos lleva a: 

\begin{equation}
\mathcal{L} = -\frac{1}{4}F_{\mu\nu}F^{\mu\nu}
+ (\partial_\mu \phi)^\ast (\partial^\mu \phi)
- \mu^2 \phi^2 - \lambda \phi^4
- i g B_\mu \phi^\ast (\partial^\mu \phi)
+ i g (\partial_\mu \phi^\ast) B^\mu \phi
+ g^2 B_\mu B^\mu \phi^\ast \phi,
\end{equation}
donde 

\begin{align} 
	\Ucal_{\text{int}} & = \Ucal^1 + \Ucal^2 + \Ucal^3 \nonumber \\
	& =  \lambda v\eta^3 + \frac{1}{4}\lambda\eta^4
	+ \frac{1}{4}\lambda\xi^4 + \lambda v\eta\,\xi^2 + \frac{1}{2}\lambda \eta^2\xi^2 + \frac{1}{4}\lambda v^4 + - i g  B^\mu \eta \partial_\mu \xi + i g B^\mu \xi \partial_\mu \eta  \nonumber \\ 
	& + \frac{1}{2}g^2  \eta^2 B_\mu B^\mu +
		\frac{1}{2}g^2 v \eta  B_\mu B^\mu +
		\frac{1}{2}g^2  \xi^2 B_\mu B^\mu    
\end{align}
conteniendo todos los vérices triples y cuárticos. 


%%%%%%%%%%%%%%%%%%%%%%%%%%%%%%%%%%%%%%%%%%%%%%%%%%
%%%%%%%%%%%%%%%%%%%%%%%%%%%%%%%%%%%%%%%%%%%%%%%%%%
%%%%%%%%%%%%%%%%%%%%%%%%%%%%%%%%%%%%%%%%%%%%%%%%%%
%%%%%%%%%%%%%%%%%%%%%%%%%%%%%%%%%%%%%%%%%%%%%%%%%%
\begin{Ejercicio}{Término del lagrangiano del Modelo Estándar con derivadas covariantes}\label{Ej:20}
Verifique que el término del lagrangiano del Modelo Estándar que contiene las derivadas covariantes del campo de Higgs puede escribirse como:
\[
(D_\mu \phi)^\dagger (D^\mu \phi)
= \frac{1}{2}(\partial_\mu h)(\partial^\mu h)
+ \frac{1}{8} g_W^2 (W_\mu^{(1)} + iW_\mu^{(2)})(W^{(1)\mu} - iW^{(2)\mu})(v+h)^2
\]
\[
+ \frac{1}{8}(g_W W_\mu^{(3)} - g' B_\mu)(g_W W^{(3)\mu} - g' B^\mu)(v+h)^2.
\]
\end{Ejercicio}

%%%%%%%%%%%%%%%%%%%%%%%%%%%%%%%%%%%%%%%%%%%%%%%%%%
%%%%%%%%%%%%%%%%%%%%%%%%%%%%%%%%%%%%%%%%%%%%%%%%%%
%%%%%%%%%%%%%%%%%%%%%%%%%%%%%%%%%%%%%%%%%%%%%%%%%%
%%%%%%%%%%%%%%%%%%%%%%%%%%%%%%%%%%%%%%%%%%%%%%%%%%
\begin{Ejercicio}{Transformación del doblete de Higgs bajo $SU(2)$}\label{Ej:21}
Demuestre que el campo $\phi_c$ transforma bajo $SU(2)$ de la misma forma que $\phi$ (el doblete de Higgs), donde
\[ \small
\phi =
\begin{pmatrix}
\phi^+ \\[4pt]
\phi^0
\end{pmatrix},
\qquad
\phi_c = -i\sigma_2 \phi^\ast
= -\frac{i}{\sqrt{2}}
\begin{pmatrix}
0 & -i\\[4pt]
i & 0
\end{pmatrix}
\begin{pmatrix}
\phi_1 - i\phi_2\\[4pt]
\phi_3 - i\phi_4
\end{pmatrix}
=
\frac{1}{\sqrt{2}}
\begin{pmatrix}
-\phi_3 + i\phi_4\\[4pt]
\phi_1 - i\phi_2
\end{pmatrix}
=
\begin{pmatrix}
-\phi^{0\ast}\\[4pt]
\phi^{+\ast}
\end{pmatrix}.
\]
\end{Ejercicio}
