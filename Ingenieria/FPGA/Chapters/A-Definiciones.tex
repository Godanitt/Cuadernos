
\chapter{Definciones básicas}

\section{Acarreo} \label{Sec:A-Acarreo}

En general, el acarreo está ligado a todo circuito o algoritmo que requiera propagación de valores que exceden la capacidad de una posición binaria, por lo que es fundamental para la aritmética digital y el diseño de ALUs, sumadores, procesadores y sistemas embebidos.

Los \textbf{bits de acarreo} son los bits que se generan y se propagan entre etapas sucesivas cuando se suman números binarios. Se llaman así porque ``transportan'' el exceso cuando la suma de dos bits en una posición supera la capacidad de esa posición (es decir, cuando la suma da un resultado mayor que 1 en binario).

Por ejemplo, al sumar dos bits \(A_i\) y \(B_i\) en la posición \(i\) junto con un acarreo de entrada \(C_i\), el resultado del bit de suma es

\[
S_i = A_i \oplus B_i \oplus C_i,
\]

y el bit de acarreo de salida (hacia la siguiente posición más significativa) es

\[
C_{i+1} = (A_i \cdot B_i) + (A_i \cdot C_i) + (B_i \cdot C_i),
\]

donde \(\oplus\) representa la operación XOR, \(\cdot\) es la operación AND, y \(+\) es la operación OR en lógica booleana. Cuando se suman \(1 + 1\) en binario, se obtiene \(10\): el 0 es el bit de suma \(S_i\) y el 1 es el bit de acarreo \(C_{i+1}\) que debe sumarse a la siguiente posición más significativa. Los bits de acarreo aseguran que la suma sea correcta, de forma similar a como se ``lleva'' una unidad a la siguiente columna al sumar en el sistema decimal, por ejemplo al sumar 9 + 5. En un sumador binario de varios bits, estos bits de acarreo forman la \textbf{cadena de acarreo}, ya que cada etapa depende del acarreo generado en la etapa anterior.

La \textbf{cadena de acarreo} (en inglés, carry chain) es un elemento fundamental en los sumadores digitales. Es la secuencia de propagación de los bits de acarreo de una etapa de suma a la siguiente cuando se suman dos números binarios. En un sumador, cada bit de la suma depende de los bits de entrada y del acarreo que llega de la posición anterior: si el bit menos significativo genera un acarreo, este debe sumarse con el siguiente bit más significativo, y así sucesivamente.

En un sumador de varios bits (por ejemplo, un sumador de 4 bits), este proceso crea una cadena de acarreos que se propaga desde el bit menos significativo hasta el más significativo. Este retardo de propagación del acarreo es un factor crítico que limita la velocidad de los sumadores implementados como sumadores en serie, porque el acarreo debe recorrer cada bit antes de que el resultado final sea estable.

Por eso se diseñan sumadores más rápidos como el sumador de acarreo anticipado (carry lookahead adder) o el sumador de bloques, que intentan predecir los acarreos sin esperar a que la cadena de acarreo se propague completamente, reduciendo así el retardo total.

El concepto aparece de forma central en la aritmética digital y es clave en el diseño de ALUs (unidades aritmético-lógicas) y procesadores, porque limita el tiempo de operación de sumas, restas y otras operaciones que dependen del acarreo.