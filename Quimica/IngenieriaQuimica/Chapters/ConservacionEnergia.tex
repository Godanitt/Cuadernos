
\section{Tem 3. Principios de conservación II: Balances de energía}

Como dice la ley de conservación de la energía \textit{\textbf{La energía ni se crea ni se destruye, sólo se transforma}}. En esta afirmación queda excluida la energía nuclear que se genera por desintegración de materia ($E=m\cdotc^2$).\\

El principio de conservación de la energía constituye la \textit{Primera ley de la Termodinámica} ($\Delta U= Q+W$). Aplicando a un sistema material sometido a transformaciones físicas y químicas que transcurren en régimen no estacionario el balance de energía completa el balance de materia:

\begin{equation}
    [E]-[S]+[G]=[A]
\end{equation}

Donde en el caso de la energía cada termino representa:

\begin{itemize}
    \item $[E]$: Cantidad de energía recibida del exterior por el sistema.
    \item $[S]$: Cantidad de energía liberada al exterior por el sistema.
    \item $[G]$: Cantidad de energía generada en el interior del sistema.
    \item $[A]$: Cantidad de energía acumulada en el sistema. 
\end{itemize}

Trataremos diferentes tipos de sistemas:

\begin{enumerate}
    \item \textbf{Sistema abierto}: en estos sistemas hablaremos de procesos continuos. El sistema permite la entrada y la salida de materia y energía ( $[E]\neq 0$ y $[S]\neq 0$.

    \item \textbf{Sistema cerrado:} en estos sistemas hablaremos de procesos discontinuos. Se trata de un sistema que no permite la entrada y salida de materia pero si de energía.
    \item \textbf{Sistema adiabático:} sistema que no intercambia materia y energía con los alrededores. ($[E]=0$ y$[S]=0$).

\end{enumerate}

También tenemos que tener claro las diferentes formas de expresión de la energía.\\

Definimos \textbf{Energía mecánica} como la suma de la enregía cinética y la energía potencial. Esta relacionada con la energía acumulada por un cuerpo en movimiento y la energía asociada a la localización de un cuerpo dontro de un campo de fuerza.\\

La \textbf{Energía electromagnética} incluye las diferentes manifestaciones de la energía debida a los campos electrostático y magnético y a la corriente electrica.\\

Otro tipo de energía es la \textbf{Energía interna o térmica} que engloba la energía potencial y cinética internas de las particulas elementales de los cuerpos (moléculas, atomos), manifestándose al exterior en forma de temperatura.\\

También trataremos la \textbf{Energía química} que es la energía de los compuestos relacionada con las reacciones químicas.\\

La \textbf{Energía metabólica} es la energía generada por los organismos vivos por oxidaión de los alimentos digeridos.\\

Por último la \textbf{Energía nuclear} es un caso particular de energía química, que está relacionada con las reacciones de fusión y fisión nuclear.\\

Las expresiones matemáticas de cada una de estas formas de energía es diferente,  debiendo hacerse en función de las variables susceptibles de medida experimental.\\

\textbf{Energía potencial}\\


Esta energía es función de la posición en el campo gravitatorio.

\begin{equation}
    E_p=m\cdot g\cdot z
\end{equation}

Donde $m$ es la masa del cuerpo ($kg$), $z$ la altura respecto al nivel de referencia al que se le asigna una energía potencial nula y $g$ la aceleración de la gravedad.\\

\textbf{Energía cinética}\\

Esta energía es función de la velocidad:

\begin{equation}
    E_c=K=\frac{1}{2}\cdot m \cdot v^2
\end{equation}

Siendo $m$ la masa del cuerpo en $kg$ y $v$ la velocidad del fluido el la sección transversal.\\


Siendo la \textbf{Energía mecánica} la suma de ambas energías:

\begin{equation}
    E_{mecánica}= E_p+E_c
\end{equation}

\textbf{Energía interna}\\


El contenido de energía interna de un cuerpo se corresponde con la suma de las energías potencial y cinética asocidas a las partículas elementales constituidas por el mismo, moléculas, átomos y partículas subatómicas. Se manifiesta a partir de la Temperatura. \\

Es función de la temperatura, de la concentración y del estado de agregación y prácticamente independiente de la presión  (no se puede calcular de forma absoluta, sólo diferencias).\\
\begin{equation}
    U=m\cdot\int_{T_{ref}}^{T} C_p\cdot dT=m\cdot{C_p}\cdot(T-T_{ref})
\end{equation}


Para un cuerpo de masa $m $ ($kg$) que se encuentra a una temperatura $T$, $C_p$ es el calor especifico a volumen constante ($J/kg\cdot K$) y $T_{ref}$ es la temperatura de referencia considerada, habiendose asignado arbitrariamente un valor no nulo de la energóa interna en el estado de referencia considerado (i.e. $U_{ref}=0$).\\

Es una función de estado (variable de estado).\\

\textbf{Calor y Trabajo}\\

La transferencia de energía entre un sistema cerrado y sus alrededores puede darse en forme de calor o de trabajo (son formas de energía de transito: nunca se almacenan). El aporte de calor y/o trabajo son formas de aumentar su contenido en energía en alguna de sus formas anteriores (potencial, cinética o interna).\\

El \textbf{calor (Q)} es la energía en tránsito debida a una diferencia de temperaturas entre un sistema y sus alrededores. Siempre se transmite del foco caliente al frio, sienso el gradiente de temperaturas la fuerza impulsora. Se representa en la ecuación \ref{calor} donde $\dot{U}$ es el coeficiente empírico ($W\cdot m^{-2}\cdot K^{-1}$.\\

\begin{equation}
    Q=\dot{U} \cdot A\cdot \Delta T
    \label{calor}
\end{equation}

El \textbf{trabajo (W)} es la energía en tránsito debida a la acción de las fuerzas mecánicas que vencen una resistencia al recorrer un espacio, energía que fluye en respuesta a la aplicación de una fuerza. Sólo tiene sentido cuando se produce o se consume en un sistema determinado, no se puede hablar de contenido en trabajo. El trabajo se puede calcular como se muestra en la ecuación \ref{trabajo} (el signo es negativo por que es el trabajo realizado por el sistema). El trabajo realizado por el sistema será negativo ($W<0$) y el trabajo realizado sobre el sistema positivo ($W>0$).

\begin{equation}
    W=-\int_{v_1}^{v_2}P\cdot dV
    \label{trabajo}
\end{equation}

\textbf{La entalpía (H)}\\

La entalpía es la combinación de dos variables que aparecen con mucha frecuencia en los balances de energía:

\begin{equation}
    H=U+P\cdot V 
\end{equation}

Tambien se puede calcular la entalpía específica  $J/kg$.

\begin{equation}
    h=u+P\cdot\theta= u+ \frac{P}{\rho}
\end{equation}

Donde $P$ es la presión, $V$ es el volumen ($m^3$), $\theta$ es el volumen específico ($m^3/kg$). \\

La entalpía es una función de estado (variable de estado), resultado de la combinación de la energía interna con una parte del trabajo que genera el sistema. $H$ es función de la temperatura, composición y estado de agregación.\\

La entalpía no tiene un valor absoluto, al igual que la energía interna, solo se pueden evaluar cambios de entalpía ($\Delta H$) y a menudo se utiliza un conjunto de condiciones de referencia.

\subsection{Balances macroscópicos de energía}

Como ya se mostro anteriormente la ecuación de balance de energía sigue la siguiente ecuación:

\begin{equation}
    [E]-[S]+[G]=[A]
\end{equation}

Donde en el caso de la energía cada termino representa:

\begin{itemize}
    \item $[E]$: Cantidad de energía recibida del exterior por el sistema.
    \item $[S]$: Cantidad de energía liberada al exterior por el sistema.
    \item $[G]$: Cantidad de energía generada en el interior del sistema.
    \item $[A]$: Cantidad de energía acumulada en el sistema. 
\end{itemize}


Los dos primeros términos representan la energía intercambiada con los alrededores. Podemos tener intercambio de trabajo debido a fuerzas mecánicas, transmisión de calor debido a un flujo de energía de el foco caliente al frio o transferencia de materia ya que la materia lleva asociada una cantidad de energía (potencial, cinética o interna).\\

La cantidad de energía generado por el sistema (tercer término de la ecuación) se debe a las reacciones químicas que tienen lugar. La energía acumulada la expresaremos como $dE_{total}/dt$.\\

\textbf{Sistema cerrado}\\

Recordemos que un sistema cerrado es un sistema discontinuo (por lotes) en el cual si hay transferencia de energía con el exterior pero \textbf{no hay intercambio de materia}. La energía dentro del sistema se divide en tres categorias $E_c$, $E_p$ y $U$. La energía transportada a través de la frontera se puede hacer en forma de trabajo o calor. De modo que, la variación de energía del sistema:

\begin{equation}
    \Delta E=E_{t2}-E_{t1}=\Delta U +\Delta E_c + \Delta E_p =Q+W
\end{equation}\\


\textbf{Sistema abierto}\\

Un sistema abierto es un sistema continuo por lo que existe una transferencia de materia. Recordemos la expresión del balance de materia obtenida en el Tema 2:

\begin{equation}
    \frac{d(\pi_m V)}{dt}=- \sum_i \phi_i \Vec{S}_i+(\pi_1\Q_{v,1}-\pi_2Q_{V,2})+G_mV
\end{equation}

La energía asociada a la materia recordemos que es la energía interna $U$, la energía cinética $E_c$ y la energía potencial $E_p$. De modo que la energía del sistema.

\begin{equation}
    E=U+E_p+E_c+Q+W+ \text{Fuerzas de presión}
\end{equation}

De modo que el balance energético será de la siguiente forma:

\begin{equation}
    \frac{d(U_T+E_{P,T}+E_{C,T})}{dt}=\rho_1\cdot Q_{v1}\cdot (U+E_p+E_c)_1-\rho_2\cdot Q_{v2}\cdot (U+E_p+E_c)_2+Q+W+[P_1Q_{v1}-P_2\cdot Q_{v2}]
\end{equation}

En un sistema en estado estacionario la acumulación de energía se anula de modo que $dE/dt=0$. Además tenemos que tener en cuenta que $m=\rho\cdot Q_v$. De moso que para un sistema en estado estacionario podemos expresar la ecuación del balance de energía como:
\begin{equation}
    0=m\cdot (U+E_p+E_c)_1-m\cdot (U+E_p+E_c)_2+Q+W+m\left[\frac{P_1}{\rho_1}-\frac{P_2}{\rho_2}\right]
    \label{eq: conservacion e}
\end{equation}

Si sustituimos en esta ecuación las expresiones de la energía cinética ($E_c=1/2\cdot m\cdotv^2$) y de la energía potencial ($E_p=m \cdot g \cdot z$) llegamos a :

\begin{equation}
    g\cdot(z_1-z_2)+(U_1-U_2)+\frac{1}{2}\cdot(v_1^2-v_2^2)+\dot{Q}+\dot{W}+\left(\frac{P_1}{\rho_1}-\frac{P_2}{\rho_2}\right)=0
\end{equation}
Además sabemos que la entalpía se puede expresar como $H=U+P/\rho$ por lo que sustituyendo llegamos a la \textbf{ecuación de conservación de energía en régimen estacionario}.

\begin{equation}
    g\cdot(z_1-z_2)+(H_1-H_2)+\frac{1}{2}\cdot(v_1^2-v_2^2)+\dot{Q}+\dot{W}=0
\end{equation}

\subsection{Balances entálpicos}
Partu¡iendo de la ecuación de valance de enegía expresada en vatios (W):

\begin{equation}
            m\cdot g\cdot(z_1-z_2)+\cdot(H_1-H_2)+\frac{1}{2}m\cdot(v_1^2-v_2^2)+Q+W=0
\end{equation}

En muchos de los procesos de la industría química en régimen estacionario las variaciones de energía potencial ($\Delta E_p$) y cinética ($\Delta E_c$) son despreciables. Además si no existe una máquina el trabajo también es nulo por lo que llegamos a la ecuación de \textbf{Balance entálpico}.

\begin{equation}
    \delta H =(H_2-H_1)= \dot{Q}
\end{equation}

Dado que no se pueden calcular los valores absolutos de entalpías, para aplicar la ecuación anterior es necesario establecer un \textbf{estado de referencia}. Este estado de referencia conlleva el cálculo de entalpías de todos los componentes de la $T$ y $P$ que se encuentre, tomando el estado de referencia. \\

La entalpía de una sustancia depende de su composición química, estado de agregación y temperatura y es prácticamente independiente de la presión (para gases ideales es rigurosamente independiente de $P$). De acuerdo con la termodinámica, \textbf{la entalpía se considera variable de estado}, lo que quiere decir que una transformación su variación depende del estado final e inicial y no del camino recorriso.\\

Suponiendo como despreciable la variación de la entalpía especifica con la presión, para un \textbf{compuesto puro}, ésta se podróa calcular respecto a una temperatura de referencia ($T_{ref}$) mediante la ecuación:

\begin{equation}
    H_T=H_{ref}+\int_{T_{ref}}^T C_pdT=H_{T_{ref}}+\bar{C_p} \cdot (T-T_{ref}) \text{   (J/kg)}
\end{equation}

Ecuación que es rigurosamente cierta para transformaciones a $P $ constante y cualquier tranformación de un gas ideal. La capacidad calorífica especifica $\bar{C_P}$ se calcula de forma rápida como el valor de $C_p$ correspondiente al intermedio entre $T_1$ y $T_2$ ($\bar{C_P=\Delta H/ \Delta T$). \\

Para sustancias en una fase con variaciones entre dos temperaturas, al calcular los cambios de entalpía las condiciones de referencia se calcelan:

\begin{equation}
    \Delta H_{T_1-t_2}=\int_{T_1}^{T_2}C_P
dT=\bar{C_P}\cdot(T_2-T_1) \text{     }(J/kg)
\end{equation}


Para los sólido, los líquidos y los gases reales \textbf{la capacidad calorífica varia con la $T$} pero es una función continua de la T en la reagión entre las transiciones de fase.

\begin{equation}
    C_P=a+b\cdot T+c\cdot T^2
\end{equation}

Sustituyendo en la ecuación anterior:

\begin{equation}
    \Delta H=\int^{T_2}_{T_1}C_PdT= \int^{T_2}_{T_1} (a+b\cdot T+c\cdot T^2)dT=a\cdot(T_2-T_1)+\frac{b}{2}\cdot(T_2^2-T_1^2)+\frac{c}{3}\cdot(T_2^3-T_1^3) 
\end{equation}

\begin{equation}
    \bar{C_p}=\frac{\int_{T_1}^{T_2}C_PdT}{\int_{T_1}^{T_2}dT}=\frac{a\cdot(T_2-T_1)+\frac{b}{2}\cdot(T_2^2-T_1^2)+\frac{c}{3}\cdot(T_2^3-T_1^3) }{(T_2-T_1)}
\end{equation}
\\

\textbf{Variación de la entalpía con el cambio de fase y la temperatura}\\

Durante las transiciones de una fase a otra ($S\xrightarrow{} L$, $L\xrightarrow{}G$ y viceversa) ocurren grandes cambios en el valor de la entalpía de las sustancias, es el llamdo \textbf{calor latente}. En el caso de una sola fase la entalpía varia con la temperatura como se muestra en la figura \ref{fig:cambio fase}.

\begin{figure}[h!]
    \centering
    \includegraphics[width=0.75\linewidth]{IMG_0425.jpeg}
    \caption{Variación de la entalpía en función de la temperatura en un sistema de una sola fase.}
    \label{fig:cambio fase}
\end{figure}

Los cambios de entalpía que tienen lugar en una sola fase se conocen como cambios de \textbf{calor sensible}.\\

Los cambios de entalpía para las transiciones de fase se denominan \textbf{calor/entalpía de fusion} y \textbf{calor/entalpía de vaporización} que se corresponden al calor latente. El \textbf{calor de condensación } es el negativo del calor de vaporización y el \textbf{calor de sublimación} es el cambio de entalpía de la transición de sólido a vapor.\\

En un sistema de un componente con un cambio de estado $A\xrightarrow{}B$ a una temperatura $T_{A\xrightarrow{}B$ (figura \ref{fig: cambio fase 2}). La variación de entalpía entre el punto 1 y 2 se calcula como:

\begin{equation}
    H_{T_1}=H_{T_{ref}^A}+\bar{C_{p,A}}\cdot(T_1-T_{ref})
\end{equation}
\begin{equation}
    H_{T_2}=H_{T_{ref}^A}+\bar{C_{p,A}}\cdot(T_{A\xrightarrow{}B}-T_{ref})+\lambda_{A\xrightarrow{}B}+\bar{C_{p,B}}\cdot(T_2-T_{A\xrightarrow{}B})
\end{equation}

Donde $\lambda_{A\xrightarrow{}B}$ se corresponde al calor latente de la transición de fase. De modo que la difenrencia de entalpía entre 1 y 2 es:

\begin{equation}
    \Delta H_{T_1\xrightarrow{}T_2}^{A\xrightarrow{}B}=\bar{C_{p,A}}\cdot(T_{A\xrightarrow{}B}-T_{1})+\lambda_{A\xrightarrow{}B}+\bar{C_{p,B}}\cdot(T_2-T_{A\xrightarrow{}B})
\end{equation}

\begin{figure}[h!]
    \centering
    \includegraphics[width=0.5\linewidth]{IMG_0426.jpeg}
    \caption{Sistema de un componente para calculo de entalpía entre 1 y 2.}
    \label{fig: cambio fase 2}
\end{figure}


Cuando tenemos N componentes tenemso que tener en cuenta que :

\begin{itemize}
    \item Cada componente i tiene un calor específico en cada estado .($C_{p,i}^A,C_{p,i}^B$)
    \item En el cambio de estado, cada componente tiene su calor latente.($\lambda_{A\xrightarrow[]{}B,i}$
    \item Al ser una mezcla de componentes, hay que tener en cuenta la cantidad, concentración o proporción de cada componente ($x_i$)
\end{itemize}

Por lo que si consideramos N componentes a la misma $T_{inicial}$($T_1$) que cambian de estado y son calentados hasta la misma $T_{final}$($T_2$):

\begin{equation}
    \Delta H_{T_1\xrightarrow{}T_2}^{A\xrightarrow{}B}=\sum_{i=1}^Nx_i\cdot C_{p,i}^A\cdot(T_{A\xrightarrow{}B}-T_1)+\sum_{i=1}^Nx_i\cdot\lambda_{A\xrightarrow{}B,l}+\sum_{i=1}^Nx_i\cdot C_{p,i}^B\cdot (T_2-T_{A\xrightarrow{}B})
\end{equation}
\subsubsection{Balance entálpico con reacción química}

En el caso de que exista una reacción química se deben tener una serie de consideraciones:

\begin{itemize}
    \item Se debe tomar una \textbf{temperatura de referencia } que suele tomarse la tempereatura igual o inferior a la más bajas existente en cada corriente. Normalmente se toma $298K$ puesto que las entalpías de formación o entalpías de reacción se encuentran tabuladas a esta temperatura, como entalpías de reacción normal o estandar.
    \item Las variaciones de entalpía con la $P$ son muy pequeñas por lo que suele emplearse como referencia la $P $ media del sistema.
\end{itemize}

\begin{equation}
    H=H_{T_{ref}}+\int _{T_{Ref}}^TC_pdT=H_{T_{Ref)}}+\bar{C_p}\cdot(T-T_{Ref})
\end{equation}

Aplicaremos la \textbf{Lay de Hess} que es una alplicación del primer principio de la termodinámica a las reacciones químicas. El calor de reacción $Q$ es una función de estado.
\begin{equation}
    Q=\Vec{\Delta H_1}+\sum \Delta H_R^{T_{ref}} +\Vec{\Delta H_2}
\end{equation}

Donde la diferencia de calor sensible viene dada por:

\begin{equation}
    \Vec{\Delta H_1}=\sum_{i=1}^Nx_{i,1}\cdot \bar{C}_{p,i}\cdot (T_{ref}-T_1)
\end{equation}
\begin{equation}
    \Vec{\Delta H_2}=\sum_{i=1}^Nx_{i,1}\cdot \bar{C}_{p,i}\cdot (T_2-T_{ref)
\end{equation}

Y la variación de entalpía asociada a la reacción química:
\begin{equation}
    \sum \Delta H_R^{T_{ref}}=\sum_{i=1}^Nx_{i,2} \Delta H^{T_{ref}}_{f,i}-\sum_{i=1}^Nx_{i,1} \Delta H^{T_{ref}}_{f,i}
    \label{quimicas}
\end{equation}

De modo que el calor es positivo que gana calor del exterior ($Q>0$) y negativo si el sistema pierde calor cediendolo al exterior ($Q<0$). En cuanto a la entalpía asociada a la reacción correspondiente a la ecuación \ref{quimicas} la reacción se denomina \textbf{endotértmina } si $\sum \Delta H_R^{T_{ref}}>0$ y \textbf{exotérmica} si $\sum \Delta H_R^{T_{ref}}<0$. LA entalpía de reacción a $1atm$ y $298K$ se denomina Entalpía de reacción normal o estandar ($\Delta H_R^o$) y sus valores estab tabuladas.\\

Debemos introducir un nuevo concepto, la \textbf{entalpía de combustión}. Esta entalpía es la variación de entalpía producida en la combustión completa de un mol o una unidad de masa de un compuesto a presión y temperatura constantes.Conceptualmente es sinónimo de calor de combustión (siempre libera calor).
\begin{equation}
    \sum \Delta H_c^{T_{ref}}=\sum_{i=1}^N \Delta H^{T_{ref}}_{c,R}-\sum_{i=1}^N \Delta H^{T_{ref}}_{c,P}
    \label{quimicas}
\end{equation}