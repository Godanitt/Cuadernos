\chapter{Introducción al \textit{machine learning}}

El \textit{deep learning} o aprendizaje profundo permite a los modelos computacionales compuestos de varias capas de procesamiento apreder representaciones de datos con múltiples niveles de abstracción. Estso métodos han mejorado dramáticamente el estado del arte en el sentido del reconocimiento de voz, de imágenes, detección de objetos y otros muchos dominios como el descrubrimiento de fármacos y la genómica. El aprendizaje prfundo descubre patrones realmente complejos en grandes volúmenes de datos usando algoritmos de \textit{backpropagation} (retropropagación) que le indica a la máquina si debería cambiar sus parámetros internos usados para calcular las representaciones en cada una de las capas a partir de las representaciones de la capa anterior. Las redes convonucionales han traido además inovaciones en el campo del procesamiento de imágenes, video, discursos y audio, mientras que las redes recurrentes tienen un papel fundamental en el procesamiento de dato secuencial como pueden ser textos y discursos. 


\section{Introducción}

El aprendizaje máquina o \textit{machine learning}  ayuda a la sociedad en múltiples aspectos: desde búsquedas de infromación en la web, como filtros en redes sociales, recomendaciones en comercio electrónico... Los sistemas de aprendizaje máquina se usan paara identificar objetos en imágenes, transcribir discursos a texto plano o seleccionar los resultados relevantes en una búsqueda en internet. Cada vez más en estas aplicaciones usamos un tipo particular de técnicas llamdas aprendizaje profundo o \textit{deep learning}.

Las primeras técnicas de aprendizaje máquina estuvieron muy limitidas en su capacidad de procesar datos crudos. Incluso décadas después de su inicio, construir un patrón eficiencia de reconocimiento de datos (un sistema de aprendizaje máquina) requería una cuidadosa construcción, además de un gran dominio en la materia para diseñar tanto la extracción como la transformación de datos crudos (como podrían ser los valores de información de cada pixel de una imagen) en una represntación interna como podría ser un vector, a partir la cual el sistema de aprendizaje, normalmente un clasificador, podría detectar patrones en la entrada (\textit{input})

El aprendizaje de representaciones o \textit{represntation learning} es una sier de métodos que le permite a la máquina, a partir de datos crudos, descrubir representaciones necesitadas para la clasificación o detección. Los métodos de aprendizaje profundo son métodos de aprendizaje de represntaciónes con múltiples niveles de representación, obteniendo de un cojunto simple de modulos no lineales, los cuales transforman la representación a un solo nivel (empezando con los datos crudos) en una represntación mucho más abstracta. Con la composición de suficientes transformaciones, un conjunto complejo de funciones pueden llegar a ser aprendidas por la máquina. Cuando el fin de la máquina es la clasificación, capas de representación más altas amplificarán aspectos del \textit{input} que son importantes para la posterior discriminación, a la par que suprimirá las variaciones irrelevantes. Una imagen por ejemplo viene dada como una matriz de valores pixel, y la primera capa en general suele detectar o representar la ausencia o presencia de bordes, a través de cambios de color, luminosidad... La segunda capa detecta motivos/patrones en un conjunto particular de bordes, o variaciones pequeñas de los bordes. La tercera capa juntará todos los motivos en un conjunto mucho más grande que corresponderá con partes de objetos familiares. Las siguientes capas podrán detectar objetos como combinaciones de esas partes. El aspecto clave del aprendizaje profundo es que estas capas no están diseñadas por un humano, sino que se aprenden a partir de datos utilizando un procedimiento de aprendizaje de propósito general.

El aprendizaje profundo esta haciendo grandes avances en la resolución de problemas que hasta ahora resistían los mejores intentos de la inteligencia artificial. Se ha vuelvo una herramienta muy importante en el descubrimiento de estructuras intrincadas con grandes dimensiones y por tanto es apicable a numerosos campos de la ciencia, negocios y gobierno. Este método es el mejor en reconocimiento de imágenes y de voz, y además predice la activación de drogas, analiza información de aceleradores de partículas, recostruye circuitos cerebrales... 

\section{Aprendizaje supervisado}

\section{Retropopagación en el entrenamiento de arquitectura multicapa}

\section{Redes neuronales convolucionales}

\section{Representaciones distribuidas y procesamiento de lengauje}

\section{Redes neuronales recurrentes}

\section{El futuro del \textit{deep learning}}

El aprendizaje sin supervisión ha revivido el interés en el \textit{deep learning}, aunque ha sido opacado por el éxito rotundo el aprendizaje supervisado. Es de esperar que el aprendizaje no supervisado se haga cada vez mas importante, sobretodo teniendo en cuenta que tanto los humanos como el resto de los animales aprendemos sin una supervisión. 

La visión humana es un proceso activo que muestrea secuencialmente el conjunto óptico de forma inteligente y específica para cada tarea, utilizando una fóvea pequeña y de alta resolución con un entorno grande y de baja resolución. Esperamos que la mayor parte del desarrollo de algoritmos de visión venga dado por sistemas que son entrenados \textit{end-to-end} con combinaiones de ConvNets y RNNs  que usen el aprendizaje reeforzado para decidir a donde mirar. Sistemas que combinenen aprendizaje profundo y lenguaje reforzado son todavía muy nuevos, pero ya han demostrado su efectividad en el uso de sistemas de visión pasiva en la clasficiación de elementos, además de ser capaces de jugar o aprender a jugar a diferentes videojuegos.

El lenguaje natural es otro área en la que el aprendizaje profundo tendrá un impacto en los próximos años, ya que esperamos que los procesos RNN permitan entender frases o incluso documentos enteros, lo que nos lelvará a procesos mucho mejores en tanto en cuanto sean capaces de aprender estrategias para seleccioanr partes del texto. Finalmente, también se espera que procesos mayores en la inteligencia artificial permitirá pasar a través de sistemas que combinen la representación en leguaje y un razonamiento complejo.