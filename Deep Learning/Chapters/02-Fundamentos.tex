\chapter{Fundamentos del aprendizaje máquina}

\section{Clasficiación Binaria}

Para poder avanzar en el tema, primero tenemos que definir algunos de los términos que hemos usado de manera intuitiva: aprendizaje, experiencia, mejora y tarea. Empecemos con la tarea de la clasificación binaria. 

Supongamos un problema abstracto en el cual tenemos los datos de la siguiente forma: 

\begin{equation}
    D = \{ (x_1,y_1),(x_2,y_2),\ldots, (x_n,y_n) \}
\end{equation}
donde $x\in \mathbb{R}^n$ e $y=\pm 1$. A pesar de que $D$ represente todos los datos, nosotros solo tenemos acesso a un subconjunto de ellos, $S\in D$. Usando $S$, nuestra tarea será generar un proceso computacional que implemente la función $f:x\to y$  tal que podemos usar $f$ para predecir los datos no pertenecientes a $S$, esto es,  $x_i, y_i\notin S$, tal que $f(x_i)=y_i$. Si dennotamos  $U$ como el conjunto contario a $S$ ($U=\bar{S}$), podemos suponer que una medida de la eficiencia/calidad de nuestro sistema para realizar esta tarea vendrá dada por una función error sobre los datos no vistos: 

\begin{equation}
    E(f,D,U) = \frac{\sum_{x_i,y_i \in U} [f(x_i)\neq y_i] }{|U|}
\end{equation} 
Medimos el rendimiento de la tarea usando la función error $E(f,D,U)$ sobre los datos no conocidos $U$. EL tamaño de los datos conocidos $S$ será nuestra ``experiencia''. Lo que nosotros queremos hacer es crear algoritmos que generen funciones $f$ (comunmente denominadas modelos).  En general a los valores $x$ se le llaman valores de entradas o \textit{inputs}, mientras que a $y$ valores de salida o \textit{outputs}.

Como en cualquier otra disciplina, las características computacionales es un aspecto relevante, pero además de tener en cuenta los costos computacionales, tendremos que tratar de hallar el modelo $f$ que tenga el menor errror posible $E(f,D,S)$ con el menor $|S|$ posible. 

\begin{Ejemplo}{Web de Comercio Electrónico}
    Supongamos que queremos implementar en una página web una página personalizada para usuarios registrados, que les muestre los productos que más dispuestos estarían a comprar. La web registra la información de los usuarios. ¿Cómo se relaciona este problema con la clasificación binaria? 

    Esta claro que el problema relaciona productos con personas. Centremonos entonces en la relación entre un solo producto y un solo usuario. Lo que nosotros queremos saber es si va a ser comprado. Podemos denotar el estado ``deseado'' con $y=+1$ mientras que el estado ``no deseado'' por $y=-1$. Lo siguiente que debemos hacer es coger información del producto (tipo de producto...) y la información histórica de búsqueda y compra del usuario, representado todo esto en un vector $x \in \mathbb{R}^n$. Basados en esta información y en el mapeo conocido $\{(x_1,y_1),(x_2,y_2),\ldots,(x_n,y_n)\}$, entonces podemos generar la función modelo $f:x\to y$ a partir lo cual podemos determinar que productos querrá comprar y llenarle la página con esta información. Podríamos medir lo ``buena'' que es esta medida en función de si el usuario compra los productos mostrados, evaluando luego la función error con los datos predichos y si los compra o no. 

\end{Ejemplo}


\section{Regresión}

\section{Regularización}