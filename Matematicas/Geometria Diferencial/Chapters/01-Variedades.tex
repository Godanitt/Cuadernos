\chapter{Variedades y Campos vectoriales}

La geometía diferencia letudia las propiedades geométricas como $S \in \mathbb{R}^3$, o más general (hiper-superficies). Hay dos tipos de propiedades, las intrínsecas, que son las más importantes, y las que dependen de la propia dimensión. Las propiedades geométricas de un objeto son, por ejemplo, la simetría, distancia, curvatura, pendiente... 

\begin{Definicion}
    Una \textbf{variedad} es un espacio topológico $\Mcal$ que puede ser cubierto por subconjuntos abiertos $\Ucal_a$, tal que $\Mcal=\cup_{a} \Ucal_a$,t tal que para cada $\Ucal_a$ existe una aplicación $\phi_{\Ucal_T}: \Ucal_a \to \Ucal \subset \mathbb{R}^m$ dessde $\Ucal_i$ a un un subconjutno $\phi_{\Ucal_a}(\Ucal_a) \in \mathbb{R}^m$. De manera naif, podemos decir que una variedad de $m$ dimensión es un espacio topológico que localmente es como $\mathbb{R}^m$. 
\end{Definicion}

\begin{Ejemplo}{Círculo $S^1$}
    El círculo $S^1$ se puede describir con dos cartas. Lo normal sería pensar que se puede describir con una, ya que el espacio $\phi(p\in S^1) \to \varphi = (0,2\pi) \in R^1$. Sin embargo no es posible debido a que $\phi^{-1}(0) = \phi^{-1} (2\pi)$ pero $0 \neq 2 \pi$. Las cartas que lo describen serían: 

    \begin{equation}
    \phi_{\Ucal_1}: \varphi \in (-\varepsilon,\pi + \varepsilon) \qquad  
    \phi_{\Ucal_2}: \chi \in (-\varepsilon,\pi + \varepsilon) 
    \end{equation}
    
\end{Ejemplo}