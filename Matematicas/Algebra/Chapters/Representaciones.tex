\chapter{Representaciones}

\section{Homomorfismos}

Un homomorfismo es una aplicación $f$ de un conjunto $A$ a otro conjunto $B$ (escribimos $f:A\to B$) que preserva alguna estrucutra interna. En particular, estamos interesados en la estructura de grupo de $A$ By $B$.

\begin{Definicion}
    Sean $A$ y $B$ dos grupos y $f:A \to B$ una aplicación. Decimos que $f$ es un \textbf{homomorfismo} cuando verifica que para cualesquiera $a_1,a_2\in A$
    \begin{equation}
        f(a_1a_2) = f(a_1)f(a_2)
    \end{equation}    
\end{Definicion}
Cuando $B$ coincide con $A$ decimos que $f$ es un \textbf{endorfismo}. Es importante recordar la definición de imagen de $f$ y núcleo de $f$:

\begin{Definicion}
    Defimos como \textbf{imagen} de $f$ ($\Imagen f$) a:
    \begin{equation}
        \Imagen f := \{b\in B / b=f(a) \text{ para algún } a \in A\}
    \end{equation}
\end{Definicion}
\begin{Definicion}
    Defimos como \textbf{núcleo} (\textit{kernel}, del alemán) de $f$ ($\Kernel f$) a:
    \begin{equation}
        \Kernel f := \{a \in A / f(a) = e_B \in B \}
    \end{equation}
\end{Definicion}
Luego además tenemos los diferentes tipos de aplicaiocnes y homomorfismos. Una aplicación es \textit{biyectiva} cuando $\Kernel f = e_A$. Una aplicación es \textit{suprayectiva} caundo $\Imagen f =B$. En el caso de ser suprayectiva y biyectiva simultáneamente estamos ante una aplicación \textit{biyectiva}. Si además $f$ es un homomorfismo tenemos un \textit{isomoforfismo}. Y si tenemos un endorfismo e isomorfismo estamos ante un \textit{automorfismo}.


\section{Reducibilidad}