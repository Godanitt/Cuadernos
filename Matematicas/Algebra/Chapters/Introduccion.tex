\chapter{Introducción}

\section{Definición de Grupo}

\begin{Definicion}
    Un \textbf{grupo} es un conjunto de elementos $G = \{g_1,g_2 ...\}$ que dotados de una ley de composición $*$, tal que a cada par ordenado  $g_i,g_j\in G$ se le asigna otro elemento que satisface las siguientes propiedades: 
    
    \begin{itemize}
        \item \textbf{Cierre:} la ley de composición $*$ es interna, es decir, si $g_i,g_j\in G$ entonces $g_i*g_j \in G$. 
        \begin{equation}
            \forall g_i,g_j \in G \quad g_i * g_j \in G
        \end{equation}
        \item \textbf{Asociatividad:} para todo $g_i,g_j,g_k\in G$:
        \begin{equation}
            g_i*(g_j*g_k) = (g_i*g_j)*g_k
        \end{equation}
        \item \textbf{Elemento neutro:} existe un único elemento denotado como $e$, con la propiedad que $\forall g_i \in G$ tal que
        \begin{equation}
            e * g_i = g_i * e = g_i
        \end{equation}
        \item \textbf{Elemento inverso:} para cada $g_i$ existe un único elemento $g_i^{-1}$ tal que
        \begin{equation}
            g_i^{-1} * g_i = g_i * g_i^{-1} = e
        \end{equation}
    \end{itemize}
    En general $g_i g_j \equiv g_i * g_j$.
\end{Definicion}

Dado un grupo, podemos ampliarlo un \textbf{álgebra}. Un álgebra es un conjunto de elemento que forman un espacio vectorial sobre un cuerpo (por ejemplo $\mathbb{R}$ o $\mathbb{C}$), de forma que, junto a la adición, se define una operación de multiplicación que verifica los postulados que definen un grupo, excepto que el cero del álgebra no tiene inverso. Así pues, por ejemplo, dado un grupo $G$ con elementos $g_i$ ($i=0,1,\ldots,h$), las combinaciones lineales $\sum_{i=1}^{h}c_ig_i$, de elementos del grupo con coeficientes en el cuerpo, forman el álgebra del grupo. El producto se define por distrivuiddad como

\begin{equation}
    \pqty{\sum_{i=1}^{h}c_ig_i} \pqty{\sum_{j=1}^{h}c_jg_j} = \sum_{i=1}^{h}\sum_{j=1}^{h}c_ic_jg_ig_j
\end{equation}
que por ser $g_ig_j$ es un elemento del grupo, es un nuevo elemento del álgebra. 

\begin{Definicion}
    Decimos que $G = \{g_1,g_2 ...\}$ es un \textbf{grupo abeliano} si su ley de composición $*$ sea conmutativa, es decir: 

    \begin{equation}
        g_i * g_j = g_j * g_i \quad \forall g_i,g_j \in G 
    \end{equation}
\end{Definicion}

\section{Estrucutra de los Grupos}

\begin{Definicion}
    Dos elementos $g_1$ y $g_2$ de un grupo $G$ son \textbf{conjugados} si existe un tercer elemento $g$ tal que $g_2 = gg_1g^{-1}$. Decir que $g_1$ y $g_2$ son conjugaos se puede expresar como $g_1 \sim g_2$.s
\end{Definicion}

Las relaciones de conjugación entre dos o más elementos, es una relación de equivalencia $\sim$. Efectivamente, podemos afirmar que se cumplan las propiedades: 

\begin{itemize}
    \item Reflexiva, tal que $a\sim a$ en virtud de lo que $a = ea = ae$.
    \item Simétrica, tal que $a=gbg^{-1} \Rightarrow b = gag^{-1}$.
    \item Transivitva, tal que $a = gbg^{-1}$ y $b=hch^{-1} \rightarrow a = (gh)c(gh)^{-1}$.
\end{itemize}

\begin{Definicion}
    Un \textbf{subgrupo} $H$ de un grupo $G$ es un subconjunto de $G$ ($H \leq G$) que a su vez forma un grupo bajo la misma ley de composición de $G$. 
\end{Definicion}

Cuando $G$ es finito, una definición equivalente sería afirmar que $H$ es un subgrupo de $G$ cuando es cerrado, bajo la ley de composición de $G$: 

\begin{equation}
    \forall h_1,h_2 \in H \leq G \Rightarrow h_1 * h_2 \in H 
\end{equation}
La asociatividad, la existencia de un elemento neutro y de un inverso son propiedades \textit{heredadas} de $G$. En todo grupo $G$ hay dos ejemplos triviales: que $H=\{e\}$ y $H=G$. Cualquier otro subgrupo que no sean estos se llaman \textit{subgrupos propios}.

\begin{Definicion}
    Dados un elemento $g\in G$ y un subgrupo $H=\{h_1,h_2...\}$ de $G$, definimos el subgrupo \textbf{coset izquierda} al conjutno de elementos obtenidos al multiplicar $g$ por todos los elementos de $H$, y se define como
    \begin{equation}
        gH = \Bqty{gh_1,gh_2,...}
    \end{equation}
\end{Definicion}

La pertenencia de dos elementos a un mismo coset define de nuevo una relación de equivalencia. 

\begin{Definicion}
    Un \textbf{subgrupo normal} es un subgrupo $H$ verifica
    \begin{equation}
        g H g^{-1} = H \qquad \forall g \in G
    \end{equation}
\end{Definicion}

