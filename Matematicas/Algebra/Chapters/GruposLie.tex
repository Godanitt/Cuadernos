\chapter{Grupo de Lie}

\section{Introducción}

En los grupos continuos los elementos pueden parametrizarse en un entorno de cualquier punto mediante un conjunto de variables reales. Escribimos entonces para un elemento genérico $g(x_1,x_2,...,x_d) = g(\xn)$. Si $d$ es el número de mínimos parámetros encesiarios para alcanzar cualquier elemento, hablamos de un grupo de \textit{dimensión d}. 

Es evidente que no podmeos escribir una tabla de multiplicar en el mismo sentido que para un grupo finito. Si el producto de $g(\xn)$ por $g(\yn)$ es $g(\zn)$ tenemos que:

\begin{equation}
    g(x_1,x_2,...,x_d) g(y_1,y_2,...,y_d) = g(z_1,z_2,...,z_d)
\end{equation}
entonces los parámetros $z_1,...,z_d$ son funciones de $x_i$ y $y_i$. Es decir, la tabla de multiplicación consta de $n$ funciones relaes de $2d$ argumentos, $z_i = f_i(\xn,\yn)$ tal que $i=1,...,d$. 

Las propiedades que definen un grupo imponen restricciones sobre las posibles funciones $f_i$. La más severa es la que proviene de la asociatividad: 

\begin{equation}
    (g(\xn) g(\yn)) g(\zn) = g(\xn) (g(\yn) g(\zn))
\end{equation}
válida para todos los valores de $x,y,z$. 


\begin{Definicion}
    Un \textbf{grupo de Lie} es un grupo continuo, en el cual las funciones $f_i$ que expresan la multiplicación, a parte de satisfacer los requisitos que provienen de las propiedades del grupo, son $C^{\infty}$ (contínuas e infinitavamente derivables).
\end{Definicion}

\section{Representaciones de Grupos de Lie}

\subsection{Grupos Unitarios}

\subsection{Grupos Ortogonales}

\section{Estructura Local de los Grupos de Lie}

\subsection{Generadores Infinitesimales de un Grupo de Lie}

\subsection{Álgebras de Lie y Grupos de Lie}

\begin{Definicion}
    Un \textbf{álgebra de Lie} $\Lcal$ de dimensión $d\geq 1$ es un espacio vectorial real de dimensión $d$, dotado de una operación interna llamada \textbf{corchete de Lie} $[,]:\Lcal \times \Lcal \to \Lcal$, definida para todo par $u,v\in\Lcal$ y que satisface las siguientes propiedades:
    \begin{itemize}
        \item \textbf{Cierre}: $[u,v] \in \Lcal$ para todo $u,v\in \Lcal$.
        \item \textbf{Antisimetría}: $[u,v] = - [v,u]$.
        \item \textbf{Linealidad}: $[\alpha u + \beta v,w] = \alpha [u,w] + \beta [v,w]$ para $a,b\in \mathbb{R}$.
        \item \textbf{Identidad de Jacobi}: $[u,[v,w]] + [w,[u,v]] + [v,[w,u]] = 0$.
    \end{itemize}
\end{Definicion}

El concepto de álebra de Lie es una definición abstracta que en cada caso requiere una definición concreta para el corchete de Lie subyacente. 

Dada una base $L_1,...,L_d$ para un álgebra de Lie viene especificada por un conjunto de $d^3$ números $f_{ij}^k$ denominados \textbf{constantes de estrucutra} con respecto a la base $\{L_i\} i=1,...,d$ que se definen según la siguiente expresión: 

\begin{equation}
    [L_i,L_j] = \sum_{k=1}^d f^{k}_{ij} L_k  \qquad i,j = 1,...,d
\end{equation}
Estos números no son independientes como se deduce de las propiedades de \textit{antisimetría} e \textit{indentidad de jacobi}. 

Frentes a cambios de base ${L_i \to \tilde{L}_i} $ con $i=1,...,d$, las constantes de estrucutra tarnsforman con dos índices covariantes y uno contravariante. Es decir, si $S$ especifica el cambio, tenemos que 

\begin{equation}
    \tilde{L}_j = S^i L_j
\end{equation}

\begin{Teorema}
    A cada grupo de Lie lineal, G, le corresponde un álgebra de Lie $\Gcal$ de la misma dimensión. De forma más precisa, si $\Gcal$ tiene dimensión $d$, entonces los generadores infinitesimales $L-1,...,L_d$ forman una base de $\Gcal$. 
\end{Teorema}

\section{Representaciones de Grupos y Álgebras de Lie}